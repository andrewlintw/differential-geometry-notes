\documentclass[12pt]{article}
\usepackage{mystyle}

% Custom math commands
\newcommand{\manifold}[1][M]{\mathcal{#1}}
\newcommand{\tvec}[2]{\left(\partial\indices{_{#1}}\right)_{#2}}
\newcommand{\tfld}[1]{\partial\indices{_{#1}}}
\newcommand{\lder}[1]{\mathcal{L}_{#1}}
\newcommand{\eucForm}[2]{\Lambda^{#1}(\mathbb{R}^{#2})}
\newcommand{\form}[2]{\Lambda^{#1}(\manifold[#2])}
\newcommand{\bnd}[1]{\partial\manifold[#1]}
\newcommand{\irrshape}[3]{% r, dr, sample
    \draw plot 
        [smooth cycle, samples=#3, domain={1:#3}] 
        ({(\x*360/#3+8*(2*rnd-1))}:{#1+#2*(2*rnd-1)})
}
\DeclareMathOperator{\idd}{id}
\DeclareMathOperator{\sgn}{sgn}
\DeclareMathOperator{\grad}{grad}
\DeclareMathOperator{\curl}{curl}
\DeclareMathOperator{\divv}{div}
\DeclareMathOperator{\supp}{supp}

\begin{document}
    \section{Differentiable Manifolds}
    \subsection{Definition}
    \subsubsection{Coordinate Charts}
    \begin{definition}[Coordinate Charts]\label{def:coordinate-charts}
        An \(m\)-dimensional, \(m \neq \infty\) coordinate chart on a topological 
        space \(\manifold\) is a pair 
        \[
        (U, \phi)
        \begin{cases}
            U \subseteq \manifold[M], U~\text{open}\\
            \phi : U \to \mathbb{R}^{m}, \phi~\text{homeomorphism}
        \end{cases}
        \]
        \begin{center}
            \begin{tikzpicture}
    \begin{scope}
        \irrshape{2}{0.1}{6};
    \end{scope}
    \node at (-1, 1) {$\manifold[M]$};
    \begin{scope}[xshift=0.5cm, yshift=-0.5cm]
        \irrshape{1}{0.05}{5}[dashed];
    \end{scope}
    \node at (1, -1) {$U$};
    \begin{scope}[xshift=4cm, yshift=-3cm]
        \begin{scope}[xshift=0.5cm, yshift=0.5cm]
            \irrshape{1}{0.2}{5}[dashed];
        \end{scope}
        \draw[->] (0, -0.5) -- (0, 2) node[above] {$\mathbb{R}^m$};
        \draw[->] (-0.5, 0) -- (2, 0);
    \end{scope}
    \draw[->] (1, -0.5) to [out=-10, in=105] node [auto] {$\phi$} (3.8, -2.5) ;
\end{tikzpicture}
        \end{center}
    \end{definition}
    \begin{remark}
        If \(U=\manifold[M]\), then we say the coordinate chart \(\phi\) is 
        globally defined; if not, then it is locally defined. Few manifolds 
        have globally defined property.
    \end{remark}
    \begin{remark}
        The basic method of studying manifolds is to analyze it in the familiar 
        Euclidean space via coordinate charts.
    \end{remark}
    \newpage
    \begin{definition}[Overlap Function]\label{def:overlap-function}
        Let \((U_1, \phi_1), (U_2, \phi_2)\) be a pair of \(m\)-dimensional 
        coordinate charts with \(U_1 \cap U_2 \neq \varnothing\). Then the 
        overlap function is defined as 
        \[
        \phi_2\circ\phi_1 ^{-1} : \phi_1(U_1 \cap U_2) \subseteq \mathbb{R}^m 
        \to \phi_2(U_1 \cap U_2) \subseteq \mathbb{R}^m.
        \]
        \begin{center}
            \begin{tikzpicture}
    \irrshape{2.5}{0.5}{6};
    \node at (-1, 1.5) {$\manifold[M]$};
    \begin{scope}[xshift=0.5cm]
        \irrshape{1}{0.05}{5}[dashed];
    \end{scope}
    \node at (1, -1.3) {$U_1$};
    \begin{scope}[xshift=-0.5cm]
        \irrshape{1}{0.05}{5}[dashed];
    \end{scope}
    \node at (-1.3, -1) {$U_2$};
    \node at (0, 0) {$U_1 \cap U_2$};
    \begin{scope}[xshift=4cm, yshift=-3cm]
        \begin{scope}[xshift=0.5cm, yshift=0.5cm]
            \irrshape{1}{0.2}{5}[dashed];
        \end{scope}
        \draw[->] (0, -0.5) -- (0, 2) node[above] {$\mathbb{R}^m$};
        \draw[->] (-0.5, 0) -- (2, 0);
    \end{scope}
    \draw[<-] (1, -0.5) to [out=-10, in=105] node [auto] {$\phi_1 ^{-1}$} (3.8, -2.5) ;
    \begin{scope}[xshift=-4cm, yshift=-3cm]
        \begin{scope}[xshift=0.5cm, yshift=0.5cm]
            \irrshape{1}{0.2}{5}[dashed];
        \end{scope}
        \draw[->] (0, -0.5) -- (0, 2) node[above] {$\mathbb{R}^m$};
        \draw[->] (-0.5, 0) -- (2, 0);
    \end{scope}
    \draw[->] (-1, -0.5) to [out=190, in=75] node [auto] {$\phi_2$} (-3.8, -2.5);
    \draw[->] (3, -2.5) to [out=190, in=-10] 
        node [auto] {$\phi_2\circ\phi_1 ^{-1}$} (-2, -2.5);
\end{tikzpicture}
        \end{center}
    \end{definition}
    \begin{definition}[Atlas]\label{def:atlas}
        An \(m\)-dimensional atlas on \(\manifold[M]\) is a family of 
        \(m\)-dimensional coordinate charts \((U_i, \phi_i), i \in I\) s.t. 
        \begin{enumerate}
            \item \(\manifold[M] = \bigcup_{i \in I}^{} U_i\).
            \item Each overlap function \(\phi_j \circ \phi_i ^{-1}, i, j \in I\) 
            is \(C ^{\infty}\).
        \end{enumerate}
    \end{definition}
    \begin{definition}[Differentiable Manifolds]\label{def:differentiable-manifolds}
        An \(m\)-dimensional differentiable manifold is a topological space 
        \(\manifold[M]\) equipped with an atlas.
    \end{definition}
    \begin{remark}
        We didn't define a differentiable manifold by regulating the 
        differentiability of the coordinate charts themselves. That's because 
        differentiation is not defined on a manifold, so we need to rely on 
        Euclidean spaces.
    \end{remark}
    \newpage
    \subsection{Dimension of a Manifold}
    \begin{remark}
        Consider a manifold that consists of a rod attached to a disk. The dimension 
        is not same everywhere. We give a criterion on how to describe 
        such a scenario.
    \end{remark}
    \begin{theorem}[Invariance of Domain]\label{thm:invariance-domain}
        For all \(A, B \subseteq S^n\), if \(\exists~f : A \to B\) homeomorphic 
        and \(B \in \tau_{S^n}\), then \(A \in \tau_{S^n}\) too.
    \end{theorem}
    \begin{remark}
        \cref{thm:invariance-domain} is an early theorem in algebraic topology.
    \end{remark}
    \begin{corollary}[Dimension is Well-defined]\label{cor:manifold-dim-well-defined}
        Given \(U \in \tau_{\mathbb{R}^n}, U' \in \tau_{\mathbb{R}^{n'}}\), and if 
        \(\exists~f : U \to U'\) homeomorphic, then \(n = n'\).
    \end{corollary}
    \begin{proof}
        If \(n = n'\), it is trivially true. \par
        If \(n < n'\), embed \(\mathbb{R}^n\) to \(\mathbb{R}^{n'}\) by 
        \(f : \vec{x} \mapsto (\vec{x}, \vec{0})\). Via stereographic projection, 
        we can map homeomorphically
        \begin{align*}
            \phi &: U  \mapsto V \subseteq S^{n'}, \\
            \phi'&: U' \mapsto V' \subseteq S^{n'}.
        \end{align*}
        Since the compositions are also homeomorphic, we see \(V\) and \(V'\) are 
        homeomorphic. However, \(V'\) is not an open subset of \(\mathbb{R}^{n'}\) because 
        of the \(0\)'s, contradicting \cref{thm:invariance-domain}. 
        \textreferencemark
    \end{proof}
    \begin{remark}
        Since the definition of a differentiable manifold requires every overlap 
        function to be diffeomorphic, if \(U \cap U' \neq \varnothing\), their 
        dimensions must be equal via the above corollary. We can bypass this by 
        demanding \(U \cap U' = \varnothing\), as in the rod and disk case.
    \end{remark}
    \begin{corollary}\label{cor:injection-homeomorphism}
        If \(g : V \to \mathbb{R}^n\) is a continuous injection and \(V \in 
        \tau_{\mathbb{R}^n}\), then \(g(V)\) is homeomorphic to \(V\), and 
        \(g(V) \in \tau_{\mathbb{R}^n}\).
    \end{corollary}
    \begin{proof}
        On \(g(V)\), \(g\) is surjective and therefore a homeomorphism. Use 
        stereographic projection and the result is obvious.
    \end{proof}
    \subsection{Coordinate Functions}
    \begin{definition}[Coordinate Functions]\label{def:coordinate-functions}
        The coordinate functions are the (Euclidean) components of coordinate.
        % \[
        \begin{align*}
            \phi &: U \to \mathbb{R}^m & p \mapsto &\phi(p), \\
            \phi^\mu &: U \to \mathbb{R} & ~\text{s.t.}~ &\phi(p) = \begin{pmatrix}
            \phi^1(p) \\ \vdots \\ \phi^m(p)
            \end{pmatrix}.
        \end{align*}
        % \]
        An alternative notation is 
        \[
        x^\mu := \phi^\mu.
        \]
    \end{definition}
    \begin{remark}
        There are (Euclidean) projection functions, 
        \[
        u^\mu : \mathbb{R}^m \to \mathbb{R}.
        \]
        But I think mention it will cause a lot of confusion. Just remember in 
        the future when we say \(\frac{\partial }{\partial u^\mu}\), we are 
        referring to the Euclidean partial derivative wrt the \(\mu\)-th 
        component.
    \end{remark}
    \begin{definition}[Jacobian Matrix of Coordinate Transformation]\label{def:jacobian-matrix}
        Let \(\manifold[M]\) be a \(C^\infty\) manifold of \(\dim \manifold[M]=m\).
        Choose two coordinate functions 
        \[
        \begin{aligned}
            \phi : U_\phi \to V_\phi,& \; p \mapsto (x^1(p), \dots, x^m(p)), \\
            \psi : U_\psi \to V_\psi,& \; p \mapsto (y^1(p), \dots, y^m(p)).
        \end{aligned}
        \]
        Define the Jacobian matrix of coordinate transformation to be
        \[
        J := \begin{pmatrix}
        \frac{\partial y^1}{\partial x^1} & \cdots & \frac{\partial y^1}{\partial x^m} \\
        \vdots                            & \ddots & \vdots                            \\
        \frac{\partial y^m}{\partial x^1} & \cdots & \frac{\partial y^m}{\partial x^m} \\
        \end{pmatrix}
        \]
        In short, 
        \[
        \begin{aligned}
            J\indices{^\nu_\mu} := \frac{\partial y^\nu}{\partial x^\mu}, \\
            (J^{-1})\indices{^\nu_\mu} := \frac{\partial x^\nu}{\partial y^\mu}, 
        \end{aligned}
        \]
        where \(\frac{\partial y^\nu}{\partial x^\mu} := 
        \frac{\partial (y^\nu \circ \phi ^{-1})}{\partial x^\mu}\).
    \end{definition}
    \newpage
    \subsection{Manifold With Boundary}
    \subsubsection{Generalized Coordinate Charts}
    \begin{definition}[Generalized Coordinate Charts]\label{def:half-plane-chart}
        A generalized coordinate chart allows chart that 
        \[
        \phi : U \to \phi(U) \subseteq (-\infty, 0] \times \mathbb{R}^{n-1},
        \]
        \(U\) is open, and \(\phi\) is homeomorphic. \par
    \end{definition}
    \begin{remark}
        Essentially, this allows a chart to map to "half planes". In this case, 
        even if a set \(\phi(U)\) contains \(\{0\} \times \mathbb{R}^{n-1}\) and 
        therefore not open in the Euclidean topology of \(\mathbb{R}^n\), it 
        is still considered open in the product topology of 
        \((-\infty, 0] \times \mathbb{R}^{n-1}\).
    \end{remark}
    \begin{definition}[Manifold With Boundary]\label{def:bounded-manifold}
        A manifold with boundary is a manifold whose atlas consists of generalized 
        coordinate charts.
    \end{definition}
    \begin{definition}[Boundary Points of a Manifold]\label{def:manifold-boundary}
        For all \(p \in \manifold[M]\) is a boundary point of a manifold with 
        boundary \(\manifold[M]\) if \(\exists~\phi_\alpha \in \Phi\) atlas 
        s.t. \(\phi_\alpha^1(p)=0\). \\
        The set of all boundary points of \(\manifold[M]\) is denoted \(\bnd{M}\).
    \end{definition}
    \subsubsection{Boundary is Well-defined}
    \begin{remark}
        A natural question regarding \cref{def:manifold-boundary} is that, the 
        definition only asks for existence, but it does not guarantee the existence 
        of 
        \[
        \exists~\phi_\alpha, \phi_\beta ~\text{s.t.}~ \phi_\alpha^1(p) = 0, 
        \phi_\beta^1(p) \neq 0.
        \]
        We resolve this in the following.
    \end{remark}
    \begin{theorem}\label{thm:manifold-boundary-well-defined}
        Suppose \(U, U'\) are open sets in the product topology \((-\infty, 0] 
        \times \mathbb{R}^{n-1}\), and \(\exists~ f : U \to U'\) homeomorphic. Then 
        \[
        f(U \cap (\{0\}\times \mathbb{R}^{n-1})) = U' \cap (\{0\}\times 
        \mathbb{R}^{n-1}).
        \]
    \end{theorem}
    \begin{proof}
        We show instead that 
        \[
        f(U \cap ((-\infty, 0)\times \mathbb{R}^{n-1})) = 
        U' \cap ((-\infty, 0)\times \mathbb{R}^{n-1}).
        \]
        Via \cref{cor:injection-homeomorphism}, we see 
        \(f(U \cap ((-\infty, 0)\times \mathbb{R}^{n-1}))\) must be an open set in 
        the Euclidean topology of \(\mathbb{R}^n\). Therefore, it cannot contain 
        \(\{0\}\times \mathbb{R}^{n-1}\).
    \end{proof}
    \begin{corollary}[Boundary is Well-defined]\label{cor:manifold-boundary-well-defined}
        \(\forall p \in \manifold[M]\), if \(\exists~ \phi_\alpha^1(p) = 0\), 
        then \(\forall \phi \in \Phi\) atlas, \(\phi^1(p)=0\).
    \end{corollary}
    \begin{proof}
        Make use of the fact that \(\phi\circ\phi_\alpha ^{-1}\) is a homeomorphism.
    \end{proof}
    \subsection{Submanifolds}
    \subsubsection{Definition}
    \begin{definition}[Local Submanifold]\label{def:local-submanifold}
        \(A \subseteq \manifold[M]\) is a submanifold of codimension \(r\) around 
        \(p \in A \subseteq \manifold[M]\) if there exists a chart 
        \(\phi : U \to \phi(U) \in \Phi\) atlas s.t. \(p \in U\) and 
        \[
        \phi(U \cap A) = \phi(U) \cap (\mathbb{R}^{m-r} \times 
        \{\underbrace{0, \dots, 0}_{r\text{ zeroes}}\}).
        \]
    \end{definition}
    \begin{definition}[Submanifold]\label{def:submanifold}
        If \(A\) is a \(C^k\) local submanifold of \(\manifold[M]\) of codimension 
        \(r\) around every \(p \in \manifold[M]\), then we say \(A\) is a 
        submanifold of \(\manifold[M]\) of codimension \(r\).
    \end{definition}
    \newpage
    \section{Tangent Spaces}
    \begin{remark}
        The definition of manifold do not require the entity to be embeded in 
        a higher dimensional space. Therefore, the traditional view of tangency 
        is not valid here.
        
    \end{remark}

    \subsection{Curves and Vectors}
    \begin{definition}[Curve]\label{def:curve}
        A curve on \(\manifold[M]\) is a \(C^\infty\) map,
        \[
        \sigma : (-\epsilon, \epsilon) \to \manifold[M].
        \]
    \end{definition}
    \begin{definition}[Curve Tangency]\label{def:curve-tangency}
        Two curves \(\sigma_1, \sigma_2\) are tangent at \(p \in \manifold[M]\) 
        if 
        \begin{enumerate}
            \item \(\sigma_1(0)=\sigma_2(0)=p\).
            \item \(\left.\frac{d}{dt}(x^i \circ \sigma_1(t))\right|_{t=0}
             = \left.\frac{d}{dt}(x^i \circ \sigma_2(t))\right|_{t=0},~1 \leq 
             i \leq m\) .
        \end{enumerate}
    \end{definition}
    \begin{remark}
        Written more compactly, 
        \[
        \left.\frac{d}{dt}(\phi\circ\sigma_1)\right|_{t=0}
        =\left.\frac{d}{dt}(\phi\circ\sigma_2)\right|_{t=0}
        \]
        
    \end{remark}
    \begin{definition}[Tangent Vectors]\label{def:tangent-vector-curve}
        A tangent vector at \(p \in \manifold[M]\) is an equivalence class of 
        curves where the equivalence relation is that they are tangent. It will 
        be denoted as 
        \[
        v = [\sigma].
        \]
    \end{definition}
    \begin{definition}[Tangent Space]\label{def:tangent-space-curve}
        The tangent space \(T_p\manifold[M]\) at point \(p\) is the set of all 
        tangent vectors at point \(p\).
    \end{definition}
    \begin{definition}[Tangent Bundle]\label{def:tangent-bundle-curve}
        The tangent bundle \(T\manifold[M]\) is 
        \[
        T\manifold[M] := \bigcup_{p \in \manifold[M]}^{} T_p\manifold[M].
        \]
    \end{definition}
    
    % \subsection{Addition and Scalar Multiplication}
    % \begin{definition}[Addition and Scalar Multiplication]\label{def:addition-and-scalar-multiplication-curve}
    %     Let \(v_1 = [\sigma_1], v_2 = [\sigma_2] \in T_p\manifold[M]\), and 
    %     \(r \in \mathbb{R}\). Then 
    %     define 
    %     \[
    %     \begin{aligned}
    %         v_1+v_2 &:= [\phi ^{-1}\circ(\phi\circ\sigma_1+\phi\circ\sigma_2)], \\
    %         rv_1 &:= [\phi ^{-1} \circ (r\phi \circ \sigma_1)].
    %     \end{aligned}
    %     \]
    % \end{definition}
    % \begin{theorem}
    %     The definition \cref{def:addition-and-scalar-multiplication-curve}
    %     is well-defined. That is, they are independent of the 
    %     choice of chart \((U, \phi)\) and \(\sigma_1, \sigma_2\) as long as 
    %     \(v_1=[\sigma_1]\) and \(v_2=[\sigma_2]\).\par
    %     Therefore, \(T_p\manifold[M]\) is a real vector space.
    % \end{theorem}
    % \begin{proof}
    %     Let \(v_1 = [\sigma_1] = v_1' := [\tau_1], 
    %     v_2 = [\sigma_2] = v_2' := [\tau_2]\). 
    %     First check (1) of \cref{def:curve-tangency}, 
    %     \[
    %     \begin{aligned}
    %         (rv_1+v_2)(0) &= (\phi ^{-1} \circ (r\phi\circ\sigma_1(0) + 
    %         \phi\circ\sigma_2(0))) \\
    %         &= (\phi ^{-1} \circ (r\phi\circ\tau_1(0) + 
    %         \phi\circ\tau_2(0))) \\
    %         &= (rv_1'+v_2')(0),
    %     \end{aligned}
    %     \]
    %     since \(\phi\circ\sigma_1(0) = \phi\circ\tau_1(0) = \phi(p)\) by 
    %     equivalence, and the same for \(\sigma_2\). \par
    %     Now consider 
    %     \[
    %     \begin{aligned}
    %         \left.\frac{d}{dt}(\phi\circ(rv_1+v_2))\right|_{t=0}
    %         &=\left.\frac{d}{dt}(r\phi\circ\sigma_1+\phi\circ\sigma_2)
    %         \right|_{t=0}\\
    %         &=r\left.\frac{d}{dt}(\phi\circ\sigma_1)\right|_{t=0}+
    %         \left.\frac{d}{dt}(\phi\circ\sigma_2)\right|_{t=0}\\
    %         &=r\left.\frac{d}{dt}(\phi\circ\tau_1)\right|_{t=0}+
    %         \left.\frac{d}{dt}(\phi\circ\tau_2)\right|_{t=0}\\
    %         &=\left.\frac{d}{dt}(\phi\circ(rv_1'+v_2'))\right|_{t=0},
    %     \end{aligned}
    %     \]
    %     since \(\left.\frac{d}{dt}(\phi\circ\sigma_1)\right|_{t=0}
    %     =\left.\frac{d}{dt}(\phi\circ\tau_1)\right|_{t=0}\) by equivalence, and 
    %     the same for \(\sigma_2\).
    % \end{proof}

    \subsection{Curves and Derivation}
    \begin{definition}[Directional Derivative]\label{def:directional-derivative-curve}
        For any \(f : \manifold[M] \to \mathbb{R}\) s.t. \(f \in C^\infty\), we 
        define
        \[
        v(f) := \left.\frac{d}{dt}(f \circ \sigma(t))\right|_{t=0},
        \]
        where \(v = [\sigma]\).
    \end{definition}
    \begin{theorem}
        The definition \cref{def:directional-derivative-curve} is well-defined.
        That is, \(v(f)\) is independent of the curve \(\sigma\) chosen as well 
        as \(v=[\sigma]\).
    \end{theorem}
    \begin{proof}
        Let \(v_1 = [\sigma_1] = [\sigma_2] = v_2\). Then 
        \[
        \begin{aligned}
            v_1(f) &= \left.\frac{d}{dt}(f \circ \sigma_1)\right|_{t=0}, \\
            v_2(f) &= \left.\frac{d}{dt}(f \circ \sigma_2)\right|_{t=0}, \\
        \end{aligned}
        \]
        \[
        \left.\frac{d}{dt}(\phi\circ\sigma_1)\right|_{t=0}
            =\left.\frac{d}{dt}(\phi\circ\sigma_2)\right|_{t=0}.
        \] \par
        Then
        \[
        \begin{aligned}
            v_1(f) &= \left.\frac{d}{dt}
            (\underbrace{(f\circ\phi ^{-1})}_{\mathbb{R}\leftarrow \mathbb{R}^m} \circ 
            \underbrace{(\phi \circ \sigma_1)}_{\mathbb{R}^m\leftarrow \mathbb{R}})
            \right|_{t=0} \\
            &= \left.(f \circ \phi ^{-1})'\circ(\phi\circ\sigma_1)\cdot
            (\phi\circ\sigma_1)'\right|_{t=0} \\
            &= \left.(f \circ \phi ^{-1})'\circ(\phi\circ\sigma_2)\cdot
            (\phi\circ\sigma_2)'\right|_{t=0} \\
            &= v_2(f),
        \end{aligned}
        \]
        since \(\phi\circ\sigma_1(0) = \phi\circ\sigma_2(0) =\phi(p)\), and 
        \((\phi\circ\sigma_1)'=(\phi\circ\sigma_2)'\) by equivalence.
    \end{proof}

    \subsection{Isomorphism with Euclidean Spaces}
    \begin{definition}["Straight Lines"]\label{def:map-straight}
        Choose a coordinate chart \(\phi \in \Phi\) atlas near \(p\). 
        For all \(v \in \mathbb{R}^m\), we define 
        \[
        \gamma_v^\phi(t) := \phi ^{-1}(\phi(p)+tv).
        \]
        In simple words, it is such a curve on manifold that it is a straight line 
        on maps.
    \end{definition}
    \begin{theorem}[Isomorphism with Straight Lines]\label{thm:curve-straight-iso}
        Let \(p \in \manifold[M]\), \(p \in \phi \in \Phi\). For any curve 
        \(\gamma\) passing through \(p\), \(\exists!~v \in \mathbb{R}^m\) s.t. 
        \(\gamma\) is tangent to \(\gamma_v^\phi\), where \(v\) can be explicitly 
        given by \((\phi \circ \gamma)'(0)\). \\
        In other words, the map
        \[
        \begin{aligned}
            \ell_p^\phi : & \mathbb{R}^m \to T_p\manifold[M] \\
                          & v \mapsto [\gamma_v^\phi]
        \end{aligned}
        \]
        is a bijection with inverse 
        \[
        \begin{aligned}
            (\ell_p^\phi) ^{-1} : & T_p\manifold[M] \to \mathbb{R}^m \\
                          & [\gamma] \mapsto (\phi \circ \gamma)'(0).
        \end{aligned}
        \]
    \end{theorem}
    \begin{proof}
        It is almost by definition that \(\gamma\) is tangent to 
        \((\phi \circ \gamma)'(0)\). \par
        Suppose \(v_1, v_2\) satisfies \(\gamma\) is tangent to 
        \(\gamma_{v_1}^\phi, \gamma_{v_2}^\phi\). Then they are tangent too. So 
        \[
        (\phi \circ \gamma_{v_1}^\phi)'(0) = (\phi \circ \gamma_{v_2}^\phi)'(0).
        \]
        Therefore, \(v_1=v_2\).
    \end{proof}
    \begin{definition}[Alternative Definition for Addition]\label{def:curve-add-alt}
        For \(v_1, v_2 \in T_p\manifold[M]\), we define addition via the isomorphism 
        \cref{thm:curve-straight-iso}, 
        \[
        v_1 + v_2 := (\ell_p^\phi)^{-1}(\ell_p^\phi(v_1)+\ell_p^\phi(v_2)).
        \]
    \end{definition}
    \begin{definition}[Alternative Definition of Basis Tangent Vectors]\label{def:basis-alt}
        \[
        \tvec{\mu}{p} := \ell_p^\phi(e^\mu).
        \]
    \end{definition}
    \begin{theorem}
        \[
        \tvec{\mu}{p}(x^\nu) = \delta\indices{_\mu^\nu}.
        \]
    \end{theorem}
    \begin{proof}
        By \cref{def:basis-alt}, 
        \[
        \tvec{\mu}{p} := \ell_p^\phi(e^\mu) = \phi ^{-1}(\phi(p)+te^\mu), 
        \]
        where \(e^\mu\) is the column vector with \(\mu\)-th component 1, others 
        0. Then 
        \[
        \begin{aligned}
            \tvec{\mu}{p}(x^\nu)&=(x^\nu\circ\phi ^{-1}(\phi(p)+te^\mu))'(0) \\
            &=(x^\nu \circ \phi ^{-1})'(\phi(p))e^\mu.
        \end{aligned}
        \]
        But since \(x^\nu\) has only one component, \((x^\nu \circ \phi ^{-1})'\) 
        is a row vector with \(\nu\)-th component 1, others 0. So the result 
        follows immediately.
    \end{proof}
    \begin{theorem}[Linear Independence of Basis Tangent Vectors]\label{thm:linear-indep-basis}
        The basis tangent vectors \(\tvec{\mu}{p}, 1 \leq \mu \leq \dim 
        \manifold[M]\) are linear independent.
    \end{theorem}
    \begin{proof}
        Suppose \(a^\mu\tvec{\mu}{p}=0\). Then 
        \[
        a^\mu\tvec{\mu}{p}(x^\nu)=a^\mu\delta\indices{^\nu_\mu}=0(x^\nu)=0.
        \]
        So \(a^\nu=0\).
    \end{proof}
    \begin{theorem}[Coordinate Expansion of Tangent Vectors]\label{thm:tangent-vector-coordinates}
        For all \(v \in T_p\manifold[M]\), we have 
        \[
        v=v^\mu\tvec{\mu}{p},
        \]
        where Einstein notation was used, and \(v^\mu = v(x^\mu)\).
    \end{theorem}
    \begin{remark}
        A natural question is that whether two charts behave "the same" if 
        \(U_1 \cap U_2 \neq \varnothing\). Under this perspective, the criterion 
        is clear: we need only to check whether a straight line is still a straight 
        line in another chart, which is true indeed. It also produces the coordinate 
        transformation formula for free. See below.
    \end{remark}
    \begin{theorem}[Coordinate Transformation of Straight Lines]\label{thm:straight-overlap}
        Choose \(\phi, \psi \in \Phi\), \(\phi : U_1 \to \mathbb{R}^m\), 
        \(\psi : U_2 \to \mathbb{R}^m\) s.t. \(U_1 \cap U_2 \neq \varnothing\) and 
        \(p \in U_1 \cap U_2\). Let the corresponding straight line isomorphisms 
        \(\ell_p^\phi, \ell_p^\psi\). Then \((\ell_p^\psi)^{-1} \circ \ell_p^\phi\) 
        is a linear isomorphism. \\
        Let the local coordinates induced by \(\phi\) be \(x^1, \dots, x^m\), 
        \(\psi\) be \(y^1, \dots, y^m\), then 
        \((\ell_p^\psi)^{-1} \circ \ell_p^\phi\) can be expressed in terms of 
        the Jacobian \cref{def:jacobian-matrix} \(J\indices{^\nu_\mu}:=
        \left.\frac{\partial y^\nu}{\partial x^\mu}\right|_{\phi(p)}\), namely, 
        \[
        (\ell_p^\psi)^{-1} \circ \ell_p^\phi(v) = Jv.
        \]
    \end{theorem}
    \begin{proof}
        \[
        \begin{aligned}
            (\ell_p^\psi)^{-1} \circ \ell_p^\phi(v) &= 
            (\ell_p^\psi)^{-1}(\phi ^{-1}(\phi(p)+vt)) \\
            &= ((\psi \circ \phi ^{-1})(\phi(p)+vt))'(0) \\
            & = Jv \\
        \end{aligned}
        \]
    \end{proof}
    \begin{definition}[Contravariancy and Covariancy]\label{def:contravariant-covariant}
        Let \(\manifold[M]\) be a \(m\)-dimensional \(C^\infty\) manifold. 
        Choose two coordinate charts
        \[
        \begin{aligned}
            \phi : U_\phi \to V_\phi,& \; p \mapsto (x^1(p), \dots, x^m(p)), \\
            \psi : U_\psi \to V_\psi,& \; p \mapsto (y^1(p), \dots, y^m(p)).
        \end{aligned}
        \]
        and the corresponding Jacobian matrix \cref{def:jacobian-matrix} 
        \(J\indices{^\nu_\mu} := \left.\frac{\partial y^\nu}{\partial x^\mu}
        \right|_{\phi(p)}\).
        We define covariancy to be anything that transforms like 
        \[
        (\text{new})_\nu = (\text{old})_\mu(J^{-1})\indices{^\mu_\nu}.
        \]
    \end{definition}
    \begin{corollary}[Contravariancy of Tangent Vectors]\label{cor:vec-coord-trans}
        The basis vectors transform covariantly, whereas the components of 
        vectors transform contravariantly. That is, choose two overlapping 
        coordinate charts \((x^1, \dots, x^m)\) and \((y^1, \dots, y^m)\) and define 
        their Jacobian matrix \cref{def:jacobian-matrix} \(J\indices{^\nu_\mu} := 
        \left.\frac{\partial y^\nu}{\partial x^\mu}\right|_{\phi(p)}\), 
        \[
        \begin{cases}
            \left(\frac{\partial }{\partial y^\nu}\right) = 
            \left(\frac{\partial }{\partial x^\mu}\right)(J^{-1})\indices{^\mu_\nu}
            & \text{(covariant)} \\
            v^{\nu'} = J\indices{^{\nu'}_\mu}v^\mu & \text{(contravariant)}
        \end{cases}
        \]
    \end{corollary}
    \subsection{Pushforward}
    \subsubsection{Definition and Linearity}
    \begin{remark}
        The pushforward \(h_* : T_p\manifold[M] \to 
        T_{h(p)}\manifold[N]\) of a specific function \(h : \manifold[M] \to 
        \manifold[N]\) can be thought of as local linearization of the function.
        
    \end{remark}
    \begin{definition}[Pushforward]\label{def:pushforward}
        Given a function \(h : \manifold[M] \to \manifold[N]\) and 
        \(v \in T_p\manifold[M]\), then we define the pushforward 
        \(h_* : T_p\manifold[M] \to T_{h(p)}\manifold[N]\) by 
        \[
        h_*(v) := [h \circ \sigma],~v=[\sigma].
        \]
    \end{definition}
    \begin{theorem}
        The pushforward operation \cref{def:pushforward} is well-defined. That is, 
        \(h_*(v_1) = h_*(v_2)\) if \(v_1 = [\sigma_1] = [\sigma_2] = v_2\).
    \end{theorem}
    \begin{theorem}[Algebraic Definition of Pushforward]\label{thm:algebraic-pushforward}
        The definition of pushforward \cref{def:pushforward} is equivalent to the 
        following: let \(h : \manifold[M] \to \manifold[N]\), \(h_* : 
        D_p\manifold[M] \to D_{h(p)}\manifold[M]\) is defined by,
        \[
        (h_*v)(f) := v(f\circ h).
        \]
    \end{theorem}
    \begin{proof}
        (\textrightarrow) 
        \[
        \begin{aligned}
            h_*(v)(f) = [h \circ \sigma](f) 
            &= \left.\frac{d}{dt}(f\circ h\circ\sigma(t))\right|_{t=0} \\
            &= \left.\frac{d}{dt}((f\circ h)\circ\sigma(t))\right|_{t=0} \\
            &:= v(f\circ h).
        \end{aligned}
        \] \par
        (\textleftarrow) This direction is similar.
    \end{proof}
    \begin{theorem}[Linearity of Pushforward]\label{thm:linearity-of-pushforward}
        The pushforward map \(h_* : T_p\manifold[M] \to T_{h(p)}\manifold[N]\) 
        is linear.
        \[
        h_*(rv_1+v_2) = rh_*(v_1) + h_*(v_2).
        \]
    \end{theorem}
    \begin{proof}
        (Using \cref{def:pushforward})
        Let \(p \in (U, \phi) \subseteq \manifold[M]\), and 
        \(h(p) \in (V, \psi) \subseteq \manifold[N]\). Choose \(\phi\) s.t. 
        \(\phi(p) = 0\).
        It is obvious that \(h_*(rv_1+v_2)(0) = (rh_*(v_1)+h_*(v_2))(0) = h(p)\). 
        \par
        Consider 
        \[
        \begin{aligned}
            \left.\frac{d}{dt}
            \underbrace{(\psi \circ h_*(rv_1+v_2))}_{\mathbb{R}^n \leftarrow 
            \manifold[N] \leftarrow \mathbb{R}}\right|_{t=0}
            &= \left.\frac{d}{dt}(\underbrace{\psi\circ h\circ (\phi ^{-1}}
            _{\mathbb{R}^n \leftarrow \manifold[N] \leftarrow \manifold[M] 
            \leftarrow \mathbb{R}^m}\circ
            \underbrace{(r\phi\circ\sigma_1+\phi\circ\sigma_2)}_{\mathbb{R}^m 
            \leftarrow \manifold[M] \leftarrow \mathbb{R}}))\right|_{t=0} \\
            &= \left.(\psi \circ h \circ \phi ^{-1})'\circ 
            (r\phi\circ\sigma_1+\phi\circ\sigma_2) \cdot (r\phi\circ\sigma_1 + 
            \phi\circ\sigma_2)'\right|_{t=0} \\
            &= \left.(\psi \circ h \circ \phi ^{-1})'(0) \cdot 
            ((r\phi\circ\sigma_1)' + (\phi\circ\sigma_2)')\right|_{t=0}.
        \end{aligned}
        \]
        And 
        \[
        \begin{aligned}
            \left.\frac{d}{dt}
            (\underbrace{\psi}_{\mathbb{R}^n\leftarrow} \circ 
            \underbrace{(rh_*(v_1)+h_*(v_2))}_{\manifold[N]\leftarrow 
            \mathbb{R}})
            \right|_{t=0}
            &= \left.\frac{d}{dt}
            \underbrace{(r\psi\circ h \circ \sigma_1 + \psi \circ h \circ \sigma_2)}_
            {\mathbb{R}^n \leftarrow \manifold[N] \leftarrow \manifold[M] \leftarrow 
            \mathbb{R}}
            \right|_{t=0} \\
        \end{aligned}
        \]
        \[
        \begin{aligned}
        =&(\underbrace{r\psi\circ h\circ \phi ^{-1}}_
        {\mathbb{R}^n\leftarrow\manifold[N]\leftarrow\manifold[M]
        \leftarrow\mathbb{R}^m} \circ \underbrace{\phi \circ \sigma_1}_
        {\mathbb{R}^m\leftarrow\manifold[M]\leftarrow\mathbb{R}} + 
        \psi\circ h\circ \phi ^{-1} \circ \phi \circ \sigma_2
        )'|_{t=0} \\
        =&(r(\psi\circ h\circ \phi ^{-1})'\circ(\phi\circ\sigma_1)
        \cdot(\phi\circ\sigma_1)')|_{t=0} \\
        &+((\psi\circ h\circ \phi ^{-1})'\circ(\phi\circ\sigma_2)
        \cdot(\phi\circ\sigma_2)')|_{t=0} \\
        =&(\psi\circ h\circ \phi ^{-1})'(0)\cdot(r(\phi\circ\sigma_1)' + 
        (\phi\circ\sigma_2)')|_{t=0}.
        \end{aligned}
        \]
        So we see the two are equal. \par
        (Using \cref{thm:algebraic-pushforward})
        \[
        \begin{aligned}
            (h_*(rv_1+v_2))(f) &= (rv_1+v_2)(f\circ h) \\
            &= rv_1(f\circ h)+v_2(f\circ h) \\
            &= r(h_*v_1)f+(h_*v_2)f.
        \end{aligned}
        \]
        \par
        (Using straight line isomorphism: \cref{def:curve-add-alt})
        \begin{center}
            \begin{tikzpicture}
    \node (tpm) at (0, 0) {$T_p\manifold[M]$};
    \node (thpn) at (4, 0) {$T_{h(p)\manifold[N]}$};
    \node (rm) at (0, -2) {$\mathbb{R}^m$};
    \node (rn) at (4, -2) {$\mathbb{R}^n$}
        edge[<-] (rm);
    \draw[->] (tpm.east) -- (thpn.west) node[midway, above] {$h_*$};
    \draw[<-] (tpm.south) -- (rm.north) node[midway, left] {$\ell_p^\phi$};
    \draw[->] (thpn.south) -- (rn.north) node[midway, left] 
        {$(\ell_{h(p)}^\psi)^{-1}$};
\end{tikzpicture}
        \end{center}
        Since \(\ell_p^\phi\) and \((\ell_{h(p)}^\psi)^{-1}\) are linear, to prove 
        \(h_*\) is linear, we need only to show \((\ell_{h(p)}^\psi)^{-1} \circ 
        h_* \circ \ell_p^\phi\) is linear. \par
        \[
        \begin{aligned}
            &(\ell_{h(p)}^\psi) ^{-1} \circ h_* \circ \ell_p^\phi(v) \\
            =&(\ell_{h(p)}^\psi)^{-1} \circ h_*([\phi ^{-1}(\phi(p)+tv)]) \\
            =&(\ell_{h(p)}^\psi)^{-1}([h\circ\phi ^{-1}(\phi(p)+tv)]) \\
            =&(\psi \circ h \circ \phi ^{-1}(\phi(p)+tv))'(0) \\
            =&(\psi \circ h \circ \phi ^{-1})'(\phi(p))v.
        \end{aligned}
        \]
    \end{proof}
    \begin{theorem}[Associativity of Pushforwards]\label{thm:pushforward-associativity}
        Given manifolds \(\manifold[M], \manifold[N], \manifold[P]\) and 
        \(h : \manifold[M] \to \manifold[N]\), \(k : \manifold[N] \to 
        \manifold[P]\), then 
        \[
        (k\circ h)_*=k_*\circ h_*.
        \]
    \end{theorem}

    \subsubsection{Jacobian}
    \begin{theorem}[Local Representative of Pushforward]\label{thm:pushforward-local}
        Let \(\dim \manifold[M]=m, \dim \manifold[N]=n\), \(h : \manifold[M] 
        \to \manifold[N]\), \(\{x^1, \dots, x^m\}\) be the local coordinates of 
        \(\manifold[M]\) around \(p\), and \(\{y^1, \dots, y^n\}\) be the local 
        coordinates of \(\manifold[N]\) around \(h(p)\). Then 
        \[
        h_*v = \sum_{\mu=1}^{m} \sum_{\nu=1}^{n} \tvec{\nu}{h(p)} 
        \left.\frac{\partial h^\nu}{\partial x^\mu}\right|_{p}v^\mu,
        \]
        where \(J\indices{^\nu_\mu}:=\left.\frac{\partial h^\nu}{\partial x^\mu}\right|_{p}
        :=\tvec{\mu}{p}(y^\nu\circ h)\) is the Jacobian matrix.
    \end{theorem}
    \begin{proof}
        First expand \(v\) in terms of local coordinates and use linearity, 
        \[
        h_*v = h_*(v^\mu \tvec{\mu}{p}) = v^\mu h_*(\tvec{\mu}{p}).
        \]
        Expand the result in local coordinates of \(\manifold[N]\), 
        \[
        h_*(\tvec{\mu}{p}) = \left(h_*\tvec{\mu}{p}\right)^\nu\tvec{\nu}{h(p)}.
        \]
        Using \cref{thm:algebraic-pushforward}, 
        \[
        \begin{aligned}
            \left(h_*\tvec{\mu}{p}\right)^\nu 
            &= \left(h_*\tvec{\mu}{p}\right)\circ y^\nu \\
            &= \tvec{\mu}{p}(y^\nu \circ h) \\
            &:= \tvec{\mu}{p}h^\nu.
        \end{aligned}
        \]
        So, 
        \[
        h_*(\tvec{\mu}{p}) = \tvec{\mu}{p}h^\nu\tvec{\nu}{h(p)}.
        \]
        And, 
        \[
        h_*v = v^\mu\tvec{\mu}{p}h^\nu\tvec{\nu}{h(p)}.
        \]
    \end{proof}
    \begin{theorem}[Using Curve to Pushforward]\label{thm:curve-pushforward}
        Given \(c : (-\epsilon, \epsilon) \to \manifold[M]\) a curve, and choose 
        the coordinate chart of \(\mathbb{R}\) to be the identity, then 
        \[
        c_*\left(\frac{d}{dt}\right)_0=[c] \in T_p\manifold[M].
        \]
    \end{theorem}
    \begin{proof}
        First we clarify what is \(\left(\frac{d}{dt}\right)_0\). Since on the 
        trivial manifold \(\mathbb{R}\) there is only one coordinate, namely 
        \(t\), we need not specify the number. Also, considering our functions 
        are scalar valued \(f : \manifold[M] \to \mathbb{R}\), this motivates us 
        to write "total differential". \par
        For all \(f \in C^\infty\), 
        \[
        c_*\left(\frac{d}{dt}\right)_0f=\left(\frac{d}{dt}\right)_0(f\circ c).
        \]
        Since the coordinate chart is the identity, 
        \[
        \begin{aligned}
            \left(\frac{d}{dt}\right)_0(f\circ c) 
            &= \left.\frac{d}{dt}(f\circ c\circ I)\right|_{I(t)=0} \\
            &= \left.\frac{d}{dt}(f\circ c)\right|_{t=0} \\
            &=[c]f.
        \end{aligned}
        \]
    \end{proof}
    \newpage
    \section{Formal Differential Form}
    \begin{remark}
        In this section, we follow a local, coordinate approach. We focus on the 
        requirements that makes form a form. We will postpone the realization of 
        differential forms.
    \end{remark}
    \subsection{Euclidean Spaces}
    \begin{definition}[Formal Differential Form on \(\mathbb{R}^m\)]\label{def:formal-form}
        A formal differential \(k\)-form on \(\mathbb{R}^m\) is composed of \(m^k\) 
        functions, arranged in the form
        \[
        \omega = \sum_{1 \leq i_1, \dots, i_k \leq m}^{} 
        \omega \indices{_{i_1}_\dots_{i_k}} dx^{i_1} \wedge \dots \wedge dx^{i_k}.
        \]
        We require,
        \begin{enumerate}
            \item All the functions \(\omega \indices{_{i_1}_\dots_{i_k}}\)
            are \((-\infty, 0]\times\mathbb{R}^{m-1} \to (-\infty, 0]\times\mathbb{R}^{m-1}\) and \(C^\infty\).
            \item The operation \(+\) is commutative and associative, just like the 
            usual addition.
            \item The operation \(\wedge\) is associative and distributes over 
            \(+\), resembling the usual multiplication.
            \item \(\wedge\) also has anticommutativity, 
            \[
            \begin{cases}
            dx^i \wedge dx^j = -dx^j \wedge dx^i &\forall i \neq j\\
            dx^i \wedge dx^i = 0
            \end{cases}
            \]
        \end{enumerate}
    \end{definition}
    \begin{definition}[Equality of Formal Differential k-forms]\label{def:formal-form-equality}
        Let two formal \(k\)-forms \(\omega, \eta\) be 
        \[
        \begin{aligned}
            \omega &= \omega\indices{_{i_1}_\dots_{i_k}} 
                dx^{i_1} \wedge \dots \wedge dx^{i_k}\\
            \eta &= \eta\indices{_{i_1}_\dots_{i_k}} 
                dx^{i_1} \wedge \dots \wedge dx^{i_k}
        \end{aligned}
        \]
        We say they are "equal", denoted \(\omega \equiv \eta\), iff 
        \[
        \sum_{\sigma \in S_k}^{} \omega\indices{_{\sigma(i_1)}_\dots_{\sigma(i_k)}}
        =\sum_{\sigma \in S_k}^{} \eta\indices{_{\sigma(i_1)}_\dots_{\sigma(i_k)}}
        \;\forall 1 \leq i_1 < \dots < i_k \leq m.
        \]
    \end{definition}
    \begin{remark}
        \begin{enumerate}
            \item For a formal differential \(k\)-form \(\omega = dx^1\wedge dx^2\), 
            \(\omega_{12} = 1\), but \(\omega_{21} = 0\). The \(m^k\) components in 
            the definition just presents a general form, so that it includes the 
            scenario \(\omega' = dx^1\wedge dx^2 - dx^2 \wedge dx^1\). If you insist 
            on arranging indices even in the definition, then \(\omega'\) would not 
            satisfy the definition, which is wierd.
            \item For demonstration of equality, consider the following example in 
            \(\mathbb{R}^3\),
            \[
            \begin{aligned}
                \omega &= dx^1\wedge dx^2 + dx^2\wedge dx^3 \\
                \eta   &= -dx^2 \wedge dx^1 + dx^2 \wedge dx^3.
            \end{aligned}
            \]
            Choose \(i_1 = 1, i_2 = 2\), and 
            \[
            S_3 = \{\underbrace{\idd, (1\;2\;3), (1\;3\;2)}_{\text{even}}
            , \underbrace{(1\;2), (2\;3), (1\;3)}_{\text{odd}}\}.
            \]
            Then, in order, 
            \[
            \begin{aligned}
                &\sum_{\sigma \in S_n}^{} (\sgn \sigma)\omega \\
                =&\textcolor{red}{\omega_{12}}+\textcolor{blue}{\omega_{23}}
                +\textcolor{green}{\omega_{31}}-\textcolor{red}{\omega_{21}}
                -\textcolor{green}{\omega_{13}}-\textcolor{blue}{\omega_{32}} \\
                =&\textcolor{red}{1}+\textcolor{blue}{1}
                +\textcolor{green}{0}-\textcolor{red}{0}
                -\textcolor{green}{0}-\textcolor{blue}{0}.
            \end{aligned}
            \]
            Notice how all permutations of a given component (paired in color) 
            appears exactly once in this relation, and the sign is fixed 
            accordingly by \(\sgn \sigma\).
            \item Viewed this way, we see in the language of formal differential 
            forms, 
            \[
            \begin{aligned}
                \omega &= dx^1\wedge dx^2 \\
                \eta &= -dx^2\wedge dx^1 \\
                \nu &= dx^1\wedge dx^2 - dx^2\wedge dx^1,
            \end{aligned}
            \]
            \(\omega \equiv \eta\), since \(\omega_{12} - \omega_{21} = 1 = 
            \eta_{12} - \eta_{21}\). But \(\omega \not\equiv \nu\), since 
            \(\nu_{12} - \nu_{21} = 2\).
        \end{enumerate}
    \end{remark}
    \subsection{Operations}
    \begin{definition}[Exterior Product]\label{def:formal-exterior-product}
        Let two formal \(k\)-forms be 
        \(\omega = \omega_{i_1\dots i_k} dx^{i_1} \wedge \dots \wedge dx^{i_k}\), 
        \(\eta = \eta_{j_1\dots j_l} dx^{j_1} \wedge \dots \wedge dx^{j_l}\). Then 
        \[
        \omega\wedge \eta := \omega_{i_1\dots i_k}\eta_{j_1\dots j_l}
        dx^{i_1}\wedge\dots\wedge dx^{i_k} \wedge dx^{j_1} \wedge \dots \wedge 
        dx^{j_l}.
        \]
        Which is equivalent \cref{def:formal-form-equality} to, 
        \[
        \sum_{1 \leq s_1 < \dots < s_{k+l} \leq m}^{} \left(
            \sum_{S}^{} (\sgn\sigma)\omega_{i_1\dots i_k}\eta_{j_1\dots j_l}
        \right)dx^{s_1}\wedge \dots\wedge dx^{s_{k+l}}.
        \]
        Where \(S\) is all the combinations of \(S_1 = \{i_1, \dots, i_k, j_1, \dots, 
        j_l\}\) that is equal to \(S_2 = \{s_1, \dots, s_{k+l}\}\), and the order does 
        not matter. \(\sigma\) is the function \(S_1 \to S_2\).
    \end{definition}
    \begin{remark}
        Example. Let \(\dim \manifold[M]=6\), and 
        \[
        \begin{aligned}
            \omega &= \omega_{12}dx^1\wedge dx^2 + \omega_{21} dx^2 \wedge dx^1 \\
            \eta   &= \eta_{456} dx^4\wedge dx^5\wedge dx^6.
        \end{aligned}
        \]
        Then 
        \[
        \omega\wedge \eta = (\omega_{12}\eta_{456} 
        - \omega_{21}\eta_{456}) dx^{1, 2, 4, 5, 6}
        \]
    \end{remark}
    \begin{definition}[Pullback]\label{def:formal-pullback}
        Let \(f : U \subseteq \mathbb{R}^m \to V \subseteq (-\infty, 0] \times 
        \mathbb{R}^{n-1}\) is \(C^\infty\). Choose local coordinates on 
        \(U\) to be \(x = (x^1, \dots, x^m)\), on \(V\) to be \(y = (y^1, \dots, 
        y^n)\). Define \(f^\nu := y^\nu \circ f\).\\
        Let \(\omega = \omega_{j_1\dots j_k} dy^{j_1}\wedge \dots \wedge dy^{j_k}\) 
        be a formal \(k\)-form on \(U\). Then 
        \[
        f^*\omega := (\omega_{j_1\dots j_k}\circ f)
        \frac{\partial f^{j_1}}{\partial x^{i_1}}\dots
        \frac{\partial f^{j_k}}{\partial x^{i_k}}dx^{i_1}\wedge \dots\wedge dx^{i_k}.
        \]
    \end{definition}
    \begin{remark}
        \begin{enumerate}
            \item The motivation is just somehow \(\omega\circ f\). Chain the
            component function with \(f\), and express the (resulting) coordinates 
            as \(dy^{\nu} = \frac{\partial f^{\nu}}{\partial x^{\mu}}dx^{\mu}\).
            \item It is possible that \(U \subseteq V\). Pulling back onto a subset 
            is essentially a "limitation" on \(\omega\).
            \item It is also possible that \(U \subseteq \partial V\), i.e. \(U\) is 
            the boundary of \(V\). \(f : p \mapsto (0, p)\) is just the 
            immersion map in that case.
        \end{enumerate}
    \end{remark}
    \begin{definition}[Exterior Differentiation]\label{def:formal-exterior-diff}
        Let a formal \(k\)-form be
        \(\omega = \omega_{i_1\dots i_k} dx^{i_1} \wedge \dots \wedge dx^{i_k}\). 
        Then 
        \[
        d\omega := \left(\frac{\partial \omega_{i_1\dots i_k}}{\partial x^{i_0}}
        dx^{i_0}\right) 
        \wedge dx^{i_1} \wedge \dots \wedge dx^{i_k}.
        \]
    \end{definition}
    \begin{remark}
        Let \(U \subseteq \mathbb{R}^3\).
        \begin{enumerate}
            \item Consider a formal 0-form, i.e. \(f \in C^\infty\). Then 
            \[
            df = \frac{\partial f}{\partial x}dx+\frac{\partial f}{\partial y}dy 
            + \frac{\partial f}{\partial z}dz \rightarrow \nabla f.
            \]
            \item Consider a formal 1-form \(\omega = Pdx+Qdy+Rdz\). Then 
            \[
            \begin{aligned}
                d\omega &= \left(\frac{\partial P}{\partial y}dy+
                \frac{\partial P}{\partial z}dz\right)\wedge dx + 
                \left(\frac{\partial Q}{\partial x}dx+
                \frac{\partial Q}{\partial z}dz\right)\wedge dy + 
                \left(\frac{\partial R}{\partial x}dx+
                \frac{\partial R}{\partial y}dy\right)\wedge dz \\
                &=\left(\frac{\partial R}{\partial y}-
                \frac{\partial Q}{\partial z}\right)dy\wedge dz + 
                \left(\frac{\partial R}{\partial x}-
                \frac{\partial P}{\partial z}\right)dx\wedge dz + 
                \left(\frac{\partial Q}{\partial x}-
                \frac{\partial P}{\partial y}\right)dx\wedge dy \\
                &\rightarrow \curl \omega.
            \end{aligned}
            \]
            \item Consider a formal 2-form \(\eta = A dy\wedge dz + B dz \wedge dx + 
            C dx \wedge dy\). Then 
            \[
            \begin{aligned}
                d\eta &= \frac{\partial A}{\partial x}dx\wedge dy\wedge dz + 
                \frac{\partial B}{\partial y}dy\wedge dz\wedge dx + 
                \frac{\partial C}{\partial z}dz\wedge dx\wedge dy \\
                &= \left(\frac{\partial A}{\partial x} + 
                \frac{\partial B}{\partial y} + 
                \frac{\partial C}{\partial z} \right)dx\wedge dy\wedge dz \\
                &\rightarrow \divv \eta.
            \end{aligned}
            \]
            \item In usual vector calculus terms, we say \(\nabla\) produces a 
            vector, \(\divv\) produces a scalar, and \(\curl\) produces a vector.
            This \(r\)-form to \((m-r)\)-form correspondance is provided by 
            the Hodge star operation.
        \end{enumerate}
    \end{remark}
    \subsection{Differential Form as Equivalence Classes}
    \subsubsection{Operations are Well-Defined}
    \begin{theorem}
        If \(\omega\equiv \omega'\), \(\eta \equiv \eta'\) are formal \(k\)-forms on 
        \(\mathbb{R}^m\), 
        then 
        \[
        \omega \wedge \eta \equiv \omega' \wedge \eta'.
        \]
    \end{theorem}
    \begin{theorem}
        If \(\omega \equiv \omega'\) are formal \(k\)-forms on \(\mathbb{R}^m\), 
        and \(f \in C^\infty(\mathbb{R}^m)\), then 
        \[
        f^*\omega \equiv f^*\omega'.
        \]
    \end{theorem}
    \begin{theorem}
        If \(\omega \equiv \omega'\) are formal \(k\)-forms on \(\mathbb{R}^m\), 
        then 
        \[
        d\omega\equiv d\omega'.
        \]
    \end{theorem}
    \subsubsection{Equivalence Classes}
    \begin{definition}[Euclidean Differential k-form]\label{def:euc-form}
        Denote the set of all formal \(k\)-forms on \(\mathbb{R}^m\) be 
        \(A^k(\mathbb{R}^m)\). Then the set of all differential \(k\)-forms on 
        \(\mathbb{R}^m\) is defined to be 
        \[
        \eucForm{k}{m} := A^k(\mathbb{R}^m) / {\equiv}.
        \]
    \end{definition}
    \subsection{Properties of Operations}
    \begin{theorem}[Properties of Exterior Product]\label{thm:exterior-product-properties}
        Let \(\omega \in \eucForm{k}{m}\), \(\eta \in 
        \eucForm{l}{m}\). Then
        \[
        \begin{aligned}
            (\omega_1+\omega_2)\wedge \eta &= \omega_1\wedge \eta + \omega_2 \wedge 
            \eta \\
            \omega \wedge \eta &= (-1)^{kl}\eta \wedge \omega. \\
        \end{aligned}
        \]
    \end{theorem}
    \begin{theorem}[Properties of Pullback]\label{thm:pullback-properties}
        Let \(f \in C^\infty\), \(\omega \in \eucForm{k}{m}\). Then 
        \[
        \begin{aligned}
            f^*(\omega_1+\omega_2) &= f^*\omega_1+f^*\omega_2 \\
            f^*(\omega_1\wedge \omega_2) &= (f^*\omega_1) \wedge (f^*\omega_2) \\
            g^*(f^*\eta) &= (f\circ g)^*\eta.
        \end{aligned}
        \]
    \end{theorem}
    \begin{theorem}[Properties of Exterior Differentiation]\label{thm:diff-properties}
        Let \(f \in C^\infty\), \(\omega \in \eucForm{k}{m}\). Then
        \[
        \begin{aligned}
            f^*(d\omega) &= d(f^*\omega) \\
            d(\omega\wedge \eta) &= (d\omega)\wedge \eta + (-1)^k\omega\wedge (d\eta) \\
            d(d\omega) &= 0.
        \end{aligned}
        \]
    \end{theorem}
    \newpage
    \subsection{Differential Forms on Manifolds}
    \subsubsection{Requirements of Manifold Forms}
    \begin{definition}[Differential Form on a Manifold]\label{def:manifold-form}
        A \(C^\infty\) differential \(k\)-form \(\omega\) on manifold \(\manifold[M]\), 
        \(\omega \in \form{k}{M}\), consists of a family of 
        differential \(k\)-forms \(\omega_\phi \in \Lambda^k(\phi(U_\phi))\), 
        \(\phi \in \Phi\), \(\phi : U_\phi \to V_\phi = \phi(U_\phi) \subseteq 
        (-\infty, 0] \times \mathbb{R}^{m-1}\), with an additional requirement that 
        \[
        \left(\phi\circ{\phi'} ^{-1}\right)^*\left(\left.\omega_\phi
        \right|_{\phi(U_\phi \cap U_{\phi'})}\right) 
        = \left.\omega_{\phi'}\right|_{\phi'(U_\phi \cap U_{\phi'})},\; \forall 
        \phi, \phi' \in \Phi.
        \]
        \(\omega_\phi\) is called the local expression of \(\omega\) on \(U_\phi\) 
        via \(\phi\).
    \end{definition}
    \begin{remark}\phantom{hehe}\\
        \begin{center}
            \centering
            \def\svgwidth{0.6\textwidth}
            \input{./figs/manifold_form.pdf_tex}
        \end{center}
        The motivation is that, if two differential forms describe the same set, 
        they should "agree" on that portion of manifold. The "agreement" is done by 
        pullback using the overlap function.
    \end{remark}
    \newpage
    \begin{theorem}[Covariancy of Differential Forms]\label{thm:form-covariancy}
        Choose two overlapping charts \(x^1, \dots, x^m\) and \(y^1, \dots, y^m\) 
        and define their Jacobian matrix \cref{def:jacobian-matrix} 
        \(J\indices{^\nu_\mu}:=\left.\frac{\partial y^\nu}{\partial x^\mu}\right|
        _{\phi(p)}\), combining \cref{def:formal-pullback} and 
        \cref{def:manifold-form}, we have 
        \[
        \begin{cases}
            dy^\nu = J\indices{^\nu_\mu}dx^\mu & (\text{covariant}) \\
            \omega_{\nu'} = \omega_{\mu}(J^{-1})\indices{^\mu_\nu} & 
            (\text{contravariant})
        \end{cases}
        \]
    \end{theorem}
    \subsubsection{Operations on Manifold Forms}
    \begin{definition}[Addition]\label{def:mani-form-add}
        Given two forms \(\omega, \eta \in \form{k}{M}\), 
        \(m=\dim \manifold[M]\), define their sum to be \(\omega+\eta\), whose 
        chart components are given by 
        \[
        (\omega+\eta)_\phi := \omega_\phi + \eta_\phi.
        \]
        They satisfy \cref{def:manifold-form} thanks to the linearity of pullback.
    \end{definition}
    \begin{definition}[Exterior Product]\label{def:mani-form-exterior-product}
        Given two forms \(\omega, \eta \in \form{k}{M}\), 
        \(m=\dim \manifold[M]\), define their exterior product to be 
        \(\omega\wedge\eta\), whose chart components are given by 
        \[
        (\omega\wedge\eta)_\phi := \omega_\phi\wedge \eta_\phi.
        \]
        They satisfy \cref{def:manifold-form} because pullback commutes with 
        exterior product.
    \end{definition}
    \begin{definition}[Exterior Differentiation]\label{def:mani-form-diff}
        Given a form \(\omega \in \form{k}{M}\), 
        \(m=\dim \manifold[M]\), define its exterior derivative to be
        \(d\omega\), whose chart components are given by 
        \[
        (d\omega)_\phi := d(\omega_\phi).
        \]
        They satisfy \cref{def:manifold-form} because pullback commutes with 
        exterior product.
    \end{definition}
    \begin{definition}[Pullback]\label{def:mani-form-pullback}
        Given a form \(\omega \in \form{k}{N}\) and a function 
        \(f : \manifold[M] \to \manifold[N]\), \(m=\dim \manifold[M], 
        n=\dim \manifold[N]\). Choose coordinate functions 
        \(\phi_i : U_i \to \phi_i(U_i)\) on \(\manifold[M]\), and 
        \(\psi_j : V_j \to \psi_j(V_j)\) on \(\manifold[N]\). \\
        To define \((f^*\omega)_{\phi_1}\), choose any 
        \(V_1, \dots, V_s\) s.t. \(U_1 \subseteq \bigcup_{j=1}^{s} \tilde{f}(V_j)\).
        Then 
        \[
        (f^*\omega)_{\phi_i} := \sum_{j}^{} 
        f^*\left(\left.\omega_{\psi_j}\right|_{\phi(f(U_i) \cap V_j)}\right)
        \]
        They satisfy \cref{def:manifold-form}.
    \end{definition}
    \newpage

    \section{Integration of Differential Forms}
    \subsection{Partition of Unity}
    \begin{definition}[Support]\label{def:support}
        Let \(X\) be a topological space, and \(f : X \to \mathbb{R}\). Then 
        the support of \(f\) is defined as
        \[
        \supp f := \Set{x \in X|f(x) \neq 0}.
        \]
    \end{definition}
    \begin{theorem}[Partition of Unity]\label{thm:manifold-partition-of-unity}
        Let \(\manifold[M]\) be a \(C^\infty\) manifold with dimension \(m\) with 
        atlas \(\Phi\). \\
        Let \(\Phi = \Set{\phi_j|\phi_j : V_j \to \phi_j(V_j), j \in J}\).\\
        Then it is possible to construct a set of \(C^\infty\) functions 
        \(\rho_j, j \in J\) s.t. 
        \[
        1 = \sum_{j \in J}^{} \rho_j,\; \supp \rho_j \subseteq V_j
        \]
    \end{theorem}
    \subsection{Orientation}
    \subsubsection{Definition}
    \begin{definition}[Compatible Coordinate Charts]\label{def:compatible-coord-charts}
        Given a manifold \(\manifold[M]\), its two coordinate charts are called 
        compatible (have the same orientation) if, 
        \[
        \det J > 0.
        \]
        Where \(J\) is the Jacobian matrix \cref{def:jacobian-matrix}.\\
        The collection of all compatible charts of a specific orientation are 
        denoted \(\tilde{\Phi}\).
    \end{definition}
    \begin{theorem}
        A manifold has either no orientation (any atlas is not compatible) or 
        two orientations.
    \end{theorem}
    \subsubsection{Positively Oriented Boundary}
    \begin{definition}[Positively Oriented Boundary]\label{def:positive-boundary}
        Let \(\manifold[M]\) be a orientable \(C^\infty\) manifold with dimension 
        \(m\), with compatible atlas \(\tilde{\Phi}\). Define coordinate charts on 
        \(\partial \manifold[M]\) from \(\Phi\) as follows, 
        \[
        \phi^{\partial \manifold[M]} : U_\phi \cap \partial \manifold[M] 
        \to \mathbb{R}^{m-1}, 
        \]
        Then \(\tilde{\Phi}^{\partial \manifold[M]} := 
        \Set{\phi^{\partial \manifold[M]}| \phi \in \tilde{\Phi}}\) determines an 
        orientation on \(\partial \manifold[M]\), called the positive orientation.
    \end{definition}
    \subsection{Integration of Forms of Highest Degree}
    \begin{definition}[Integration]\label{def:top-form-integration}
        Let \(\manifold[M]\) be a paracompact \(C^\infty\) manifold of dimension 
        \(m\), oriented by compatible atlas \(\tilde{\Phi} = \Set{\phi_j|j \in J}\). 
        Choose a \(C^\infty\) partition of unity \(\rho_j, j \in J\) of 
        \(\manifold[M]\) s.t. \(\supp \rho_j \subseteq U_{\phi_j} := U_j\). \\
        Let an \(m\)-form \(\omega \in \form{m}{M}\) has local expression 
        \(\omega_{\phi_j} = f_j(x)dx^1_j\wedge \dots\wedge dx^m_j\), we say 
        \[
        \int_{\manifold[M]} \omega = 
        \sum_{j \in J}^{} \int_{(-\infty, 0] \times \mathbb{R}^{m-1}}
        (\rho_j \circ \phi_j ^{-1})(x) f_j(x) dx^1\dots dx^m
        \]
        if the finite sum exists and has the same value for all choices of 
        \(\rho_j\) and \(\phi_j\).
    \end{definition}
    \begin{remark}
        The following theorem reveals why orientation is a significant factor in 
        the context of integration.
    \end{remark}
    \begin{theorem}[Criterion of Existence of Integral]\label{thm:cpt-form-integral-exist}
        If \(\supp \omega\) is compact, then \(\int_{\manifold[M]} \omega\) exists.
    \end{theorem}
    \begin{proof}
        Let two sets of coordinate charts be 
        \[
        \begin{aligned}
            \phi_j &: U_j \to V_j, j \in J \\
            \phi_k'&: U'_k \to V'_k, k \in K.
        \end{aligned}
        \]
        And cooresponding partition of unity be \(\rho_j, \rho'_k\).\par
        (The goal) Show
        \[
        \begin{aligned}
            &\sum_{j \in J}^{} \int
            (\rho_j \circ \phi_j ^{-1})(x) f_j(x) dx^1\dots dx^m \\
            =&\sum_{k \in K}^{} \int
            (\rho'_k \circ {\phi'_k} ^{-1})(x') f'_k(x') {dx'}^1\dots {dx'}^m
        \end{aligned}
        \]
        \par
        (Split using \(\rho_k'\))
        \[
        \begin{aligned}
            \int_{\manifold[M]} \omega &= 
            \sum_{j \in J}^{} \int
            (\rho_j \circ \phi_j ^{-1})(x) f_j(x) dx^1\dots dx^m \\
            &= \sum_{j \in J}^{} \int \sum_{k \in K}^{}
            (\rho'_k\circ \phi_j ^{-1})(x)(\rho_j\circ \phi_j ^{-1})(x) f_j(x)
            dx^1\dots dx^m.
        \end{aligned}
        \]
        Since the sum is finite, and \(\supp\omega\) is compact, and therefore 
        the integral is not improper; thus, there can be no limit or Fubini 
        problems on exchanging sums and integrals. So 
        \[
        \begin{aligned}
            \int_{\manifold[M]} \omega
            &= \sum_{j \in J}^{}\sum_{k \in K}^{} \int 
            (\rho_j\rho'_k\circ \phi_j ^{-1})(x)f_j(x)
            dx^1\dots dx^m \\
        \end{aligned}
        \]
        \par
        (Change of variables)
        First fix \(j, k\).
        \[
        \begin{aligned}
            &\int (\rho_j\rho'_k\circ \phi_j ^{-1})(x)f_j(x) dx^1\dots dx^m \\
            =&\int (\rho_j\rho'_k\circ \phi_j ^{-1})(\phi_j\circ{\phi_k'}^{-1}(x'))
            f_j(\phi_j\circ{\phi_k'}^{-1}(x')) 
            \left|\det \left(\frac{\partial x}{\partial x'}\right)\right|
            {dx'}^1\dots {dx'}^m \\
            =&\int (\rho_j\rho'_k\circ{\phi_k'}^{-1}(x'))
            f_j(\phi_j\circ{\phi_k'}^{-1}(x')) 
            \left|\det \left(\frac{\partial x}{\partial x'}\right)\right|
            {dx'}^1\dots {dx'}^m
        \end{aligned}
        \]\par
        (Use pullback requirement) From \cref{def:manifold-form},
        \[
        (\phi_j\circ {\phi'_k}^{-1})^*\omega_{\phi_j} = \omega_{\phi'_k}, 
        \]
        we see
        \[
        \begin{aligned}
            &f'_j(x')dx^1\wedge \dots \wedge dx^m \\
            =&f_j(\phi_j\circ {\phi'_k}^{-1}(x'))
            \left(\frac{\partial x^1}{\partial {x'}^{\ell_1}}{dx'}^{\ell_1}\right)
            \wedge \dots\wedge 
            \left(\frac{\partial x^m}{\partial {x'}^{\ell_m}}{dx'}^{\ell_m}\right) \\
            =&\sum_{\sigma \in S_m}^{} f_j(\phi_j\circ {\phi'_k}^{-1}(x'))
            (\sgn \sigma)\frac{\partial x^1}{\partial {x'}^{\sigma(1)}}\dots
            \frac{\partial x^m}{\partial {x'}^{\sigma(m)}}
            {dx'}^1\wedge\dots\wedge{dx'}^m \\
            =&f_j(\phi_j\circ {\phi'_k}^{-1}(x'))
            \det \left(\frac{\partial x}{\partial x'}\right)
            {dx'}^1\wedge\dots\wedge{dx'}^m
        \end{aligned}
        \]\par
        (Orientation) Within a compatible atlas, however, 
        \(\left|\det \left(\frac{\partial x}{\partial x'}\right)\right|
        =\det \left(\frac{\partial x}{\partial x'}\right)\). Therefore, the 
        integral 
        \[
        \begin{aligned}
            &\int (\rho_j\rho'_k\circ{\phi_k'}^{-1}(x'))
            f_j(\phi_j\circ{\phi_k'}^{-1}(x')) 
            \left|\det \left(\frac{\partial x}{\partial x'}\right)\right|
            {dx'}^1\dots {dx'}^m \\
            =&\int (\rho_j\rho'_k\circ{\phi_k'}^{-1}(x'))
            f'_j(x')
            {dx'}^1\dots {dx'}^m
        \end{aligned}
        \]\par
        (Closing) By moving the sum wrt \(j \in J\) into the integral and using 
        the property of partition of unity, the proof is completed.
    \end{proof}
    \subsection{Integration of Forms of Lower Degree}
    \subsubsection{Definition}
    \begin{definition}[Integration of Lower Degree Forms]\label{def:pullback-integral}
        Let \(\manifold[Z]\) be an oriented \(C^\infty\) manifold of dimension \(d\), 
        \(f : \manifold[Z] \to \manifold[M]\) be a \(C^\infty\) map to a \(C^\infty\) 
        manifold \(\manifold[M]\) of dimension \(m\). \\
        Let \(\omega \in \form{d}{M}\), we define 
        \[
        \int_{\manifold[Z]}\omega := \int_{\manifold[Z]}^{} f^*\omega
        \]
        if the latter exists and the pullback function \(f\) is clear from context.
    \end{definition}
    \subsubsection{Stoke's theorem}
    \begin{theorem}[Stoke's Theorem]\label{thm:stokes-theorem}
        If \(\manifold[M]\) is an oriented \(C^\infty\) manifold of dimension \(m\) 
        and \(\omega \in \form{{m-1}}{M}\), then 
        \[
        \int_{\manifold[M]}^{}d\omega = \int_{\partial \manifold[M]}^{}\omega
        := \int_{\partial \manifold[M]}^{} i^*\omega,
        \]
        where \(i : \partial \manifold[M] \to \manifold[M]\) is just the immersion 
        map, \(i : p \mapsto p\).
    \end{theorem}
    \begin{proof}
        (Partition Using Charts) Choose a \(C^\infty\) partition of unity 
        \cref{thm:manifold-partition-of-unity} \(\rho_j, j \in J\) s.t. 
        \(\supp \rho_j\) are compact and \(\supp \rho_j \subseteq U_{\phi_j} 
        := U_j\). \par
        Now \(\omega = \sum_{j \in J}^{} \rho_j \omega\) is a finite sum. So 
        it suffices to show that, if \(\eta \in \form{m-1}{M}\), \(\supp \eta\) 
        compact and \(\supp\eta \subseteq U_\phi\) then \(\int_{\manifold[M]}^{}
        d\eta = \int_{\partial \manifold[M]}^{} \eta\), and apply 
        \(\eta = \rho_j\omega\) for all \(j \in J\). \par
        (The integral)
        Suppose the coordinates of \(\phi\) is labeled \(x^1, \dots, x^m\).
        % Since \(\supp \eta \subseteq U_\phi\), we will make no distinction between 
        % \(\eta \in \form{m-1}{M}\) and \(\eta_\phi \in \eucForm{m-1}{m}\). \par
        Locally, let 
        \[
        \eta = \sum_{\ell=1}^{m} f_\ell dx^1\wedge \cdots \cancel{dx^\ell} \cdots 
        \wedge dx^m,
        \]
        where \(f_\ell : (-\infty, 0] \times \mathbb{R}^{m-1} \to \mathbb{R}\), and 
        it is \(C^\infty\). Then, also locally, 
        \[
        \begin{aligned}
            d\eta &= \sum_{\ell = 1}^{m} \frac{\partial f_\ell}{\partial x^\ell} dx^\ell 
            \wedge dx^1\wedge \cdots \cancel{dx^\ell} \cdots \wedge dx^m \\
            &= \left(\sum_{\ell=1}^{m} (-1)^{\ell-1}
            \frac{\partial f_\ell}{\partial x^\ell}\right)dx^1\wedge \dots \wedge dx^m.
        \end{aligned}
        \]
        Then by \cref{def:top-form-integration}, 
        \[
        \int_{\manifold[M]}^{}d\eta = \int_{(-\infty, 0]\times \mathbb{R}^{m-1}}^{} 
        \left(\sum_{\ell=1}^{m} (-1)^{\ell-1}
        \frac{\partial f_\ell}{\partial x^\ell}\right)dx^1\dots dx^m.
        \]\par
        (To be precise, we need to choose another partition of unity \(\rho_j'\) to 
        do intergration. But we can just choose it to cover all of \(\supp \eta\) 
        and don't care all other parts, so that doesn't matter too much.)\par
        Choose a rectangular region \(R\) s.t. 
        \[
        \supp \eta \subseteq [a_1, 0]\times \dots \times [a_m, b_m]
        \]
        and define 
        \[
        R_\ell := [a_1, 0] \times \dots \cancel{[a_\ell, b_\ell]} \dots 
        \times [a_m, b_m].
        \]\par
        (\(\ell = 2,\dots, m\)) In this case, by Fubini and FTC, 
        \[
        \begin{aligned}
            &\int_{R}^{}\frac{\partial f_\ell}{\partial x^\ell}dx^1\dots dx^m\\
            =&\int_{R_\ell}^{}\left(\int_{a_\ell}^{b_\ell} 
            \frac{\partial f_\ell}{\partial x^\ell}dx^\ell\right)dx^1\dots
            \cancel{dx^\ell}\dots dx^m \\
            =&\int_{R_\ell}^{}
            (\underbrace{f_\ell(x^1, \dots, b_\ell, \dots, x^m)}_{0}
             - \underbrace{f_\ell(x^1, \dots, a_\ell, \dots, x^m)}_{0})
            dx^1\dots \cancel{dx^\ell}\dots dx^m \\
            =&0, 
        \end{aligned}
        \]
        since \(\supp\eta \subseteq R\), so on the boundary \(f=0\). \par
        (\(\ell = 1\)) Now the integral has only one term left.
        \[
        \begin{aligned}
            \int_{\manifold[M]}^{} d\eta &= 
            \int_{R}^{} \frac{\partial f_1}{\partial x^1}dx^1\dots dx^m \\
            &=\int_{R_1}^{} \left(\int_{a_1}^{0}\frac{\partial f_1}{\partial x^1}
            dx^1\right)dx^2\dots dx^m \\
            &=\int_{R_\ell}^{}
            (f_1(0, x^2, \dots, x^m) - \underbrace{f_1(a_1, x^2, \dots, x^m)}_{0})
            dx^2\dots dx^m \\
            &=\int_{\mathbb{R}^{m-1}}^{} (f_1\circ i) dx^2\dots dx^m \\
            &=\int_{\partial \manifold[M]}^{} i^*\eta.
        \end{aligned}
        \]
    \end{proof}

    \newpage
    \section{Vector Fields}
    \subsection{Definition}
    \begin{definition}[Vector Fields]\label{def:vector-fields}
        A vector field \(X\) on \(\manifold[M]\) is a smooth assignment of 
        a tangent vector \(X_p \in T_p\manifold[M]~\forall p \in \manifold[M]\). 
        
        "Smooth" assignment is defined to be that the Lie derivative 
        \cref{def:lie-derivative} is smooth.
    \end{definition}
    \begin{definition}[Lie Derivative]\label{def:lie-derivative}
        The Lie-derivative of function \(f\) with respect to vector field 
        \(X\) is defined as 
        \[
        \lder{X}f := Xf, 
        \]
        and at a specific point \(p \in \manifold[M]\), 
        \[
        \lder{X}f(p) := Xf(p) := X_pf.
        \]
    \end{definition}
    \begin{theorem}[Properties of Lie Derivative]\label{thm:lie-derivative-properties}
        The Lie derivative has the following properties,
        \begin{enumerate}
            \item \(X(rf+g) = rXf+Xg\)
            \item \(X(fg) = fXg+gXf\).
        \end{enumerate}
    \end{theorem}
    \begin{theorem}[Component of Vector Field]\label{thm:vector-field-component}
        \textbf{Given a chart} \((U, \phi)\) on \(\manifold[M]\), we can write 
        \[
        X_U = X_Ux^\mu \tfld{\mu}.
        \]
        When the context is clear or \textbf{for convenience}, we write 
        \[
        X = Xx^\mu \tfld{\mu} := X^\mu \tfld{\mu}.
        \]
    \end{theorem}
    \begin{proof}
        We know 
        \[
        (Xf)(p) = X_pf = X_px^\mu\tvec{\mu}{p}f = (Xx^\mu)(p)\tvec{\mu}{p}f.
        \]
    \end{proof}
    \begin{remark}
        \(\tfld{\mu}\) is a vector field that assigns each point \(p \in 
        \manifold[M]\) with the vector \(\tvec{\mu}{p} \in T_p\manifold[M]\).
        
    \end{remark}
    \begin{theorem}[Contravariancy of Vector Fields]\label{thm:}
        Given two coordinate charts \((U, \phi)\) and \((U', \phi')\) s.t. 
        \(U \cap U' = S \neq \varnothing\). On \(S\), 
        \[
        X^{\nu'} = \sum_{\mu=1}^{m} X^\mu 
        \frac{\partial x'^{\nu}}{\partial x^\mu}.
        \]
        Analogous to \cref{thm:contravariant-vector}.
    \end{theorem}
    \subsection{Lie Bracket}
    \begin{definition}[Composition of Vector Fields]\label{def:vector-field-composition}
        We can view \(X : C^\infty(\manifold[M]) \to C^\infty(\manifold[M])\), 
        and so does \(Y\). Therefore, we define 
        \[
        (X \circ Y)(f) := X(Yf).
        \]
    \end{definition}
    \begin{definition}[Lie Bracket (Commutator)]\label{def:lie-bracket}
        We define the Lie Bracket of two vector fields \(X, Y\) to be 
        \[
        [X, Y] := X\circ Y - Y\circ X.
        \]
        In particular, 
        \[
        [X, Y](f) = \lder{X}(\lder{Y}f) - \lder{Y}(\lder{X}f)
        \]
    \end{definition}
    \begin{remark}
        Lie Bracket \cref{def:lie-bracket} is a vector field, while the 
        expression \(X\circ Y\) is not, because it contains second differential 
        terms. See the following proof.
        
    \end{remark}
    \begin{theorem}[Lie Bracket Components]\label{thm:lie-bracket-components}
        \[
        [X, Y]^\mu = (X^\nu\tfld{\nu}Y^\mu - Y^\nu\tfld{\nu}X^\mu).
        \]
    \end{theorem}
    \begin{proof}
        Given \(X=X^\mu\tfld{\mu}, Y = Y^\nu\tfld{\nu}\), we try to write 
        the component of \(X\circ Y\).
        \[
        X\circ Y(f) = X^\mu \tfld{\mu}\left(Y^\nu \tfld{\nu}f\right).
        \]
        However, notice that 
        \[
        \begin{aligned}
            &Y^\nu := Yx^\nu \in C^\infty(\manifold[M]); \\
            &\tfld{\nu} : C^\infty(\manifold[M]) \to C^\infty(\manifold[M]), \\
            &\implies\tfld{\nu}f \in C^\infty(\manifold[M]).
        \end{aligned}
        \]
        So we need to use the Leibniz property of \(\tfld{\mu}\) 
        \cref{def:derivation} in order to evaluate the second term. Doing this 
        for \(X\circ Y(f)\) and \(Y\circ X(f)\), we have
        \[
        \begin{aligned}
            X\circ Y(f) &= X^\mu \left((\tfld{\mu}Y^\nu)(\tfld{\nu}f)+
            Y^\nu \tfld{\mu}\tfld{\nu}f\right). \\
            Y\circ X(f) &= Y^\nu \left((\tfld{\nu}X^\mu)(\tfld{\mu}f)+
            X^\mu \tfld{\nu}\tfld{\mu}f\right).
        \end{aligned}
        \]
        So if \(\tfld{\mu}\tfld{\nu}f=\tfld{\nu}\tfld{\mu}f\), then by 
        subtracting, we can cancel the second order terms, and we are done. We 
        prove so now.
        \[
        \begin{aligned}
            (\tfld{\mu}\tfld{\nu}f)(p)&=\frac{\partial }{\partial u^\mu}
            \left.\left((\tfld{\nu}f)\circ\phi ^{-1}\right)\right|_{\phi(p)} \\
            &=\frac{\partial }{\partial u^\mu}
            \left.\left(\tvec{\nu}{\phi ^{-1}(u)}f\right)\right|_{\phi(p)} \\
            &=\frac{\partial }{\partial u^\mu}
            \left.\left(\left.
                \frac{\partial }{\partial u^\nu}(f\circ\phi ^{-1})
            \right|_u\right)\right|_{\phi(p)} \\
            &=\frac{\partial }{\partial u^\nu}
            \left.\left(\left.
                \frac{\partial }{\partial u^\mu}(f\circ\phi ^{-1})
            \right|_u\right)\right|_{\phi(p)} \\
            &=(\tfld{\nu}\tfld{\mu}f)(p).
        \end{aligned}
        \]
    \end{proof}
    \begin{theorem}[Properties of Lie Brackets]\label{thm:lie-bracket-properties}
        \phantom{something}
        \begin{enumerate}
            \item \([X,Y] = -[Y, X]\) (antisymmetry)
            \item \(\sum_{\text{cyc}}^{} [X, [Y, Z]]=0\). (Jacobi Identity)
        \end{enumerate}
    \end{theorem}
    \subsection{Integral Curves and Flows}
    \begin{definition}[Intergral Curve]\label{def:intergral-curve}
        Let \(X\) be a vector field on \(\manifold[M]\), \(p \in \manifold[M]\). 
        Then an integral curve of \(X\) through \(p\) is a curve \(\sigma : 
        (-\epsilon, \epsilon) \to \manifold[M]\) s.t. 
        \[
        \begin{aligned}
            \sigma(0) &= p, \\
            \sigma_*\left(\frac{d}{dt}\right)_t &= X_{\sigma(t)}.
        \end{aligned}
        \]
    \end{definition}
    \begin{remark}
        Qualitatively, using \cref{thm:curve-pushforward}, this pushforward is 
        just \([\sigma] \in T_{\sigma(t)}\manifold[M]\). Therefore, the second 
        condition is saying in some sense that the curve is tangent to the 
        vector field on the manifold. For quantitative description, see below.
        
    \end{remark}
    \begin{definition}[Differential Equations of Integral Curve]\label{def:de-int-curve}
        The components \(X^\mu\) of \(X\) determine the integral curve \(\sigma\) 
        by the following ODE with boundary conditions, 
        \[
        \begin{aligned}
            X^\mu(\sigma(t)) &= \frac{d}{dt}x^\mu(\sigma(t)) \\
            x^\mu(\sigma(0)) &= x^\mu(p), \mu = 1, 2, \dots, m.
        \end{aligned}
        \]
    \end{definition}
    \newpage
    \subsubsection{One-parameter Family of Diffeomorphisms}
    \begin{definition}[Local 1D Family of Local Diffeomorphisms]\label{def:local-diff}
        A local, 1D family of local diffeomorphisms at \(p \in \manifold[M]\) 
        is made up of (1) an open neighborhood \(U\) of \(p\), (2) \(\epsilon 
        > 0\) (3) a family of diffeomorphisms \(\Set{\phi_t|\left|t\right|<\epsilon}\), 
        \(\phi_t : U \to \manifold[M]\) s.t. 
        \begin{enumerate}
            \item Every \(\phi_t\) is a smooth function in \(t\) and \(q\).
            \item \(\forall t, s \in \mathbb{R}\) and \(|t|, |s|, |t+s| < \
            \epsilon\), and \(\forall q \in U\) s.t. \(\phi_t(q), \phi_s(q), 
            \phi_{t+s}(q) \in U\), we have 
            \[
            \phi_s(\phi_t(q)) = \phi_{s+t}(q).
            \]
            \item \(\phi_0(q)=q\).
        \end{enumerate}
    \end{definition}
    \begin{remark}
        The first "local" refers to the parameter \(t\), which is limited to 
        \((-\epsilon, \epsilon)\). The second "local" refers to the spatial 
        limitation to \(U\). 
        You can view \(\phi_t(q)\) as a curve that brings \(t \in (-\epsilon, 
        \epsilon)\) to \(\phi_t(q) \in \manifold[M]\).
        
    \end{remark}
    \begin{definition}[Induced Vector Field]\label{def:induced-vector-field}
        By taking tangents to the curve family \cref{def:local-diff}, we have 
        the induced vector field \(X^\phi\) given by
        \[
        X^\phi_q(f) := \left.\frac{d}{dt}(f(\phi_t(q)))\right|_{t=0}
        \]
    \end{definition}
    \begin{theorem}
        The curve family \(t \mapsto \phi_t(q)\) is the integral curve of 
        the induced vector field \cref{def:induced-vector-field} \(X^\phi_q\).
    \end{theorem}
    \begin{proof}
        \[
        \begin{aligned}
            X^\phi_{\phi_s(q)} &= \left.\frac{d}{dt}
            (f\circ\phi_t\circ\phi_s(q))\right|_{t=0} \\
            &= \left.\frac{d}{dt}(f\circ\phi_{t+s}(q))\right|_{t=0}.
        \end{aligned}
        \]
        Let \(u = t+s\). Then 
        \[
        \begin{aligned}
            X^\phi_{\phi_s(q)} &= 
            \left.\frac{d}{du}(f\circ\phi_{u}(q))\right|_{u=s}. \\
            &= \phi_{q*}\left(\frac{d}{dt}\right)_sf.
        \end{aligned}
        \]
    \end{proof}
    \subsubsection{Local Flows}
    \begin{definition}[Local Flow]\label{def:local-flow}
        Let \(X\) be a vector field on open \(U \subseteq \manifold[M]\), and 
        \(p \in U\). A local flow at \(p\) is a local one-parameter family 
        of local diffeomorphisms \cref{def:local-diff} defined on some open 
        \(V \subseteq U\) s.t. \(p \in V\) and the induced vector field 
        \cref{def:induced-vector-field} is \(X\).
    \end{definition}
    \begin{remark}
        Local flows always exist and are unique. In contrast, global flows 
        (which means \(t \in \mathbb{R}\) instead of a restricted interval) 
        may not exist.
        
    \end{remark}
    \subsubsection{Lie Derivative}
    \begin{theorem}[Interpretation of Lie Bracket]\label{thm:lie-bracket-flow}
        If \(X, Y\) are two vector fields on \(\manifold[M]\), and define the 
        following quantity, which can be interpreted as the change of \(Y\) 
        when following the integral curves of \(X\), as
        \[
        \left.\frac{d}{dt}(\phi_{-t*}^X(Y))\right|_{t=0}:=
        \lim_{\epsilon \to 0} \frac{\phi_{-\epsilon*}^X(Y_{\phi_\epsilon^X(p)})
        -Y_p}{\epsilon}.
        \]
        Then, 
        \[
        \left.\frac{d}{dt}(\phi_{-t*}^X(Y))\right|_{t=0}=[X, Y].
        \]
    \end{theorem}
    \section{Cotangent Spaces}
    \subsection{Cotangent Vectors}
    \begin{definition}[Cotangent Spaces]\label{def:cotangent-space}
        The cotangent space \(T_p^*\manifold[M]\) at \(p \in \manifold[M]\) is 
        the set of all linear functions \(f : T_p\manifold[M] \to \mathbb{R}\). 
        \newline
        Its member is called a cotangent vector. \newline
        \(\dim T_p^*\manifold[M]=\dim T_p\manifold[M]\).
    \end{definition}
    \begin{definition}[One-Form]\label{def:one-form}
        A one-form on \(\manifold[M]\) is a smooth assignment of cotangent 
        vectors \(\omega : p \mapsto \omega_p\). \newline
        It may be understood as a covector field.
    \end{definition}
    \begin{definition}[Basis Cotangent Vectors]\label{def:cotangent-basis}
        The basis cotangent vectors is chosen to be the dual basis of the 
        basis tangent vectors \cref{def:basis-derivations}, 
        \[
        (dx^\mu)_p(\tvec{\nu}{p})=\delta\indices{^\mu_\nu}.
        \]
    \end{definition}
    \begin{theorem}[Coordinate Expression of Cotangent Vectors]\label{thm:cotangent-coordinates}
        Any \(f \in T_p^*\manifold[M]\) can be expanded as
        \[
        f = f_\mu (dx^\mu)_p.
        \]
        Any one-form \(\omega\) can be expressed as 
        \[
        \omega=\omega_\mu dx^\mu.
        \]
    \end{theorem}
    \newpage
    \subsection{Pullback}
    \subsubsection{Definition}
    \begin{definition}[Pullback]\label{def:pullback}
        Given a function and its pushforward, 
        we define pullback to be the dual of pushforward, i.e., 
        \begin{align*}
            h   &: & &\manifold[M]    & &\to & &\manifold[N], \\
            h_* &: & T_p&\manifold[M] & &\to & T_{h(p)}&\manifold[N], \\
            h^* &: & T_{h(p)}^*&\manifold[N] & &\to & T_p^*&\manifold[M], 
        \end{align*}
        s.t. given \(f \in T_{h(p)}^*\manifold[N]\) and \(v \in T_p\manifold[M]\), 
        \[
        (h^*f)(v) := f(h_*v).
        \]
    \end{definition}
    \begin{remark}
        Note especially on the direction of original function and its induced 
        pullback. This is crucial to the covariancy of one-forms.
    \end{remark}
    \begin{theorem}
        Given \(\omega\) a one-form on \(\manifold[N]\), and a function 
        \(h : \manifold[M] \to \manifold[N]\), the pullback \(h^*\omega\) is 
        defined as 
        \[
        (h^*\omega)(v)_p = \omega(h_*v)_{h(p)}.
        \]
    \end{theorem}
    \begin{theorem}[Associativity of Pullbacks]\label{thm:pullback-associativity}
        Analogous to \cref{thm:pushforward-associativity}, given 
        manifolds \(\manifold[M], \manifold[N], \manifold[P]\) and 
        \(h : \manifold[M] \to \manifold[N]\), \(k : \manifold[N] \to 
        \manifold[P]\), then 
        \[
        (k\circ h)^*=k^*\circ h^*.
        \]
    \end{theorem}
    \newpage
    \subsubsection{Jacobian}
    \begin{theorem}[Local Representative of Pullback]\label{thm:pullback-local-representative}
        Let \(\dim \manifold[M]=m, \dim \manifold[N]=n\), \(h : \manifold[M] 
        \to \manifold[N]\), \(\{x^1, \dots, x^m\}\) be the local coordinates of 
        \(\manifold[M]\) around \(p\), and \(\{y^1, \dots, y^n\}\) be the local 
        coordinates of \(\manifold[N]\) around \(h(p)\). Then 
        \[
        h^*\omega = \sum_{\mu=1}^{m} \sum_{\nu=1}^{n} \omega_\nu
        \left.\frac{\partial h^\nu}{\partial x^\mu}\right|_{p}(dx^\mu)_p,
        \]
        where \(J\indices{^\nu_\mu}:=\left.\frac{\partial h^\nu}{\partial x^\mu}\right|_{p}
        :=\tvec{\mu}{p}(y^\nu\circ h)\) is the Jacobian matrix.
    \end{theorem}
    \begin{proof}
        We know by \cref{def:pullback}, 
        \[
        (h^*\omega)_\mu(p) = h^*\omega(\tfld{\mu})=\omega(h_*\tfld{\mu}).
        \]
        Expand it in local coordinates of \(\manifold[N]\), 
        \[
        (h^*\omega)_\mu(p) = \omega_\nu dy^\nu(h_*\tfld{\mu}).
        \]
        Via similar procedure in \cref{thm:pushforward-local}, we arrive at 
        \[
        (h^*\omega)_\mu(p) = \omega_\nu \frac{\partial h^\nu}{\partial x^\mu}.
        \]
    \end{proof}
    \newpage
    \subsection{Transformation Properties}
    \begin{theorem}[Covariancy and Contravariancy]\label{thm:covariant-contravariant}
        Given two coordinate charts \((U, \phi)\) and \((U', \phi')\) s.t. 
        \(U \cap U' = S \neq \varnothing\), then on \(S\), 
        \begin{align*}
            X^{\nu'} &= \sum_{\mu=1}^{m} \frac{\partial x'^{\nu}}{\partial x^\mu}X^\mu, \\
            \omega_{\nu'} &= \sum_{\mu=1}^{m} \omega_\mu\frac{\partial x^\mu}{\partial x'^\nu}.
        \end{align*}
        If Jacobian matrix is given, 
        \begin{align*}
            J\indices{^{\nu'}_\mu} &:=
            \left.\frac{\partial x'^\nu}{\partial x^\mu}\right|_{p}
            :=\tvec{\mu}{p}x'^\nu, \\
            (J ^{-1})\indices{^\mu_{\nu'}} &:=
            \left.\frac{\partial x^\mu}{\partial x'^\nu}\right|_{p}
            :=\tvec{\nu'}{p}x^\mu, 
        \end{align*}
        then, 
        \begin{align*}
            X^{\nu'} &= J\indices{^{\nu'}_\mu}X^\mu, & &\text{(contravariant)}\\
            \omega_{\nu'} &= \omega_\mu (J ^{-1})\indices{^\mu_{\nu'}}.  &
            &\text{(covariant)}
        \end{align*}
    \end{theorem}
    \begin{proof}
        The contravariant part is proved in \cref{thm:contravariant-vector}. 
        Now we turn to the covariant part. \par
        Let \(h = \idd : (U, \phi) \subseteq \manifold[M] \to (U', \phi') 
        \subseteq \manifold[M]\), consider its pullback.
        \[
        (\idd^*\omega)_p = \omega_{\nu'}\frac{\partial x'^\nu}{\partial x^\mu} 
        dx^\mu.
        \]
        Then, 
        \[
        \omega_\mu = \omega_{\nu'}\frac{\partial x'^\nu}{\partial x^\mu}.
        \]
        Inverting the matrix equation above, we get the desired result.
    \end{proof}
    \newpage
    \section{Tensors}
    \begin{definition}[Tensors]\label{def:tensor-finite}
        If \(\dim \manifold[M] \neq \infty\), the tensors of type \((r, s)\) 
        \(T_p^{r, s}\manifold[M]\) are all the linear functions 
        \[
        f : \bigtimes^r T_p^*\manifold[M] \times \bigtimes^s T_p\manifold[M] 
        \to \mathbb{R}.
        \]
        I.e., it eats \(r\) covectors and \(s\) vectors.
    \end{definition}
    \begin{theorem}[Dimensions of General Tensor Space]\label{thm:dim-tensor}
        The dimension of \(T_p^{r, s}\manifold[M]\) is \(m^rm^s\). In particular, 
        a basis for the space is, 
        \[
        \bigotimes_{1 \leq \mu_1 \cdots \mu_r \leq m} \tvec{\mu_i}{p} 
        \otimes \bigotimes_{1 \leq \nu_1 \cdots \nu_s \leq m} (dx^{\nu_i})_{p} 
        \]
    \end{theorem}
    \begin{remark}
        For a detailed proof, see Hoffman.
    \end{remark}
    \newpage
    \section{n-Forms}
    \subsection{Definition}
    \begin{definition}[n-Forms]\label{def:n-forms}
        An n-form is a tensor field of type \((0, n)\) that is totally 
        skew-symmetric (or alternating, or totally antisymmetric), i.e., 
        \[
        \omega(X_1, X_2, \dots, X_n) = (\sgn \sigma)
        \omega(X_{\sigma(1)}, \dots, X_{\sigma(n)}), ~\forall \sigma \in S_n.
        \]
        The set of all n-forms on \(\manifold[M]\) is denoted as 
        \(\form{n}{M}\). \\
        The set of all forms is \(\form{}{M}=\bigoplus_{n=0}^{\dim \manifold[M]}
        \form{n}{M}\). \\
        Conventionally, we classify functions as 0-forms.
    \end{definition}
    \subsection{The Exterior Product}
    \begin{definition}[Exterior Product]\label{def:exterior-product}
        Given \(\omega_1 \in \form{n_1}{M}, \omega_2 \in \form{n_2}{M}\), their 
        exterior product is a \((n_1+n_2)\)-form given by, 
        \[
        \omega_1 \wedge \omega_2
         := \frac{1}{n_1!n_2!}
        \sum_{\sigma \in S_{n_1+n_2}}^{} (\sgn\sigma)(\omega_1 \otimes \omega_2)_\sigma.
        \]
        Written explicitly, 
        \begin{multline*}
        (\omega_1 \wedge \omega_2)(X_{1}, \dots, X_{n_1+n_2}):= \\
        \frac{1}{n_1!n_2!}
        \sum_{\sigma \in S_{n_1+n_2}}^{} (\sgn\sigma)(\omega_1 \otimes \omega_2)
        (X_{\sigma(1)}, \dots, X_{\sigma(n_1+n_2)})
        \end{multline*}
    \end{definition}
    \begin{remark}
        I'll take the alternating property and associativity of the exterior 
        product for granted. For a detailed proof, see Hoffman.
    \end{remark}
    \newpage
    \begin{theorem}[Commutativity with Pullback]\label{thm:exterior-commute-pullback}
        Given \(h : \manifold[M] \to \manifold[N]\) and \(\alpha, \beta \in 
        \form{}{N}\), then 
        \[
        h^*(\alpha \wedge \beta) = (h^*\alpha) \wedge (h^*\beta).
        \]
    \end{theorem}
    \begin{remark}
        For a "generalized" pullback, we have, 
        \[
        (h^*(\alpha))(X_1, \dots, X_{n_1}) = \alpha(h_*X_1, \dots, h_*X_{n_1}).
        \]
    \end{remark}
    \begin{proof}
        \begin{align*}
            (h^*\alpha)\wedge&(h^*\beta)\\
            &=\frac{1}{n_1!n_2!} \sum_{\sigma \in S_{n_1+n_2}}^{} 
            (\sgn\sigma) \alpha \otimes \beta (h_*X_{\sigma(1)}, \dots, h_*
            X_{\sigma(n_1+n_2)}). \\
            &=\frac{1}{n_1!n_2!} \sum_{\sigma \in S_{n_1+n_2}}^{} 
            (\sgn\sigma) h^*\left(\alpha \otimes \beta (X_{\sigma(1)}, \dots, 
            X_{\sigma(n_1+n_2)})\right). \\
            &=h^*\left(\frac{1}{n_1!n_2!} \sum_{\sigma \in S_{n_1+n_2}}^{} 
            (\sgn\sigma) \alpha \otimes \beta (X_{\sigma(1)}, \dots, 
            X_{\sigma(n_1+n_2)})\right). \\
            &=h^*(\alpha \wedge \beta).
        \end{align*}
    \end{proof}
    \begin{theorem}[Skew-Symmetry]\label{thm:skew-symmetry-exterior-product}
        The exterior product makes \(\form{}{M}\) a graded algebra with 
        skew-symmetry given by 
        \[
        \omega_1 \wedge \omega_2 = (-1)^{n_1n_2}\omega_2 \wedge \omega_1.
        \]
    \end{theorem}
    \begin{proof}
        In the definition of exterior product, first fix \(\sigma = \sigma_0\) 
        to consider only one term. \par
        When we switch \(\omega_1\) and \(\omega_2\), we are essentially doing 
        \begin{align*}
            &(\omega_2 \otimes \omega_1)
            (X_{\sigma_0(1)}, \dots, X_{\sigma_0(n_2)}, \underbrace{X_{\sigma_0(n_2+1)}, 
            \dots, X_{\sigma_0(n_1+n_2)}}) \\
            =&(\omega_1 \otimes \omega_2)
            (\underbrace{X_{\sigma_0(n_2+1)}, \dots, X_{\sigma_0(n_1+n_2)}}, X_{\sigma_0(1)}, 
            \dots, X_{\sigma_0(n_2)}).
        \end{align*}
        Now, \par
        \begin{center}
            \begin{tikzpicture}
            \node (init) at (0, 0) 
            {$\underbrace{1, 2, \dots, n_2}, \underbrace{n_2+1}, \dots, n_1+n_2$};
            \node[below=10pt of init] (initToOne) {$\downarrow\ n_2$ times};
            \node[below=10pt of initToOne] (one)
            {$\underbrace{n_2+1}, \underbrace{1, 2, \dots, n_2}, \dots, n_1+n_2$};
            \node[below=10pt of one] (oneToFinal) {$\downarrow\ (n_1-1)n_2$ times};
            \node[below=10pt of oneToFinal] (final)
            {$\underbrace{n_2+1, \dots, n_1+n_2}, \underbrace{1, 2, \dots, n_2}$};
            \end{tikzpicture}
        \end{center}
        \par
        So \(n_1n_2\) transposes can achieve the desired effect. Therefore, 
        every term in the summation is multiplied by \((-1)^{n_1n_2}\), and 
        we get the desired result.
    \end{proof}
    \begin{theorem}[Dimension of n-Forms]\label{thm:n-form-dimension}
        Let \(\dim \manifold[M]=m\). If \(1 \leq n \leq m\), then \(\form{n}{M}
        = \binom{m}{n}\). If \(n > m\), then \(\form{n}{M}=0\).\\
        Moreover, a basis for \(\form{n}{M}_p\) is given by, 
        \[
        (dx^{\mu_1})_p \wedge (dx^{\mu_2})_p \wedge \dots \wedge 
        (dx^{\mu_n})_p, ~1 \leq \mu_1 \leq \dots \leq \mu_n \leq m.
        \]
    \end{theorem}
    \begin{remark}
        The proof is quite a pleasure to read (and to think of). Please see 
        Hoffman.
    \end{remark}
    \newpage
    \subsection{The Exterior Derivative}
    \begin{definition}[Exterior Derivative]\label{def:exterior-derivative}
        Let \(\omega\) be an n-form on \(\manifold[M]\), \(1 \leq n < \dim 
        \manifold[M]\). Then the exterior derivative \(d\omega\) is a 
        \((n+1)\)-form. Let \(d\omega(\mathbf{X})=d\omega(X_1, \dots, X_{n+1})\), 
        then 
        \begin{align*}
            d\omega(\mathbf{X}) :=&
            \sum_{i=1}^{n+1} (-1)^{i+1} \lder{X_i}(\omega
            (\mathbf{X}\backslash\{X_i\})) \\
            +&\sum_{i < j}^{} (-1)^{i+j} \omega([X_i, X_j], \mathbf{X}\backslash 
            \{X_i, X_j\}).
        \end{align*}
        If \(\omega \in \form{\dim \manifold[M]}{M}\), we define \(d\omega=0\).
    \end{definition}
    \begin{theorem}
        In particular for a 0-form \(f \in C^\infty(\manifold[M])\), 
        \[
        df(X) := \lder{X}f.
        \]
        In coordinates, 
        \[
        df = (\tfld{\mu}f)(dx^\mu).
        \]
    \end{theorem}
    \begin{theorem}
        In particular for a 1-form \(\omega\), 
        \[
        d\omega(X, Y) = \lder{X}(\omega(Y)) - \lder{Y}(\omega(X)) - \omega([X, Y]).
        \]
    \end{theorem}
    \begin{theorem}[Coordinate Expansion for Exterior Derivative]\label{thm:exterior-derivative-coordinates}
        In local coordinates, if \(\omega = \omega\indices
        {_{\mu_1}_{\mu_2}_\dots_{\mu_n}}dx^{\mu_1}\wedge dx^{\mu_2} \wedge \dots 
        \wedge dx^{\mu_n}\), then 
        \[
        d\omega = \tfld{\nu}\omega\indices{_{\mu_1}_{\mu_2}_\dots_{\mu_n}}
        dx^\nu\wedge dx^{\mu_1}\wedge dx^{\mu_2} \wedge \dots \wedge dx^{\mu_n}
        \]
    \end{theorem}
    \begin{theorem}[Exterior Derivative and Product]\label{thm:exterior-leibniz}
        \[
        d(\omega_1 \wedge \omega_2) = d\omega_1 \wedge \omega_2 + 
        (-1)^{\deg \omega_1}\omega_1 \wedge d\omega_2.
        \]
    \end{theorem}
    \begin{theorem}[Exterior Derivative and Pullback]\label{thm:exterior-derivative-pullback}
        Given \(h : \manifold[M] \to \manifold[N]\), \(\omega\) an n-form on 
        \(\manifold[N]\), then 
        \[
        d(h^*\omega) = h^*(d\omega).
        \]
    \end{theorem}
    \begin{theorem}[Functional Linearity of Exterior Derivative]
        \label{thm:f-linear-exterior-derivative}
        Let \(\omega\) be a 1-form on \(\manifold[M]\). Then \(d\omega\) 
        satisfies,
        \[
        d\omega(fX, Y) = fd\omega(X, Y),~\forall f \in C^\infty(\manifold[M]),
        \]
        where \(fX\) is a vector field that gives \((fX)(p) = f(p)X_p\).
    \end{theorem}
    \begin{proof}
        By \cref{def:exterior-derivative}, 
        \[
        d\omega(fX, Y) = \lder{fX}(\omega(Y)) - \lder{Y}(\omega(fX)) - 
        \omega([fX, Y]).
        \]
        We break it down term by term. Firstly, 
        \[
        (\lder{fX}(\omega(Y)))(p) = f(p)X_p(\omega(Y)) = 
        f(p)(\lder{X}(\omega(Y)))(p).
        \]
        So
        \[
        \lder{fX}(\omega(Y)) = f\cdot\lder{X}(\omega(Y)).
        \]
        Secondly, we tackle \(\lder{Y}(\omega(fX))\). In particular, 
        \[
        \omega(fX)(p) = \omega_p(f(p)X_p) = f(p) \omega_p(X_p) = 
        f(p)(\omega(X))(p).
        \]
        Therefore, 
        \[
        \lder{Y}(\omega(fX)) = \lder{Y}(f\cdot\omega(X)) = 
        (\lder{Y}f)\omega(X) + f\cdot\lder{Y}(\omega(X)).
        \]
        Thirdly, 
        \[
        \omega([fX, Y]) = \omega((fX)\circ Y - Y \circ (fX)).
        \]
        In particular, 
        \[
        ((Y \circ (fX))(g))(p) = Y_p((fX)(g)) = Y_p(f\cdot Xg)
        = (Y_pf)((Xg)(p)) + f(p)\cdot Y_p(Xg).
        \]
        So, 
        \[
        Y \circ (fX) = (\lder{Y}f)X + f\cdot Y\circ X.
        \]
        Substituting back, 
        \[
        \begin{aligned}
            \omega([fX, Y]) &= \omega(f\cdot X\circ Y - (\lder{Y}f)X - f\cdot Y\circ X) \\
            &= \omega(f[X, Y] - (\lder{Y}f)X) \\
            &= f\omega([X, Y]) - (\lder{Y}f)\omega(X).
        \end{aligned}
        \]
        Finally, 
        \[
        \begin{aligned}
            d\omega(fX, Y) &= \lder{fX}(\omega(Y)) - \lder{Y}(\omega(fX)) - 
            \omega([fX, Y]) \\
            &= f\cdot\lder{X}(\omega(Y)) - (\lder{Y}f)\omega(X) - 
            f\cdot\lder{Y}(\omega(X)) - f\omega([X, Y]) + (\lder{Y}f)\omega(X) \\
            &= f(\lder{X}(\omega(Y)) - \lder{Y}(\omega(X)) - \omega([X, Y])) \\
            &= fd\omega(X, Y).
        \end{aligned}
        \]
    \end{proof}
    \begin{corollary}[Local Nature of Exterior Derivative]\label{cor:local-exterior-derivative}
        When \(\omega\) is fixed, the value of \(d\omega\) depends only on 
        the local values of vector fields.
        \[
        d\omega(X, Y)(p) = X^\mu(p)Y^\nu(p)d\omega(\tfld{\mu}, \tfld{\nu})(p).
        \]
    \end{corollary}
    \begin{proof}
        Write \(X = X^\mu\tfld{\mu}\), noting that \(X^\mu \in 
        C^\infty(\manifold[M])\), and use \cref{thm:f-linear-exterior-derivative}.
    \end{proof}
    \newpage
    \subsection{DeRham Cohomology}
    \begin{theorem}[Twice Exterior Differential]\label{thm:ddzero}
        For all \(\omega \in \form{n}{M}, 1 \leq n \leq \dim M\), we have 
        \[
        d^2\omega=0.
        \]
    \end{theorem}
    \begin{remark}
        This means 
        \[
        \mathrm{Im}(d : \form{n-1}{M} \to \form{n}{M}) \subseteq 
        \mathrm{Ker}(d : \form{n}{M} \to \form{n+1}{M}).
        \]
        This type of structure is called a differential complex, and is common 
        in many structures.
    \end{remark}
    \begin{definition}[Closed Form]\label{def:closed-form}
        An n-form \(\omega\) is closed if \(d\omega=0\). The set of all closed 
        n-forms is denoted \(Z^n(\manifold[M])\).
    \end{definition}
    \begin{definition}[Exact Form]\label{def:exact-form}
        An n-form \(\omega\) is exact if \(\omega = d\beta\) for some 
        \((n-1)\)-form \(\beta\). The set of all exact n-forms is denoted 
        \(B^n(\manifold[M])\).
    \end{definition}
    \begin{remark}
        It is guaranteed that \(B^n(\manifold[M]) \subseteq Z^n(\manifold[M])\), 
        that is, exactness implies closure.
        But how much closed form is not exact is the study of cohomology theory.
    \end{remark}
    \begin{theorem}[Poincare's Lemma]\label{thm:poincare-lemma}
        On Euclidean space \(\mathbb{R}^m\), 
        \[
        B^n(\manifold[M]) = Z^n(\manifold[M]), ~\forall n > 0.
        \]
    \end{theorem}
    \begin{definition}[DeRham Cohomology Groups]\label{def:derham-groups}
        The DeRham cohomology groups \(H^n(\manifold[M]), 0 \leq n \leq 
        \dim \manifold[M]\) are the quotient spaces
        \[
        H^n(\manifold[M]) := Z^n(\manifold[M]) / B^n(\manifold[M]).
        \]
    \end{definition}
    \begin{remark}
        Recall the definition of quotient groups that \(H^n(\manifold[M])\) 
        consists of elements of form \(z + B^n(\manifold[M]), z \in 
        Z^n(\manifold[M])\). \\
        If all closed forms are exact, \(Z^n(\manifold[M]) \subseteq 
        B^n(\manifold[M])\), then \(H^n(\manifold[M]) \cong \{0\}\).
    \end{remark}
    \begin{theorem}[Criterion of Exact ODE]\label{thm:exact-ode}
        On the Euclidean space \(\mathbb{R}^2\), given a 1-form 
        \(\omega=\omega_1dx^1+\omega_2dx^2\). Then 
        \[
        \omega \in B^1(\mathbb{R}^2) \iff \tfld{2}\omega_1 = \tfld{1}\omega_2.
        \]
    \end{theorem}
    \begin{remark}
        This is an important theorem to me, for it connects the "exactness of 
        differential forms" to the familiar notion of "exactness of 
        differential equations". \\
        It also provides the first hints that we are 
        actually integrating forms, and that exterior differentiation of a 
        0-form resembles gradient in usual vector calculus terms.
    \end{remark}
    \begin{proof}
        Via Poincare lemma \cref{thm:poincare-lemma}, on \(\mathbb{R}^2\), 
        exactness is equivalent to closure. So we need only to determine the 
        condition that \(d\omega=0\). Using 
        \cref{thm:exterior-derivative-coordinates}, 
        \[
        \begin{aligned}
            d\omega &= \tfld{\nu}\omega\indices{_{\mu_1}}dx^\nu 
            \wedge dx^{\mu_1} \\
            &= \tfld{2}\omega_1 dx^2\wedge dx^1 + 
            \tfld{1}\omega_2 dx^1\wedge dx^2 \\
            &= \left(\tfld{2}\omega_1-\tfld{1}\omega_2\right)dx^2\wedge dx^1.
        \end{aligned}
        \]
    \end{proof}
\end{document}