\documentclass[12pt]{article}
\usepackage{mystyle}

\begin{document}
    \section{Differentiable Manifolds}
    \subsection{Definition}
    \subsubsection{Coordinate Charts}
    \begin{definition}[Coordinate Charts]\label{def:coordinate-charts}
        An \(m\)-dimensional, \(m \neq \infty\) coordinate chart on a topological 
        space \(\manifold\) is a pair 
        \[
        (U, \phi)
        \begin{cases}
            U \subseteq \manifold[M], U~\text{open}\\
            \phi : U \to \mathbb{R}^{m}, \phi~\text{homeomorphism}
        \end{cases}
        \]
    \end{definition}
    \begin{remark}
        If \(U=\manifold[M]\), then we say the coordinate chart \(\phi\) is 
        globally defined; if not, then it is locally defined. Few manifolds 
        have globally defined property.
        
    \end{remark}
    \begin{definition}[Overlap Function]\label{def:overlap-function}
        Let \((U_1, \phi_1), (U_2, \phi_2)\) be a pair of \(m\)-dimensional 
        coordinate charts with \(U_1 \cap U_2 \neq \varnothing\). Then the 
        overlap function is defined as 
        \[
        \phi_2\circ\phi_1 ^{-1} : \phi_1(U_1 \cap U_2) \subseteq \mathbb{R}^m 
        \to \phi_2(U_1 \cap U_2) \subseteq \mathbb{R}^m.
        \]
    \end{definition}
    \begin{definition}[Atlas]\label{def:atlas}
        An \(m\)-dimensional atlas on \(\manifold[M]\) is a family of 
        \(m\)-dimensional coordinate charts \((U_i, \phi_i), i \in I\) s.t. 
        \begin{enumerate}
            \item \(\manifold[M] = \bigcup_{i \in I}^{} U_i\).
            \item Each overlap function \(\phi_j \circ \phi_i ^{-1}, i, j \in I\) 
            is \(C ^{\infty}\).
        \end{enumerate}
    \end{definition}
    \begin{definition}[Differentiable Manifolds]\label{def:differentiable-manifolds}
        An \(m\)-dimensional differentiable manifold is a topological space 
        \(\manifold[M]\) equipped with an atlas.
    \end{definition}
    \begin{remark}
        We didn't define a differentiable manifold by regulating the 
        differentiability of the coordinate charts themselves. That's because 
        differentiation is not defined on a manifold, so we need to rely on 
        Euclidean spaces.
        
    \end{remark}
    \begin{definition}[Coordinate Functions]\label{def:coordinate-functions}
        The coordinate functions are the (Euclidean) components of coordinate.
        % \[
        \begin{align*}
            \phi &: U \to \mathbb{R}^m & p \mapsto &\phi(p), \\
            \phi^\mu &: U \to \mathbb{R} & ~\text{s.t.}~ &\phi(p) = \begin{pmatrix}
            \phi^1(p) \\ \vdots \\ \phi^m(p)
            \end{pmatrix}.
        \end{align*}
        % \]
        An alternative notation is 
        \[
        x^\mu := \phi^\mu.
        \]
    \end{definition}
    \begin{remark}
        There are (Euclidean) projection functions, 
        \[
        u^\mu : \mathbb{R}^m \to \mathbb{R}.
        \]
        But I think mention it will cause a lot of confusion. Just remember in 
        the future when we say \(\frac{\partial }{\partial u^\mu}\), we are 
        referring to the Euclidean partial derivative wrt the \(\mu\)-th 
        component.
        
    \end{remark}
    \newpage
    \section{Tangent Spaces}
    \subsection{The Curve Formulation of Tangent Spaces}
    \begin{remark}
        The definition of manifold do not require the entity to be embeded in 
        a higher dimensional space. Therefore, the traditional view of tangency 
        is not valid here.
        
    \end{remark}
    \begin{remark}
        The curve formulation remains valid in the infinite-dimensional case, 
        while the algebraic formulation is not. However, in the 
        finite-dimensional case, they are isomorphic.
        
    \end{remark}

    \subsubsection{Curves and Vectors}
    \begin{definition}[Curve]\label{def:curve}
        A curve on \(\manifold[M]\) is a \(C^\infty\) map,
        \[
        \sigma : (-\epsilon, \epsilon) \to \manifold[M].
        \]
    \end{definition}
    \begin{definition}[Curve Tangency]\label{def:curve-tangency}
        Two curves \(\sigma_1, \sigma_2\) are tangent at \(p \in \manifold[M]\) 
        if 
        \begin{enumerate}
            \item \(\sigma_1(0)=\sigma_2(0)=p\).
            \item \(\left.\frac{d}{dt}(x^i \circ \sigma_1(t))\right|_{t=0}
             = \left.\frac{d}{dt}(x^i \circ \sigma_2(t))\right|_{t=0},~1 \leq 
             i \leq m\) .
        \end{enumerate}
    \end{definition}
    \begin{remark}
        Written more compactly, 
        \[
        \left.\frac{d}{dt}(\phi\circ\sigma_1)\right|_{t=0}
        =\left.\frac{d}{dt}(\phi\circ\sigma_2)\right|_{t=0}
        \]
        
    \end{remark}
    \begin{definition}[Tangent Vectors]\label{def:tangent-vector-curve}
        A tangent vector at \(p \in \manifold[M]\) is an equivalence class of 
        curves where the equivalence relation is that they are tangent. It will 
        be denoted as 
        \[
        v = [\sigma].
        \]
    \end{definition}
    \begin{definition}[Tangent Space]\label{def:tangent-space-curve}
        The tangent space \(T_p\manifold[M]\) at point \(p\) is the set of all 
        tangent vectors at point \(p\).
    \end{definition}
    \begin{definition}[Tangent Bundle]\label{def:tangent-bundle-curve}
        The tangent bundle \(T\manifold[M]\) is 
        \[
        T\manifold[M] := \bigcup_{p \in \manifold[M]}^{} T_p\manifold[M].
        \]
    \end{definition}
    
    \subsubsection{Addition and Scalar Multiplication}
    \begin{definition}[Addition and Scalar Multiplication]\label{def:addition-and-scalar-multiplication-curve}
        Let \(v_1 = [\sigma_1], v_2 = [\sigma_2] \in T_p\manifold[M]\), and 
        \(r \in \mathbb{R}\). Then 
        define 
        \[
        \begin{aligned}
            v_1+v_2 &:= [\phi ^{-1}\circ(\phi\circ\sigma_1+\phi\circ\sigma_2)], \\
            rv_1 &:= [\phi ^{-1} \circ (r\phi \circ \sigma_1)].
        \end{aligned}
        \]
    \end{definition}
    \begin{theorem}
        The definition \ref{def:addition-and-scalar-multiplication-curve}
        is well-defined. That is, they are independent of the 
        choice of chart \((U, \phi)\) and \(\sigma_1, \sigma_2\) as long as 
        \(v_1=[\sigma_1]\) and \(v_2=[\sigma_2]\).\par
        Therefore, \(T_p\manifold[M]\) is a real vector space.
    \end{theorem}
    \begin{proof}
        Let \(v_1 = [\sigma_1] = v_1' := [\tau_1], 
        v_2 = [\sigma_2] = v_2' := [\tau_2]\). 
        First check (1) of \ref{def:curve-tangency}, 
        \[
        \begin{aligned}
            (rv_1+v_2)(0) &= (\phi ^{-1} \circ (r\phi\circ\sigma_1(0) + 
            \phi\circ\sigma_2(0))) \\
            &= (\phi ^{-1} \circ (r\phi\circ\tau_1(0) + 
            \phi\circ\tau_2(0))) \\
            &= (rv_1'+v_2')(0),
        \end{aligned}
        \]
        since \(\phi\circ\sigma_1(0) = \phi\circ\tau_1(0) = \phi(p)\) by 
        equivalence, and the same for \(\sigma_2\). \par
        Now consider 
        \[
        \begin{aligned}
            \left.\frac{d}{dt}(\phi\circ(rv_1+v_2))\right|_{t=0}
            &=\left.\frac{d}{dt}(r\phi\circ\sigma_1+\phi\circ\sigma_2)
            \right|_{t=0}\\
            &=r\left.\frac{d}{dt}(\phi\circ\sigma_1)\right|_{t=0}+
            \left.\frac{d}{dt}(\phi\circ\sigma_2)\right|_{t=0}\\
            &=r\left.\frac{d}{dt}(\phi\circ\tau_1)\right|_{t=0}+
            \left.\frac{d}{dt}(\phi\circ\tau_2)\right|_{t=0}\\
            &=\left.\frac{d}{dt}(\phi\circ(rv_1'+v_2'))\right|_{t=0},
        \end{aligned}
        \]
        since \(\left.\frac{d}{dt}(\phi\circ\sigma_1)\right|_{t=0}
        =\left.\frac{d}{dt}(\phi\circ\tau_1)\right|_{t=0}\) by equivalence, and 
        the same for \(\sigma_2\).
    \end{proof}

    \subsubsection{Curves and Derivation}
    \begin{definition}[Directional Derivative]\label{def:directional-derivative-curve}
        For any \(f : \manifold[M] \to \mathbb{R}\) s.t. \(f \in C^\infty\), we 
        define
        \[
        v(f) := \left.\frac{d}{dt}(f \circ \sigma(t))\right|_{t=0},
        \]
        where \(v = [\sigma]\).
    \end{definition}
    \begin{theorem}
        The definition \ref{def:directional-derivative-curve} is well-defined.
        That is, \(v(f)\) is independent of the curve \(\sigma\) chosen as well 
        as \(v=[\sigma]\).
    \end{theorem}
    \begin{proof}
        Let \(v_1 = [\sigma_1] = [\sigma_2] = v_2\). Then 
        \[
        \begin{aligned}
            v_1(f) &= \left.\frac{d}{dt}(f \circ \sigma_1)\right|_{t=0}, \\
            v_2(f) &= \left.\frac{d}{dt}(f \circ \sigma_2)\right|_{t=0}, \\
        \end{aligned}
        \]
        \[
        \left.\frac{d}{dt}(\phi\circ\sigma_1)\right|_{t=0}
            =\left.\frac{d}{dt}(\phi\circ\sigma_2)\right|_{t=0}.
        \] \par
        Then
        \[
        \begin{aligned}
            v_1(f) &= \left.\frac{d}{dt}
            (\underbrace{(f\circ\phi ^{-1})}_{\mathbb{R}\leftarrow \mathbb{R}^m} \circ 
            \underbrace{(\phi \circ \sigma_1)}_{\mathbb{R}^m\leftarrow \mathbb{R}})
            \right|_{t=0} \\
            &= \left.(f \circ \phi ^{-1})'\circ(\phi\circ\sigma_1)\cdot
            (\phi\circ\sigma_1)'\right|_{t=0} \\
            &= \left.(f \circ \phi ^{-1})'\circ(\phi\circ\sigma_2)\cdot
            (\phi\circ\sigma_2)'\right|_{t=0} \\
            &= v_2(f),
        \end{aligned}
        \]
        since \(\phi\circ\sigma_1(0) = \phi\circ\sigma_2(0) =\phi(p)\), and 
        \((\phi\circ\sigma_1)'=(\phi\circ\sigma_2)'\) by equivalence.
    \end{proof}

    \subsection{The Algebraic Formulation of Tangent Spaces}
    \subsubsection{The Space of Derivations}
    \begin{definition}[Derivation]\label{def:derivation}
        A derivation at \(p \in \manifold[M]\) is a map \(v : C^\infty
        (\manifold[M]) \to \mathbb{R}\) s.t. 
        \begin{enumerate}
            \item \(v(rf+g)=rv(f)+v(g)\), (Linear)
            \item \(v(fg) = f(p)v(g)+g(p)v(f)\), (Leibniz)
        \end{enumerate}
        where \(f, g \in C^\infty\).
    \end{definition}
    \begin{definition}[Tangent Space (Algebraic)]\label{def:tangent-space-algebraic}
        The space of all derivations at \(p \in \manifold[M]\) is denoted 
        \(D_p\manifold[M]\).
    \end{definition}
    \begin{definition}[Addition and Scalar Multiplication]\label{def:addition-scalar-multiplication-algebraic}
        Given \(v_1, v_2 \in D_p\manifold[M]\), define 
        \[
        \begin{aligned}
            (v_1+v_2)(f) &:= v_1(f)+v_2(f) \\
            (rv)(f) &:= rv(f).
        \end{aligned}
        \]
    \end{definition}
    \begin{theorem}
        \(D_p\manifold[M]\) is a real vector space.
    \end{theorem}
    \subsubsection{The Basis Tangent Vectors}
    \begin{definition}[Basis Tangent Vectors]\label{def:basis-derivations}
        We define the basis tangent vectors via derivations by 
        \[
        \tvec{\mu}{p}f := \left.\frac{\partial }{\partial u^\mu}
        (f\circ\phi ^{-1}(\vec{u}))\right|_{\vec{u}=\phi(p)}, 1 \leq \mu \leq 
        \dim \manifold[M].
        \]
        where \(u \in \mathbb{R}^m, f : \manifold[M] \to \mathbb{R}, f \in 
        C^\infty\). For the use of \(u^\mu\), see \ref{def:coordinate-functions}.
    \end{definition}
    \begin{theorem}
        \[
        \tvec{\mu}{p}x^\nu = \delta\indices{^\nu_\mu}.
        \]
    \end{theorem}
    \begin{proof}
        Although a simple exercise, it was a good chance to explain the 
        sophisticated notation.
        \[
        \tvec{\mu}{p}x^\mu = \left.\frac{\partial }{\partial u^\mu}
        \left(x^\mu\circ \phi ^{-1} \begin{pmatrix}
        u^1 \\ \vdots \\ u^m
        \end{pmatrix}\right)\right|_{\phi(p)}.
        \]
        The coordinate \(u \in \mathbb{R}^m\) was brought to \(\manifold[M]\) and 
        projected to \(\mathbb{R}^m\) again and taken out the \(\mu\)-th 
        component. So 
        \[
        =\left.\frac{\partial }{\partial u^\mu}(u^\mu)\right|_{\phi(p)}=1.
        \]
    \end{proof}
    \begin{theorem}[Linear Independence of Basis Tangent Vectors]\label{thm:linear-indep-basis}
        The basis tangent vectors \(\tvec{\mu}{p}, 1 \leq \mu \leq \dim 
        \manifold[M]\) are linear independent.
    \end{theorem}
    \begin{proof}
        Suppose \(a^\mu\tvec{\mu}{p}=0\). Then 
        \[
        a^\mu\tvec{\mu}{p}(x^\nu)=a^\mu\delta\indices{^\nu_\mu}=0(x^\nu)=0.
        \]
        So \(a^\mu=0\).
    \end{proof}
    \begin{theorem}[Coordinate Expansion of Tangent Vectors]\label{thm:tangent-vector-coordinates}
        For all \(v \in D_p\manifold[M]\), we have 
        \[
        v=v^\mu\tvec{\mu}{p},
        \]
        where Einstein notation was used, and \(v^\mu = v(x^\mu)\).
    \end{theorem}
    \begin{remark}
        The proof was sophisticated and did not teach me much.
        
    \end{remark}

    \subsection{Isomorphism of Curves and Derivations}
    \begin{theorem}[Isomorphism of Curves and Derivations]\label{thm:curve-derivation-iso}
        Similar to \ref{def:directional-derivative-curve}, we define the linear 
        map \(\iota : T_p\manifold[M] \to D_p\manifold[M]\) acting on 
        \(v = [\sigma] \in T_p\manifold[M]\) by 
        \[
        \iota(v)(f) := \left.\frac{d}{dt}(f\circ \sigma(t))\right|_{t=0}.
        \]
        Then \(\iota\) is a linear isomorphism. Note that RHS \(\in 
        D_p\manifold[M]\).
    \end{theorem}
    \begin{proof}
        (linearity) Choose \(\phi\) s.t. \(\phi(p) = 0\).
        \[
        \begin{aligned}
            \iota(rv_1+v_2)(f) &= \left.\frac{d}{dt}
            (f\circ \phi ^{-1} \circ (r\phi\circ\sigma_1+\phi\circ\sigma_2))
            \right|_{t=0} \\
            &= \left.((f\circ\phi ^{-1})'\circ(r\phi\circ\sigma_1+\phi\circ\sigma_2)
            \cdot (r\phi\circ\sigma_1+\phi\circ\sigma_2)')\right|_{t=0} \\
            &= \left.((f\circ\phi ^{-1})'(0)
            \cdot ((r\phi\circ\sigma_1)'+(\phi\circ\sigma_2)'))\right|_{t=0} \\
            &= \left.(r(f\circ\phi ^{-1})'\circ(\phi\circ\sigma_1)
            \cdot(\phi\circ\sigma_1)'+(f\circ\phi ^{-1})'\circ(\phi\circ\sigma_2)
            \cdot (\phi\circ\sigma_2)')\right|_{t=0} \\
            &=r\iota(v_1)+\iota(v_2)(f).
        \end{aligned}
        \]
        (surjectivity) Since \(\iota\) is linear, surjectivity is equivalent to 
        injectivity and therefore to bijectivity. To show surjectivity, we need 
        to construct a curve for all \(v' \in D_p\manifold[M]\) s.t. \(\iota(v)
        = v'\). \par
        Let \(v' \in D_p\manifold[M]\) and construct \(\sigma : (-\epsilon, 
        \epsilon) \to \manifold[M]\) s.t. 
        \[
        \begin{aligned}
            \sigma(0) &= p, \\
            v^\mu = v(x^\mu) &= \left.\frac{d}{dt}(x^\mu\circ\sigma(t))\right|_{t=0}.
        \end{aligned}
        \]
        Then 
        \[
        v(f) = v^\mu \tvec{\mu}{p}f=\left.\frac{d}{dt}
        (x^\mu\circ\sigma(t))\right|_{t=0}\tvec{\mu}{p}f.
        \]
        Also,
        \[
        \begin{aligned}
            \left.\frac{d}{dt}(f\circ\sigma(t))\right|_{t=0}
            &= \left.\frac{d}{dt}
            (f\circ\phi ^{-1} \circ \phi\circ\sigma(t))\right|_{t=0} \\
            &= \sum_{\mu=1}^{m} \left.\frac{\partial }{\partial u^\mu}
            (f\circ\phi ^{-1})\right|_{\phi(p)}\left.\frac{d}{dt}
            (u^\mu\circ\phi\circ\sigma)\right|_{t=0} ~\text{(component-wise)} \\
            &= \sum_{\mu=1}^{m} \tvec{\mu}{p}f \left.\frac{d}{dt}
            (x^\mu\circ\sigma)\right|_{t=0} \\
            &= v(f).
        \end{aligned}
        \]
        Thus completing the proof.
    \end{proof}

    \subsection{Pushforward}
    \subsubsection{Definition and Linearity}
    \begin{remark}
        The pushforward \(h_* : T_p\manifold[M] \to 
        T_{h(p)}\manifold[N]\) of a specific function \(h : \manifold[M] \to 
        \manifold[N]\) can be thought of as local linearization of the function.
        
    \end{remark}
    \begin{definition}[Pushforward]\label{def:pushforward}
        Given a function \(h : \manifold[M] \to \manifold[N]\) and 
        \(v \in T_p\manifold[M]\), then we define the pushforward 
        \(h_* : T_p\manifold[M] \to T_{h(p)}\manifold[N]\) by 
        \[
        h_*(v) := [h \circ \sigma],~v=[\sigma].
        \]
    \end{definition}
    \begin{theorem}
        The pushforward operation \ref{def:pushforward} is well-defined. That is, 
        \(h_*(v_1) = h_*(v_2)\) if \(v_1 = [\sigma_1] = [\sigma_2] = v_2\).
    \end{theorem}
    \begin{theorem}[Algebraic Definition of Pushforward]\label{thm:algebraic-pushforward}
        The definition of pushforward \ref{def:pushforward} is equivalent to the 
        following: let \(h : \manifold[M] \to \manifold[N]\), \(h_* : 
        D_p\manifold[M] \to D_{h(p)}\manifold[M]\) is defined by,
        \[
        (h_*v)(f) := v(f\circ h).
        \]
    \end{theorem}
    \begin{proof}
        (\textrightarrow) 
        \[
        \begin{aligned}
            h_*(v)(f) = [h \circ \sigma](f) 
            &= \left.\frac{d}{dt}(f\circ h\circ\sigma(t))\right|_{t=0} \\
            &= \left.\frac{d}{dt}((f\circ h)\circ\sigma(t))\right|_{t=0} \\
            &:= v(f\circ h).
        \end{aligned}
        \] \par
        (\textleftarrow) This direction is similar.
    \end{proof}
    \begin{theorem}[Linearity of Pushforward]\label{thm:linearity-of-pushforward}
        The pushforward map \(h_* : T_p\manifold[M] \to T_{h(p)}\manifold[N]\) 
        is linear.
        \[
        h_*(rv_1+v_2) = rh_*(v_1) + h_*(v_2).
        \]
    \end{theorem}
    \begin{proof}
        (Using \ref{def:pushforward})
        Let \(p \in (U, \phi) \subseteq \manifold[M]\), and 
        \(h(p) \in (V, \psi) \subseteq \manifold[N]\). Choose \(\phi\) s.t. 
        \(\phi(p) = 0\).
        It is obvious that \(h_*(rv_1+v_2)(0) = (rh_*(v_1)+h_*(v_2))(0) = h(p)\). 
        \par
        Consider 
        \[
        \begin{aligned}
            \left.\frac{d}{dt}
            \underbrace{(\psi \circ h_*(rv_1+v_2))}_{\mathbb{R}^n \leftarrow 
            \manifold[N] \leftarrow \mathbb{R}}\right|_{t=0}
            &= \left.\frac{d}{dt}(\underbrace{\psi\circ h\circ (\phi ^{-1}}
            _{\mathbb{R}^n \leftarrow \manifold[N] \leftarrow \manifold[M] 
            \leftarrow \mathbb{R}^m}\circ
            \underbrace{(r\phi\circ\sigma_1+\phi\circ\sigma_2)}_{\mathbb{R}^m 
            \leftarrow \manifold[M] \leftarrow \mathbb{R}}))\right|_{t=0} \\
            &= \left.(\psi \circ h \circ \phi ^{-1})'\circ 
            (r\phi\circ\sigma_1+\phi\circ\sigma_2) \cdot (r\phi\circ\sigma_1 + 
            \phi\circ\sigma_2)'\right|_{t=0} \\
            &= \left.(\psi \circ h \circ \phi ^{-1})'(0) \cdot 
            ((r\phi\circ\sigma_1)' + (\phi\circ\sigma_2)')\right|_{t=0}.
        \end{aligned}
        \]
        And 
        \[
        \begin{aligned}
            \left.\frac{d}{dt}
            (\underbrace{\psi}_{\mathbb{R}^n\leftarrow} \circ 
            \underbrace{(rh_*(v_1)+h_*(v_2))}_{\manifold[N]\leftarrow 
            \mathbb{R}})
            \right|_{t=0}
            &= \left.\frac{d}{dt}
            \underbrace{(r\psi\circ h \circ \sigma_1 + \psi \circ h \circ \sigma_2)}_
            {\mathbb{R}^n \leftarrow \manifold[N] \leftarrow \manifold[M] \leftarrow 
            \mathbb{R}}
            \right|_{t=0} \\
        \end{aligned}
        \]
        \[
        \begin{aligned}
        =&(\underbrace{r\psi\circ h\circ \phi ^{-1}}_
        {\mathbb{R}^n\leftarrow\manifold[N]\leftarrow\manifold[M]
        \leftarrow\mathbb{R}^m} \circ \underbrace{\phi \circ \sigma_1}_
        {\mathbb{R}^m\leftarrow\manifold[M]\leftarrow\mathbb{R}} + 
        \psi\circ h\circ \phi ^{-1} \circ \phi \circ \sigma_2
        )'|_{t=0} \\
        =&(r(\psi\circ h\circ \phi ^{-1})'\circ(\phi\circ\sigma_1)
        \cdot(\phi\circ\sigma_1)')|_{t=0} \\
        &+((\psi\circ h\circ \phi ^{-1})'\circ(\phi\circ\sigma_2)
        \cdot(\phi\circ\sigma_2)')|_{t=0} \\
        =&(\psi\circ h\circ \phi ^{-1})'(0)\cdot(r(\phi\circ\sigma_1)' + 
        (\phi\circ\sigma_2)')|_{t=0}.
        \end{aligned}
        \]
        So we see the two are equal. \par
        (Using \ref{thm:algebraic-pushforward})
        \[
        \begin{aligned}
            (h_*(rv_1+v_2))(f) &= (rv_1+v_2)(f\circ h) \\
            &= rv_1(f\circ h)+v_2(f\circ h) \\
            &= r(h_*v_1)f+(h_*v_2)f.
        \end{aligned}
        \]
    \end{proof}
    \begin{theorem}
        Given manifolds \(\manifold[M], \manifold[N], \manifold[P]\) and 
        \(h : \manifold[M] \to \manifold[N]\), \(k : \manifold[N] \to 
        \manifold[P]\), then 
        \[
        (k\circ h)_*=k_*\circ h_*.
        \]
    \end{theorem}

    \subsubsection{Jacobian}
    \begin{theorem}[Local Representative of Pushforward]\label{thm:pushforward-local}
        Let \(\dim \manifold[M]=m, \dim \manifold[N]=n\), \(h : \manifold[M] 
        \to \manifold[N]\), \(\{x^1, \dots, x^m\}\) be the local coordinates of 
        \(\manifold[M]\) around \(p\), and \(\{y^1, \dots, y^n\}\) be the local 
        coordinates of \(\manifold[N]\) around \(h(p)\). Then 
        \[
        h_*v = \sum_{\mu=1}^{m} \sum_{\nu=1}^{n} v^\mu 
        \left.\frac{\partial h^\nu}{\partial x^\mu}\right|_{p}\tvec{\nu}{h(p)},
        \]
        where \(J_{\nu\mu}:=\left.\frac{\partial h^\nu}{\partial x^\mu}\right|_{p}
        :=\tvec{\mu}{p}(y^\nu\circ h)\) is the inverse Jacobian matrix.
    \end{theorem}
    \begin{proof}
        First expand \(v\) in terms of local coordinates and use linearity, 
        \[
        h_*v = h_*(v^\mu \tvec{\mu}{p}) = v^\mu h_*(\tvec{\mu}{p}).
        \]
        Expand the result in local coordinates of \(\manifold[N]\), 
        \[
        h_*(\tvec{\mu}{p}) = \left(h_*\tvec{\mu}{p}\right)^\nu\tvec{\nu}{h(p)}.
        \]
        Using \ref{thm:algebraic-pushforward}, 
        \[
        \begin{aligned}
            \left(h_*\tvec{\mu}{p}\right)^\nu 
            &= \left(h_*\tvec{\mu}{p}\right)\circ y^\nu \\
            &= \tvec{\mu}{p}(y^\nu \circ h) \\
            &:= \tvec{\mu}{p}h^\nu.
        \end{aligned}
        \]
        So, 
        \[
        h_*(\tvec{\mu}{p}) = \tvec{\mu}{p}h^\nu\tvec{\nu}{h(p)}.
        \]
        And, 
        \[
        h_*v = v^\mu\tvec{\mu}{p}h^\nu\tvec{\nu}{h(p)}.
        \]
    \end{proof}
    \begin{theorem}[Using Curve to Pushforward]\label{thm:curve-pushforward}
        Given \(c : (-\epsilon, \epsilon) \to \manifold[M]\) a curve, and choose 
        the coordinate chart of \(\mathbb{R}\) to be the identity, then 
        \[
        c_*\left(\frac{d}{dt}\right)_0=[c] \in T_p\manifold[M].
        \]
    \end{theorem}
    \begin{proof}
        First we clarify what is \(\left(\frac{d}{dt}\right)_0\). Since on the 
        trivial manifold \(\mathbb{R}\) there is only one coordinate, namely 
        \(t\), we need not specify the number. Also, considering our functions 
        are scalar valued \(f : \manifold[M] \to \mathbb{R}\), this motivates us 
        to write "total differential". \par
        For all \(f \in C^\infty\), 
        \[
        c_*\left(\frac{d}{dt}\right)_0f=\left(\frac{d}{dt}\right)_0(f\circ c).
        \]
        Since the coordinate chart is the identity, 
        \[
        \begin{aligned}
            \left(\frac{d}{dt}\right)_0(f\circ c) 
            &= \left.\frac{d}{dt}(f\circ c\circ I)\right|_{I(t)=0} \\
            &= \left.\frac{d}{dt}(f\circ c)\right|_{t=0} \\
            &=[c]f.
        \end{aligned}
        \]
    \end{proof}
    \begin{theorem}[Contravariancy of Tangent Vectors]\label{thm:contravariant-vector}
        The components of tangent vectors are contravariant, i.e., given 
        two coordinate charts \((U, \phi)\) and \((U', \phi')\) s.t. 
        \(U \cap U' = S \neq \varnothing\), then on \(S\), 
        \[
        v'^\nu = \sum_{\mu=1}^{m} v^\mu\frac{\partial x'^{\nu}}{\partial x^\mu}.
        \]
    \end{theorem}
    \begin{proof}
        Given that we have the local representative of pushforward at hand, 
        consider the identity pushforward \(\idd_* : T_p\manifold[M] \to T_p
        \manifold[M]\),
        \[
        \idd_*v = \sum_{\mu=1}^{m} \sum_{\nu=1}^{m} v^\mu 
        \left.\frac{\partial x'^\nu}{\partial x^\mu}\right|_p\tvec{\nu'}{p}.
        \]
        We see immediately that the result holds.
    \end{proof}
    \newpage
    \section{Vector Fields}
    \subsection{Definition}
    \begin{definition}[Vector Fields]\label{def:vector-fields}
        A vector field \(X\) on \(\manifold[M]\) is a smooth assignment of 
        a tangent vector \(X_p \in T_p\manifold[M]~\forall p \in \manifold[M]\). 
        \par
        "Smooth" assignment is defined to be that the Lie derivative 
        \ref{def:lie-derivative} is smooth.
    \end{definition}
    \begin{definition}[Lie Derivative]\label{def:lie-derivative}
        The Lie-derivative of function \(f\) with respect to vector field 
        \(X\) is defined as 
        \[
        \lder{X}f := Xf, 
        \]
        and at a specific point \(p \in \manifold[M]\), 
        \[
        \lder{X}f(p) := Xf(p) := X_pf.
        \]
    \end{definition}
    \begin{theorem}[Properties of Lie Derivative]\label{thm:lie-derivative-properties}
        The Lie derivative has the following properties,
        \begin{enumerate}
            \item \(X(rf+g) = rXf+Xg\)
            \item \(X(fg) = fXg+gXf\).
        \end{enumerate}
    \end{theorem}
    \begin{theorem}[Component of Vector Field]\label{thm:vector-field-component}
        \textbf{Given a chart} \((U, \phi)\) on \(\manifold[M]\), we can write 
        \[
        X_U = X_Ux^\mu \tfld{\mu}.
        \]
        When the context is clear or \textbf{for convenience}, we write 
        \[
        X = Xx^\mu \tfld{\mu} := X^\mu \tfld{\mu}.
        \]
    \end{theorem}
    \begin{proof}
        We know 
        \[
        (Xf)(p) = X_pf = X_px^\mu\tvec{\mu}{p}f = (Xx^\mu)(p)\tvec{\mu}{p}f.
        \]
    \end{proof}
    \begin{remark}
        \(\tfld{\mu}\) is a vector field that assigns each point \(p \in 
        \manifold[M]\) with the vector \(\tvec{\mu}{p} \in T_p\manifold[M]\).
        
    \end{remark}
    \begin{theorem}[Contravariancy of Vector Fields]\label{thm:}
        Given two coordinate charts \((U, \phi)\) and \((U', \phi')\) s.t. 
        \(U \cap U' = S \neq \varnothing\). On \(S\), 
        \[
        X^{\nu'} = \sum_{\mu=1}^{m} X^\mu 
        \frac{\partial x'^{\nu}}{\partial x^\mu}.
        \]
        Analogous to \ref{thm:contravariant-vector}.
    \end{theorem}
    \subsection{Lie Bracket}
    \begin{definition}[Composition of Vector Fields]\label{def:vector-field-composition}
        We can view \(X : C^\infty(\manifold[M]) \to C^\infty(\manifold[M])\), 
        and so does \(Y\). Therefore, we define 
        \[
        (X \circ Y)(f) := X(Yf).
        \]
    \end{definition}
    \begin{definition}[Lie Bracket (Commutator)]\label{def:lie-bracket}
        We define the Lie Bracket of two vector fields \(X, Y\) to be 
        \[
        [X, Y] := X\circ Y - Y\circ X.
        \]
    \end{definition}
    \begin{remark}
        Lie Bracket \ref{def:lie-bracket} is a vector field, while the 
        expression \(X\circ Y\) is not, because it contains second differential 
        terms. See the following proof.
        
    \end{remark}
    \begin{theorem}[Lie Bracket Components]\label{thm:lie-bracket-components}
        \[
        [X, Y]^\mu = (X^\nu\tfld{\nu}Y^\mu - Y^\nu\tfld{\nu}X^\mu).
        \]
    \end{theorem}
    \begin{proof}
        Given \(X=X^\mu\tfld{\mu}, Y = Y^\nu\tfld{\nu}\), we try to write 
        the component of \(X\circ Y\).
        \[
        X\circ Y(f) = X^\mu \tfld{\mu}\left(Y^\nu \tfld{\nu}f\right).
        \]
        However, notice that 
        \[
        \begin{aligned}
            &Y^\nu := Yx^\nu \in C^\infty(\manifold[M]); \\
            &\tfld{\nu} : C^\infty(\manifold[M]) \to C^\infty(\manifold[M]), \\
            &\implies\tfld{\nu}f \in C^\infty(\manifold[M]).
        \end{aligned}
        \]
        So we need to use the Leibniz property of \(\tfld{\mu}\) 
        \ref{def:derivation} in order to evaluate the second term. Doing this 
        for \(X\circ Y(f)\) and \(Y\circ X(f)\), we have
        \[
        \begin{aligned}
            X\circ Y(f) &= X^\mu \left((\tfld{\mu}Y^\nu)(\tfld{\nu}f)+
            Y^\nu \tfld{\mu}\tfld{\nu}f\right). \\
            Y\circ X(f) &= Y^\nu \left((\tfld{\nu}X^\mu)(\tfld{\mu}f)+
            X^\mu \tfld{\nu}\tfld{\mu}f\right).
        \end{aligned}
        \]
        So if \(\tfld{\mu}\tfld{\nu}f=\tfld{\nu}\tfld{\mu}f\), then by 
        subtracting, we can cancel the second order terms, and we are done. We 
        prove so now.
        \[
        \begin{aligned}
            (\tfld{\mu}\tfld{\nu}f)(p)&=\frac{\partial }{\partial u^\mu}
            \left.\left((\tfld{\nu}f)\circ\phi ^{-1}\right)\right|_{\phi(p)} \\
            &=\frac{\partial }{\partial u^\mu}
            \left.\left(\tvec{\nu}{\phi ^{-1}(u)}f\right)\right|_{\phi(p)} \\
            &=\frac{\partial }{\partial u^\mu}
            \left.\left(\left.
                \frac{\partial }{\partial u^\nu}(f\circ\phi ^{-1})
            \right|_u\right)\right|_{\phi(p)} \\
            &=\frac{\partial }{\partial u^\nu}
            \left.\left(\left.
                \frac{\partial }{\partial u^\mu}(f\circ\phi ^{-1})
            \right|_u\right)\right|_{\phi(p)} \\
            &=(\tfld{\nu}\tfld{\mu}f)(p).
        \end{aligned}
        \]
    \end{proof}
    \begin{theorem}[Properties of Lie Brackets]\label{thm:lie-bracket-properties}
        \begin{enumerate}
            \item \([X,Y] = -[Y, X]\) (antisymmetry)
            \item \(\sum_{\text{cyc}}^{} [X, [Y, Z]]=0\). (Jacobi Identity)
        \end{enumerate}
    \end{theorem}
    \subsection{Integral Curves and Flows}
    \begin{definition}[Intergral Curve]\label{def:intergral-curve}
        Let \(X\) be a vector field on \(\manifold[M]\), \(p \in \manifold[M]\). 
        Then an integral curve of \(X\) through \(p\) is a curve \(\sigma : 
        (-\epsilon, \epsilon) \to \manifold[M]\) s.t. 
        \[
        \begin{aligned}
            \sigma(0) &= p, \\
            \sigma_*\left(\frac{d}{dt}\right)_t &= X_{\sigma(t)}.
        \end{aligned}
        \]
    \end{definition}
    \begin{remark}
        Qualitatively, using \ref{thm:curve-pushforward}, this pushforward is 
        just \([\sigma] \in T_{\sigma(t)}\manifold[M]\). Therefore, the second 
        condition is saying in some sense that the curve is tangent to the 
        vector field on the manifold. For quantitative description, see below.
        
    \end{remark}
    \begin{definition}[Differential Equations of Integral Curve]\label{def:de-int-curve}
        The components \(X^\mu\) of \(X\) determine the integral curve \(\sigma\) 
        by the following ODE with boundary conditions, 
        \[
        \begin{aligned}
            X^\mu(\sigma(t)) &= \frac{d}{dt}x^\mu(\sigma(t)) \\
            x^\mu(\sigma(0)) &= x^\mu(p), \mu = 1, 2, \dots, m.
        \end{aligned}
        \]
    \end{definition}
    \subsubsection{One-parameter Family of Diffeomorphisms}
    \begin{definition}[Local 1D Family of Local Diffeomorphisms]\label{def:local-diff}
        A local, 1D family of local diffeomorphisms at \(p \in \manifold[M]\) 
        is made up of (1) an open neighborhood \(U\) of \(p\), (2) \(\epsilon 
        > 0\) (3) a family of diffeomorphisms \(\Set{\phi_t|\left|t\right|<\epsilon}\), 
        \(\phi_t : U \to \manifold[M]\) s.t. 
        \begin{enumerate}
            \item Every \(\phi_t\) is a smooth function in \(t\) and \(q\).
            \item \(\forall t, s \in \mathbb{R}\) and \(|t|, |s|, |t+s| < \
            \epsilon\), and \(\forall q \in U\) s.t. \(\phi_t(q), \phi_s(q), 
            \phi_{t+s}(q) \in U\), we have 
            \[
            \phi_s(\phi_t(q)) = \phi_{s+t}(q).
            \]
            \item \(\phi_0(q)=q\).
        \end{enumerate}
    \end{definition}
    \begin{remark}
        The first "local" refers to the parameter \(t\), which is limited to 
        \((-\epsilon, \epsilon)\). The second "local" refers to the spatial 
        limitation to \(U\). \par
        You can view \(\phi_t(q)\) as a curve that brings \(t \in (-\epsilon, 
        \epsilon)\) to \(\phi_t(q) \in \manifold[M]\).
        
    \end{remark}
    \begin{definition}[Induced Vector Field]\label{def:induced-vector-field}
        By taking tangents to the curve family \ref{def:local-diff}, we have 
        the induced vector field \(X^\phi\) given by
        \[
        X^\phi_q(f) := \left.\frac{d}{dt}(f(\phi_t(q)))\right|_{t=0}
        \]
    \end{definition}
    \begin{theorem}
        The curve family \(t \mapsto \phi_t(q)\) is the integral curve of 
        the induced vector field \ref{def:induced-vector-field} \(X^\phi_q\).
    \end{theorem}
    \begin{proof}
        \[
        \begin{aligned}
            X^\phi_{\phi_s(q)} &= \left.\frac{d}{dt}
            (f\circ\phi_t\circ\phi_s(q))\right|_{t=0} \\
            &= \left.\frac{d}{dt}(f\circ\phi_{t+s}(q))\right|_{t=0}.
        \end{aligned}
        \]
        Let \(u = t+s\). Then 
        \[
        \begin{aligned}
            X^\phi_{\phi_s(q)} &= 
            \left.\frac{d}{du}(f\circ\phi_{u}(q))\right|_{u=s}. \\
            &= \phi_{q*}\left(\frac{d}{dt}\right)_sf.
        \end{aligned}
        \]
    \end{proof}
    \subsubsection{Local Flows}
    \begin{definition}[Local Flow]\label{def:local-flow}
        Let \(X\) be a vector field on open \(U \subseteq \manifold[M]\), and 
        \(p \in U\). A local flow at \(p\) is a local one-parameter family 
        of local diffeomorphisms \ref{def:local-diff} defined on some open 
        \(V \subseteq U\) s.t. \(p \in V\) and the induced vector field 
        \ref{def:induced-vector-field} is \(X\).
    \end{definition}
    \begin{remark}
        Local flows always exist and are unique. In contrast, global flows 
        (which means \(t \in \mathbb{R}\) instead of a restricted interval) 
        may not exist.
        
    \end{remark}
    \subsubsection{Lie Derivative}
    \begin{theorem}[Interpretation of Lie Bracket]\label{thm:lie-bracket-flow}
        If \(X, Y\) are two vector fields on \(\manifold[M]\), and define the 
        following quantity, which can be interpreted as the change of \(Y\) 
        when following the integral curves of \(X\), as
        \[
        \left.\frac{d}{dt}(\phi_{-t*}^X(Y))\right|_{t=0}:=
        \lim_{\epsilon \to 0} \frac{\phi_{-\epsilon*}^X(Y_{\phi_\epsilon^X(p)})
        -Y_p}{\epsilon}.
        \]
        Then, 
        \[
        \left.\frac{d}{dt}(\phi_{-t*}^X(Y))\right|_{t=0}=[X, Y].
        \]
    \end{theorem}
    \section{Cotangent Spaces}
    \begin{definition}[Cotangent Spaces]\label{def:cotangent-space}
        The cotangent space \(T_p^*\manifold[M]\) at \(p \in \manifold[M]\) is 
        the set of all linear functions \(f : T_p\manifold[M] \to \mathbb{R}\). 
        \par
        Its member is called a cotangent vector. \par
        \(\dim T_p^*\manifold[M]=\dim T_p\manifold[M]\).
    \end{definition}
    \begin{definition}[One-Form]\label{def:one-form}
        A one-form on \(\manifold[M]\) is a smooth assignment of cotangent 
        vectors \(\omega : p \mapsto \omega_p\). \par
        It may be understood as a covector field.
    \end{definition}
    \begin{definition}[Basis Cotangent Vectors]\label{def:cotangent-basis}
        The basis cotangent vectors is chosen to be the dual basis of the 
        basis tangent vectors \ref{def:basis-derivations}, 
        \[
        (dx^\mu)_p(\tvec{\nu}{p})=\delta\indices{^\mu_\nu}.
        \]
    \end{definition}
    \begin{theorem}[Coordinate Expression of Cotangent Vectors]\label{thm:cotangent-coordinates}
        Any \(f \in T_p^*\manifold[M]\) can be expanded as
        \[
        f = f_\mu (dx^\mu)_p.
        \]
        Any one-form \(\omega\) can be expressed as 
        \[
        \omega=\omega_\mu dx^\mu.
        \]
    \end{theorem}
\end{document}