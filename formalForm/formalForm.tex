\documentclass[../main.tex]{subfiles}

\begin{document}
    \section{Formal Differential Form}
    \begin{remark}
        In this section, we follow a local, coordinate approach. We focus on the 
        requirements that makes form a form. We will postpone the realization of 
        differential forms.
    \end{remark}
    \subsection{Euclidean Spaces}
    \begin{definition}[Formal Differential Form on \(\mathbb{R}^m\)]\label{def:formal-form}
        A formal differential \(k\)-form on \(\mathbb{R}^m\) is composed of \(m^k\) 
        functions, arranged in the form
        \[
        \omega = \sum_{1 \leq i_1, \dots, i_k \leq m}^{} 
        \omega \indices{_{i_1}_\dots_{i_k}} dx^{i_1} \wedge \dots \wedge dx^{i_k}.
        \]
        We require,
        \begin{enumerate}
            \item All the functions \(\omega \indices{_{i_1}_\dots_{i_k}}\)
            are \((-\infty, 0]\times\mathbb{R}^{m-1} \to (-\infty, 0]\times\mathbb{R}^{m-1}\) and \(C^\infty\).
            \item The operation \(+\) is commutative and associative, just like the 
            usual addition.
            \item The operation \(\wedge\) is associative and distributes over 
            \(+\), resembling the usual multiplication.
            \item \(\wedge\) also has anticommutativity, 
            \[
            \begin{cases}
            dx^i \wedge dx^j = -dx^j \wedge dx^i &\forall i \neq j\\
            dx^i \wedge dx^i = 0
            \end{cases}
            \]
        \end{enumerate}
    \end{definition}
    \begin{definition}[Equality of Formal Differential k-forms]\label{def:formal-form-equality}
        Let two formal \(k\)-forms \(\omega, \eta\) be 
        \[
        \begin{aligned}
            \omega &= \omega\indices{_{i_1}_\dots_{i_k}} 
                dx^{i_1} \wedge \dots \wedge dx^{i_k}\\
            \eta &= \eta\indices{_{i_1}_\dots_{i_k}} 
                dx^{i_1} \wedge \dots \wedge dx^{i_k}
        \end{aligned}
        \]
        We say they are "equal", denoted \(\omega \equiv \eta\), iff 
        \[
        \sum_{\sigma \in S_k}^{} \omega\indices{_{\sigma(i_1)}_\dots_{\sigma(i_k)}}
        =\sum_{\sigma \in S_k}^{} \eta\indices{_{\sigma(i_1)}_\dots_{\sigma(i_k)}}
        \;\forall 1 \leq i_1 < \dots < i_k \leq m.
        \]
    \end{definition}
    \begin{remark}
        \begin{enumerate}
            \item For a formal differential \(k\)-form \(\omega = dx^1\wedge dx^2\), 
            \(\omega_{12} = 1\), but \(\omega_{21} = 0\). The \(m^k\) components in 
            the definition just presents a general form, so that it includes the 
            scenario \(\omega' = dx^1\wedge dx^2 - dx^2 \wedge dx^1\). If you insist 
            on arranging indices even in the definition, then \(\omega'\) would not 
            satisfy the definition, which is wierd.
            \item For demonstration of equality, consider the following example in 
            \(\mathbb{R}^3\),
            \[
            \begin{aligned}
                \omega &= dx^1\wedge dx^2 + dx^2\wedge dx^3 \\
                \eta   &= -dx^2 \wedge dx^1 + dx^2 \wedge dx^3.
            \end{aligned}
            \]
            Choose \(i_1 = 1, i_2 = 2\), and 
            \[
            S_3 = \{\underbrace{\idd, (1\;2\;3), (1\;3\;2)}_{\text{even}}
            , \underbrace{(1\;2), (2\;3), (1\;3)}_{\text{odd}}\}.
            \]
            Then, in order, 
            \[
            \begin{aligned}
                &\sum_{\sigma \in S_n}^{} (\sgn \sigma)\omega \\
                =&\textcolor{red}{\omega_{12}}+\textcolor{blue}{\omega_{23}}
                +\textcolor{green}{\omega_{31}}-\textcolor{red}{\omega_{21}}
                -\textcolor{green}{\omega_{13}}-\textcolor{blue}{\omega_{32}} \\
                =&\textcolor{red}{1}+\textcolor{blue}{1}
                +\textcolor{green}{0}-\textcolor{red}{0}
                -\textcolor{green}{0}-\textcolor{blue}{0}.
            \end{aligned}
            \]
            Notice how all permutations of a given component (paired in color) 
            appears exactly once in this relation, and the sign is fixed 
            accordingly by \(\sgn \sigma\).
            \item Viewed this way, we see in the language of formal differential 
            forms, 
            \[
            \begin{aligned}
                \omega &= dx^1\wedge dx^2 \\
                \eta &= -dx^2\wedge dx^1 \\
                \nu &= dx^1\wedge dx^2 - dx^2\wedge dx^1,
            \end{aligned}
            \]
            \(\omega \equiv \eta\), since \(\omega_{12} - \omega_{21} = 1 = 
            \eta_{12} - \eta_{21}\). But \(\omega \not\equiv \nu\), since 
            \(\nu_{12} - \nu_{21} = 2\).
        \end{enumerate}
    \end{remark}
    \subsection{Operations}
    \begin{definition}[Exterior Product]\label{def:formal-exterior-product}
        Let two formal \(k\)-forms be 
        \(\omega = \omega_{i_1\dots i_k} dx^{i_1} \wedge \dots \wedge dx^{i_k}\), 
        \(\eta = \eta_{j_1\dots j_l} dx^{j_1} \wedge \dots \wedge dx^{j_l}\). Then 
        \[
        \omega\wedge \eta := \omega_{i_1\dots i_k}\eta_{j_1\dots j_l}
        dx^{i_1}\wedge\dots\wedge dx^{i_k} \wedge dx^{j_1} \wedge \dots \wedge 
        dx^{j_l}.
        \]
        Which is equivalent \cref{def:formal-form-equality} to, 
        \[
        \sum_{1 \leq s_1 < \dots < s_{k+l} \leq m}^{} \left(
            \sum_{S}^{} (\sgn\sigma)\omega_{i_1\dots i_k}\eta_{j_1\dots j_l}
        \right)dx^{s_1}\wedge \dots\wedge dx^{s_{k+l}}.
        \]
        Where \(S\) is all the combinations of \(S_1 = \{i_1, \dots, i_k, j_1, \dots, 
        j_l\}\) that is equal to \(S_2 = \{s_1, \dots, s_{k+l}\}\), and the order does 
        not matter. \(\sigma\) is the function \(S_1 \to S_2\).
    \end{definition}
    \begin{remark}
        Example. Let \(\dim \manifold[M]=6\), and 
        \[
        \begin{aligned}
            \omega &= \omega_{12}dx^1\wedge dx^2 + \omega_{21} dx^2 \wedge dx^1 \\
            \eta   &= \eta_{456} dx^4\wedge dx^5\wedge dx^6.
        \end{aligned}
        \]
        Then 
        \[
        \omega\wedge \eta = (\omega_{12}\eta_{456} 
        - \omega_{21}\eta_{456}) dx^{1, 2, 4, 5, 6}
        \]
    \end{remark}
    \begin{definition}[Pullback]\label{def:formal-pullback}
        Let \(f : U \subseteq \mathbb{R}^m \to V \subseteq (-\infty, 0] \times 
        \mathbb{R}^{n-1}\) is \(C^\infty\). Choose local coordinates on 
        \(U\) to be \(x = (x^1, \dots, x^m)\), on \(V\) to be \(y = (y^1, \dots, 
        y^n)\). Define \(f^\nu := y^\nu \circ f\).\\
        Let \(\omega = \omega_{j_1\dots j_k} dy^{j_1}\wedge \dots \wedge dy^{j_k}\) 
        be a formal \(k\)-form on \(U\). Then 
        \[
        f^*\omega := (\omega_{j_1\dots j_k}\circ f)
        \frac{\partial f^{j_1}}{\partial x^{i_1}}\dots
        \frac{\partial f^{j_k}}{\partial x^{i_k}}dx^{i_1}\wedge \dots\wedge dx^{i_k}.
        \]
    \end{definition}
    \begin{remark}
        \begin{enumerate}
            \item The motivation is just somehow \(\omega\circ f\). Chain the
            component function with \(f\), and express the (resulting) coordinates 
            as \(dy^{\nu} = \frac{\partial f^{\nu}}{\partial x^{\mu}}dx^{\mu}\).
            \item It is possible that \(U \subseteq V\). Pulling back onto a subset 
            is essentially a "limitation" on \(\omega\).
            \item It is also possible that \(U \subseteq \partial V\), i.e. \(U\) is 
            the boundary of \(V\). \(f : p \mapsto (0, p)\) is just the 
            immersion map in that case.
        \end{enumerate}
    \end{remark}
    \begin{definition}[Exterior Differentiation]\label{def:formal-exterior-diff}
        Let a formal \(k\)-form be
        \(\omega = \omega_{i_1\dots i_k} dx^{i_1} \wedge \dots \wedge dx^{i_k}\). 
        Then 
        \[
        d\omega := \left(\frac{\partial \omega_{i_1\dots i_k}}{\partial x^{i_0}}
        dx^{i_0}\right) 
        \wedge dx^{i_1} \wedge \dots \wedge dx^{i_k}.
        \]
    \end{definition}
    \begin{remark}
        Let \(U \subseteq \mathbb{R}^3\).
        \begin{enumerate}
            \item Consider a formal 0-form, i.e. \(f \in C^\infty\). Then 
            \[
            df = \frac{\partial f}{\partial x}dx+\frac{\partial f}{\partial y}dy 
            + \frac{\partial f}{\partial z}dz \rightarrow \nabla f.
            \]
            \item Consider a formal 1-form \(\omega = Pdx+Qdy+Rdz\). Then 
            \[
            \begin{aligned}
                d\omega &= \left(\frac{\partial P}{\partial y}dy+
                \frac{\partial P}{\partial z}dz\right)\wedge dx + 
                \left(\frac{\partial Q}{\partial x}dx+
                \frac{\partial Q}{\partial z}dz\right)\wedge dy + 
                \left(\frac{\partial R}{\partial x}dx+
                \frac{\partial R}{\partial y}dy\right)\wedge dz \\
                &=\left(\frac{\partial R}{\partial y}-
                \frac{\partial Q}{\partial z}\right)dy\wedge dz + 
                \left(\frac{\partial R}{\partial x}-
                \frac{\partial P}{\partial z}\right)dx\wedge dz + 
                \left(\frac{\partial Q}{\partial x}-
                \frac{\partial P}{\partial y}\right)dx\wedge dy \\
                &\rightarrow \curl \omega.
            \end{aligned}
            \]
            \item Consider a formal 2-form \(\eta = A dy\wedge dz + B dz \wedge dx + 
            C dx \wedge dy\). Then 
            \[
            \begin{aligned}
                d\eta &= \frac{\partial A}{\partial x}dx\wedge dy\wedge dz + 
                \frac{\partial B}{\partial y}dy\wedge dz\wedge dx + 
                \frac{\partial C}{\partial z}dz\wedge dx\wedge dy \\
                &= \left(\frac{\partial A}{\partial x} + 
                \frac{\partial B}{\partial y} + 
                \frac{\partial C}{\partial z} \right)dx\wedge dy\wedge dz \\
                &\rightarrow \divv \eta.
            \end{aligned}
            \]
            \item In usual vector calculus terms, we say \(\nabla\) produces a 
            vector, \(\divv\) produces a scalar, and \(\curl\) produces a vector.
            This \(r\)-form to \((m-r)\)-form correspondance is provided by 
            the Hodge star operation.
        \end{enumerate}
    \end{remark}
    \subsection{Differential Form as Equivalence Classes}
    \subsubsection{Operations are Well-Defined}
    \begin{theorem}
        If \(\omega\equiv \omega'\), \(\eta \equiv \eta'\) are formal \(k\)-forms on 
        \(\mathbb{R}^m\), 
        then 
        \[
        \omega \wedge \eta \equiv \omega' \wedge \eta'.
        \]
    \end{theorem}
    \begin{theorem}
        If \(\omega \equiv \omega'\) are formal \(k\)-forms on \(\mathbb{R}^m\), 
        and \(f \in C^\infty(\mathbb{R}^m)\), then 
        \[
        f^*\omega \equiv f^*\omega'.
        \]
    \end{theorem}
    \begin{theorem}
        If \(\omega \equiv \omega'\) are formal \(k\)-forms on \(\mathbb{R}^m\), 
        then 
        \[
        d\omega\equiv d\omega'.
        \]
    \end{theorem}
    \subsubsection{Equivalence Classes}
    \begin{definition}[Euclidean Differential k-form]\label{def:euc-form}
        Denote the set of all formal \(k\)-forms on \(\mathbb{R}^m\) be 
        \(A^k(\mathbb{R}^m)\). Then the set of all differential \(k\)-forms on 
        \(\mathbb{R}^m\) is defined to be 
        \[
        \eucForm{k}{m} := A^k(\mathbb{R}^m) / {\equiv}.
        \]
    \end{definition}
    \subsection{Properties of Operations}
    \begin{theorem}[Properties of Exterior Product]\label{thm:exterior-product-properties}
        Let \(\omega \in \eucForm{k}{m}\), \(\eta \in 
        \eucForm{l}{m}\). Then
        \[
        \begin{aligned}
            (\omega_1+\omega_2)\wedge \eta &= \omega_1\wedge \eta + \omega_2 \wedge 
            \eta \\
            \omega \wedge \eta &= (-1)^{kl}\eta \wedge \omega. \\
        \end{aligned}
        \]
    \end{theorem}
    \begin{theorem}[Properties of Pullback]\label{thm:pullback-properties}
        Let \(f \in C^\infty\), \(\omega \in \eucForm{k}{m}\). Then 
        \[
        \begin{aligned}
            f^*(\omega_1+\omega_2) &= f^*\omega_1+f^*\omega_2 \\
            f^*(\omega_1\wedge \omega_2) &= (f^*\omega_1) \wedge (f^*\omega_2) \\
            g^*(f^*\eta) &= (f\circ g)^*\eta.
        \end{aligned}
        \]
    \end{theorem}
    \begin{theorem}[Properties of Exterior Differentiation]\label{thm:diff-properties}
        Let \(f \in C^\infty\), \(\omega \in \eucForm{k}{m}\). Then
        \[
        \begin{aligned}
            f^*(d\omega) &= d(f^*\omega) \\
            d(\omega\wedge \eta) &= (d\omega)\wedge \eta + (-1)^k\omega\wedge (d\eta) \\
            d(d\omega) &= 0.
        \end{aligned}
        \]
    \end{theorem}
    \newpage
    \subsection{Differential Forms on Manifolds}
    \subsubsection{Requirements of Manifold Forms}
    \begin{definition}[Differential Form on a Manifold]\label{def:manifold-form}
        A \(C^\infty\) differential \(k\)-form \(\omega\) on manifold \(\manifold[M]\), 
        \(\omega \in \form{k}{M}\), consists of a family of 
        differential \(k\)-forms \(\omega_\phi \in \Lambda^k(\phi(U_\phi))\), 
        \(\phi \in \Phi\), \(\phi : U_\phi \to V_\phi = \phi(U_\phi) \subseteq 
        (-\infty, 0] \times \mathbb{R}^{m-1}\), with an additional requirement that 
        \[
        \left.\omega_{\phi'}\right|_{\phi'(U_\phi \cap U_{\phi'})}
        = \left(\phi\circ{\phi'} ^{-1}\right)^*\left(\left.\omega_\phi
        \right|_{\phi(U_\phi \cap U_{\phi'})}\right) ,\; \forall 
        \phi, \phi' \in \Phi.
        \]
        \(\omega_\phi\) is called the local expression of \(\omega\) on \(U_\phi\) 
        via \(\phi\).
    \end{definition}
    \begin{remark}\phantom{hehe}\\
        \begin{center}
            \centering
            \def\svgwidth{0.6\textwidth}
            \input{./figs/manifold_form.pdf_tex}
        \end{center}
        The motivation is that, if two differential forms describe the same set, 
        they should "agree" on that portion of manifold. The "agreement" is done by 
        pullback using the overlap function.
    \end{remark}
    \newpage
    \begin{theorem}[Covariancy of Differential Forms]\label{thm:form-covariancy}
        Choose two overlapping charts \(x^1, \dots, x^m\) and \(y^1, \dots, y^m\) 
        and define their Jacobian matrix \cref{def:jacobian-matrix} 
        \(J\indices{^\nu_\mu}:=\left.\frac{\partial y^\nu}{\partial x^\mu}\right|
        _{\phi(p)}\), combining \cref{def:formal-pullback} and 
        \cref{def:manifold-form}, we have 
        \[
        \begin{cases}
            dy^\nu = J\indices{^\nu_\mu}dx^\mu & (\text{covariant}) \\
            \omega_{\nu'} = \omega_{\mu}(J^{-1})\indices{^\mu_\nu} & 
            (\text{contravariant})
        \end{cases}
        \]
    \end{theorem}
    \subsubsection{Operations on Manifold Forms}
    \begin{definition}[Addition]\label{def:mani-form-add}
        Given two forms \(\omega, \eta \in \form{k}{M}\), 
        \(m=\dim \manifold[M]\), define their sum to be \(\omega+\eta\), whose 
        chart components are given by 
        \[
        (\omega+\eta)_\phi := \omega_\phi + \eta_\phi.
        \]
        They satisfy \cref{def:manifold-form} thanks to the linearity of pullback.
    \end{definition}
    \begin{definition}[Exterior Product]\label{def:mani-form-exterior-product}
        Given two forms \(\omega, \eta \in \form{k}{M}\), 
        \(m=\dim \manifold[M]\), define their exterior product to be 
        \(\omega\wedge\eta\), whose chart components are given by 
        \[
        (\omega\wedge\eta)_\phi := \omega_\phi\wedge \eta_\phi.
        \]
        They satisfy \cref{def:manifold-form} because pullback commutes with 
        exterior product.
    \end{definition}
    \begin{definition}[Exterior Differentiation]\label{def:mani-form-diff}
        Given a form \(\omega \in \form{k}{M}\), 
        \(m=\dim \manifold[M]\), define its exterior derivative to be
        \(d\omega\), whose chart components are given by 
        \[
        (d\omega)_\phi := d(\omega_\phi).
        \]
        They satisfy \cref{def:manifold-form} because pullback commutes with 
        exterior product.
    \end{definition}
    \begin{definition}[Pullback]\label{def:mani-form-pullback}
        Given a form \(\omega \in \form{k}{N}\) and a function 
        \(f : \manifold[M] \to \manifold[N]\), \(m=\dim \manifold[M], 
        n=\dim \manifold[N]\). Choose coordinate functions 
        \(\phi_i : U_i \to \phi_i(U_i)\) on \(\manifold[M]\), and 
        \(\psi_j : V_j \to \psi_j(V_j)\) on \(\manifold[N]\). \\
        To define \((f^*\omega)_{\phi_1}\), choose any 
        \(V_1, \dots, V_s\) s.t. \(U_1 \subseteq \bigcup_{j=1}^{s} \tilde{f}(V_j)\).
        Then 
        \[
        (f^*\omega)_{\phi_i} := \sum_{j}^{} 
        (\phi_i\circ f\circ \psi_j ^{-1})^*
        \left(\left.\omega_{\psi_j}\right|_{\phi(f(U_i) \cap V_j)}\right)
        \]
        They satisfy \cref{def:manifold-form}.
    \end{definition}
\end{document}