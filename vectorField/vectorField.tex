\documentclass[../main.tex]{subfiles}

\begin{document}
    \section{Vector Fields and the Tangent Bundle}
    \subsection{Vector Fields}
    \begin{definition}[Vector Fields]\label{def:vector-fields}
        A vector field \(X\) on \(V \subseteq\manifold[M]\) is any rule of assigning 
        a tangent vector \(X(p) = X_p \in T_p\manifold[M]\) for all
        \(p \in V\).
    \end{definition}
    \begin{definition}[Components of a Vector Field]\label{def:vector-field-component}
        Let \(X\) be a vector field on \(V \subseteq \manifold[M]\), 
        \(\dim \manifold[M] = m\). For any chart \(\phi : U \to \phi(U) \in \Phi\) 
        with induced coordinates \(x^1, \dots, x^m\) and any \(p \in V \cap U\), 
        the decomposition \(X(p) = X^j(p)\tvec{j}{p}\) is unique, and therefore 
        we write 
        \[
        X = X^j \tfld{j}
        \]
        on \(V \cap U\), and \(X^j : V \cap U \to \mathbb{R}\) are called the 
        components of \(X\) on \(V \cap U\).
    \end{definition}
    \begin{definition}[Smoothness of Vector Field]\label{def:vec-field-smoothness}
        A vector field \(X\) on \(V \subseteq \manifold[M]\) is \(C^k\) near 
        \(p \in V\) iff \(\exists~\phi \in \Phi\) s.t. \(p \in U_\phi\) and all 
        the components \(X^j\) induced by \(\phi\) are \(C^k\). That is, 
        \[
        X^j\circ \phi ^{-1} \text{ are } C^k \text{ on } \phi(V \cap U).
        \]
    \end{definition}
    \subsection{Tangent Bundle}
    \begin{definition}[Tangent Bundle]\label{def:tangent-bundle}
        The tangent bundle \(T\manifold[M]\) is 
        \[
        T\manifold[M] := \bigcup_{p \in \manifold[M]}^{} T_p\manifold[M].
        \]
    \end{definition}
    \begin{remark}
        Why not \(T\manifold[M] := \Set{T_p\manifold[M]|p \in \manifold[M]}\)?
    \end{remark}
    \begin{definition}[Canonical Projection]\label{def:canonical-projection}
        The canonical projection is the map 
        \[
        \begin{aligned}
            \pi : T\manifold[M] &\to \manifold[M] \\
            T_p\manifold[M] &\mapsto p.
        \end{aligned}
        \]
    \end{definition}
    \begin{definition}[Alternative Definition of Vector Fields]\label{def:vector-fields-alt}
        A tangent vector field \(X\) on \(V \subseteq \manifold[M]\) is a map 
        \(X : V \to T\manifold[M]\) s.t. \((\pi \circ X)(p) = p\) for all \(p \in V\).
    \end{definition}
    \begin{remark}
        Let's see why vector fields are often called a "cross-section" of a tangent 
        bundle.
        \begin{center}
            \centering
            \def\svgwidth{0.8\textwidth}
            \input{./figs/bundle.pdf_tex}
        \end{center}
    \end{remark}
    \begin{definition}[Lie Derivative]\label{def:lie-derivative}
        The Lie-derivative of function \(f\) with respect to vector field 
        \(X\) is defined as 
        \[
        \lder{X}f := Xf, 
        \]
        and at a specific point \(p \in \manifold[M]\), 
        \[
        \lder{X}f(p) := Xf(p) := X_pf.
        \]
    \end{definition}
    \begin{theorem}[Properties of Lie Derivative]\label{thm:lie-derivative-properties}
        The Lie derivative has the following properties,
        \begin{enumerate}
            \item \(X(rf+g) = rXf+Xg\)
            \item \(X(fg) = fXg+gXf\).
        \end{enumerate}
    \end{theorem}
    \begin{proof}
        We know 
        \[
        (Xf)(p) = X_pf = X_px^\mu\tvec{\mu}{p}f = (Xx^\mu)(p)\tvec{\mu}{p}f.
        \]
    \end{proof}
    \begin{remark}
        \(\tfld{\mu}\) is a vector field that assigns each point \(p \in 
        \manifold[M]\) with the vector \(\tvec{\mu}{p} \in T_p\manifold[M]\).
        
    \end{remark}
    \begin{theorem}[Contravariancy of Vector Fields]\label{thm:}
        Given two coordinate charts \((U, \phi)\) and \((U', \phi')\) s.t. 
        \(U \cap U' = S \neq \varnothing\). On \(S\), 
        \[
        X^{\nu'} = \sum_{\mu=1}^{m} X^\mu 
        \frac{\partial x'^{\nu}}{\partial x^\mu}.
        \]
        Analogous to \cref{thm:contravariant-vector}.
    \end{theorem}
    \subsection{Lie Bracket}
    \begin{definition}[Composition of Vector Fields]\label{def:vector-field-composition}
        We can view \(X : C^\infty(\manifold[M]) \to C^\infty(\manifold[M])\), 
        and so does \(Y\). Therefore, we define 
        \[
        (X \circ Y)(f) := X(Yf).
        \]
    \end{definition}
    \begin{definition}[Lie Bracket (Commutator)]\label{def:lie-bracket}
        We define the Lie Bracket of two vector fields \(X, Y\) to be 
        \[
        [X, Y] := X\circ Y - Y\circ X.
        \]
        In particular, 
        \[
        [X, Y](f) = \lder{X}(\lder{Y}f) - \lder{Y}(\lder{X}f)
        \]
    \end{definition}
    \begin{remark}
        Lie Bracket \cref{def:lie-bracket} is a vector field, while the 
        expression \(X\circ Y\) is not, because it contains second differential 
        terms. See the following proof.
        
    \end{remark}
    \begin{theorem}[Lie Bracket Components]\label{thm:lie-bracket-components}
        \[
        [X, Y]^\mu = (X^\nu\tfld{\nu}Y^\mu - Y^\nu\tfld{\nu}X^\mu).
        \]
    \end{theorem}
    \begin{proof}
        Given \(X=X^\mu\tfld{\mu}, Y = Y^\nu\tfld{\nu}\), we try to write 
        the component of \(X\circ Y\).
        \[
        X\circ Y(f) = X^\mu \tfld{\mu}\left(Y^\nu \tfld{\nu}f\right).
        \]
        However, notice that 
        \[
        \begin{aligned}
            &Y^\nu := Yx^\nu \in C^\infty(\manifold[M]); \\
            &\tfld{\nu} : C^\infty(\manifold[M]) \to C^\infty(\manifold[M]), \\
            &\implies\tfld{\nu}f \in C^\infty(\manifold[M]).
        \end{aligned}
        \]
        So we need to use the Leibniz property of \(\tfld{\mu}\) 
        \cref{def:derivation} in order to evaluate the second term. Doing this 
        for \(X\circ Y(f)\) and \(Y\circ X(f)\), we have
        \[
        \begin{aligned}
            X\circ Y(f) &= X^\mu \left((\tfld{\mu}Y^\nu)(\tfld{\nu}f)+
            Y^\nu \tfld{\mu}\tfld{\nu}f\right). \\
            Y\circ X(f) &= Y^\nu \left((\tfld{\nu}X^\mu)(\tfld{\mu}f)+
            X^\mu \tfld{\nu}\tfld{\mu}f\right).
        \end{aligned}
        \]
        So if \(\tfld{\mu}\tfld{\nu}f=\tfld{\nu}\tfld{\mu}f\), then by 
        subtracting, we can cancel the second order terms, and we are done. We 
        prove so now.
        \[
        \begin{aligned}
            (\tfld{\mu}\tfld{\nu}f)(p)&=\frac{\partial }{\partial u^\mu}
            \left.\left((\tfld{\nu}f)\circ\phi ^{-1}\right)\right|_{\phi(p)} \\
            &=\frac{\partial }{\partial u^\mu}
            \left.\left(\tvec{\nu}{\phi ^{-1}(u)}f\right)\right|_{\phi(p)} \\
            &=\frac{\partial }{\partial u^\mu}
            \left.\left(\left.
                \frac{\partial }{\partial u^\nu}(f\circ\phi ^{-1})
            \right|_u\right)\right|_{\phi(p)} \\
            &=\frac{\partial }{\partial u^\nu}
            \left.\left(\left.
                \frac{\partial }{\partial u^\mu}(f\circ\phi ^{-1})
            \right|_u\right)\right|_{\phi(p)} \\
            &=(\tfld{\nu}\tfld{\mu}f)(p).
        \end{aligned}
        \]
    \end{proof}
    \begin{theorem}[Properties of Lie Brackets]\label{thm:lie-bracket-properties}
        \phantom{something}
        \begin{enumerate}
            \item \([X,Y] = -[Y, X]\) (antisymmetry)
            \item \(\sum_{\text{cyc}}^{} [X, [Y, Z]]=0\). (Jacobi Identity)
        \end{enumerate}
    \end{theorem}
    \subsection{Integral Curves and Flows}
    \begin{definition}[Intergral Curve]\label{def:intergral-curve}
        Let \(X\) be a vector field on \(\manifold[M]\), \(p \in \manifold[M]\). 
        Then an integral curve of \(X\) through \(p\) is a curve \(\sigma : 
        (-\epsilon, \epsilon) \to \manifold[M]\) s.t. 
        \[
        \begin{aligned}
            \sigma(0) &= p, \\
            \sigma_*\left(\frac{d}{dt}\right)_t &= X_{\sigma(t)}.
        \end{aligned}
        \]
    \end{definition}
    \begin{remark}
        Qualitatively, using \cref{thm:curve-pushforward}, this pushforward is 
        just \([\sigma] \in T_{\sigma(t)}\manifold[M]\). Therefore, the second 
        condition is saying in some sense that the curve is tangent to the 
        vector field on the manifold. For quantitative description, see below.
        
    \end{remark}
    \begin{definition}[Differential Equations of Integral Curve]\label{def:de-int-curve}
        The components \(X^\mu\) of \(X\) determine the integral curve \(\sigma\) 
        by the following ODE with boundary conditions, 
        \[
        \begin{aligned}
            X^\mu(\sigma(t)) &= \frac{d}{dt}x^\mu(\sigma(t)) \\
            x^\mu(\sigma(0)) &= x^\mu(p), \mu = 1, 2, \dots, m.
        \end{aligned}
        \]
    \end{definition}
    \newpage
    \subsubsection{One-parameter Family of Diffeomorphisms}
    \begin{definition}[Local 1D Family of Local Diffeomorphisms]\label{def:local-diff}
        A local, 1D family of local diffeomorphisms at \(p \in \manifold[M]\) 
        is made up of (1) an open neighborhood \(U\) of \(p\), (2) \(\epsilon 
        > 0\) (3) a family of diffeomorphisms \(\Set{\phi_t|\left|t\right|<\epsilon}\), 
        \(\phi_t : U \to \manifold[M]\) s.t. 
        \begin{enumerate}
            \item Every \(\phi_t\) is a smooth function in \(t\) and \(q\).
            \item \(\forall t, s \in \mathbb{R}\) and \(|t|, |s|, |t+s| < \
            \epsilon\), and \(\forall q \in U\) s.t. \(\phi_t(q), \phi_s(q), 
            \phi_{t+s}(q) \in U\), we have 
            \[
            \phi_s(\phi_t(q)) = \phi_{s+t}(q).
            \]
            \item \(\phi_0(q)=q\).
        \end{enumerate}
    \end{definition}
    \begin{remark}
        The first "local" refers to the parameter \(t\), which is limited to 
        \((-\epsilon, \epsilon)\). The second "local" refers to the spatial 
        limitation to \(U\). 
        You can view \(\phi_t(q)\) as a curve that brings \(t \in (-\epsilon, 
        \epsilon)\) to \(\phi_t(q) \in \manifold[M]\).
        
    \end{remark}
    \begin{definition}[Induced Vector Field]\label{def:induced-vector-field}
        By taking tangents to the curve family \cref{def:local-diff}, we have 
        the induced vector field \(X^\phi\) given by
        \[
        X^\phi_q(f) := \left.\frac{d}{dt}(f(\phi_t(q)))\right|_{t=0}
        \]
    \end{definition}
    \begin{theorem}
        The curve family \(t \mapsto \phi_t(q)\) is the integral curve of 
        the induced vector field \cref{def:induced-vector-field} \(X^\phi_q\).
    \end{theorem}
    \begin{proof}
        \[
        \begin{aligned}
            X^\phi_{\phi_s(q)} &= \left.\frac{d}{dt}
            (f\circ\phi_t\circ\phi_s(q))\right|_{t=0} \\
            &= \left.\frac{d}{dt}(f\circ\phi_{t+s}(q))\right|_{t=0}.
        \end{aligned}
        \]
        Let \(u = t+s\). Then 
        \[
        \begin{aligned}
            X^\phi_{\phi_s(q)} &= 
            \left.\frac{d}{du}(f\circ\phi_{u}(q))\right|_{u=s}. \\
            &= \phi_{q*}\left(\frac{d}{dt}\right)_sf.
        \end{aligned}
        \]
    \end{proof}
    \subsubsection{Local Flows}
    \begin{definition}[Local Flow]\label{def:local-flow}
        Let \(X\) be a vector field on open \(U \subseteq \manifold[M]\), and 
        \(p \in U\). A local flow at \(p\) is a local one-parameter family 
        of local diffeomorphisms \cref{def:local-diff} defined on some open 
        \(V \subseteq U\) s.t. \(p \in V\) and the induced vector field 
        \cref{def:induced-vector-field} is \(X\).
    \end{definition}
    \begin{remark}
        Local flows always exist and are unique. In contrast, global flows 
        (which means \(t \in \mathbb{R}\) instead of a restricted interval) 
        may not exist.
        
    \end{remark}
    \subsubsection{Lie Derivative}
    \begin{theorem}[Interpretation of Lie Bracket]\label{thm:lie-bracket-flow}
        If \(X, Y\) are two vector fields on \(\manifold[M]\), and define the 
        following quantity, which can be interpreted as the change of \(Y\) 
        when following the integral curves of \(X\), as
        \[
        \left.\frac{d}{dt}(\phi_{-t*}^X(Y))\right|_{t=0}:=
        \lim_{\epsilon \to 0} \frac{\phi_{-\epsilon*}^X(Y_{\phi_\epsilon^X(p)})
        -Y_p}{\epsilon}.
        \]
        Then, 
        \[
        \left.\frac{d}{dt}(\phi_{-t*}^X(Y))\right|_{t=0}=[X, Y].
        \]
    \end{theorem}
\end{document}