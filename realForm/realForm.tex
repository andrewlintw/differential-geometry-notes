\documentclass[../main.tex]{subfiles}

\begin{document}
    \section{Realization of Differential Forms}
    \subsection{Cotangent Spaces}
    \begin{definition}[Cotangent Spaces]\label{def:cotangent-space}
        The cotangent space \(T_p^*\manifold[M]\) at \(p \in \manifold[M]\) is 
        the set of all linear functions \(f : T_p\manifold[M] \to \mathbb{R}\). 
        \newline
        Its member is called a cotangent vector. \newline
        \(\dim T_p^*\manifold[M]=\dim T_p\manifold[M]\).
    \end{definition}
    \begin{definition}[One-Form]\label{def:one-form}
        A one-form on \(\manifold[M]\) is a smooth assignment of cotangent 
        vectors \(\omega : p \mapsto \omega_p\). \newline
        It may be understood as a covector field.
    \end{definition}
    \begin{definition}[Basis Cotangent Vectors]\label{def:cotangent-basis}
        The basis cotangent vectors is chosen to be the dual basis of the 
        basis tangent vectors \cref{def:basis-alt}, 
        \[
        (dx^\mu)_p(\tvec{\nu}{p})=\delta\indices{^\mu_\nu}.
        \]
    \end{definition}
    \begin{theorem}[Coordinate Expression of Cotangent Vectors]\label{thm:cotangent-coordinates}
        Any \(f \in T_p^*\manifold[M]\) can be expanded as
        \[
        f = f_\mu (dx^\mu)_p.
        \]
        Any one-form \(\omega\) can be expressed as 
        \[
        \omega=\omega_\mu dx^\mu.
        \]
    \end{theorem}
    \newpage
    \begin{theorem}[Pullback as Dual of Pushforward]\label{thm:pullback-dual-pushforward}
        Given two manifolds \(\manifold[M]\) with dimension \(m\) and \(\manifold[N]\) 
        with dimension \(n\) and a \(C^\infty\) function \(f : \manifold[M] \to 
        \manifold[N]\), the pullback of a one-form is the dual of pushforward. 
        That is, 
        \[
        (h^*\omega)_p(v) := \omega_p(h_*v).
        \]
    \end{theorem}
    % \begin{definition}[Pullback]\label{def:pullback}
    %     Given a function and its pushforward, 
    %     we define pullback to be the dual of pushforward, i.e., 
    %     \begin{align*}
    %         h   &: & &\manifold[M]    & &\to & &\manifold[N], \\
    %         h_* &: & T_p&\manifold[M] & &\to & T_{h(p)}&\manifold[N], \\
    %         h^* &: & T_{h(p)}^*&\manifold[N] & &\to & T_p^*&\manifold[M], 
    %     \end{align*}
    %     s.t. given \(f \in T_{h(p)}^*\manifold[N]\) and \(v \in T_p\manifold[M]\), 
    %     \[
    %     (h^*f)(v) := f(h_*v).
    %     \]
    % \end{definition}
    % \begin{remark}
    %     Note especially on the direction of original function and its induced 
    %     pullback. This is crucial to the covariancy of one-forms.
    % \end{remark}
    % \begin{theorem}
    %     Given \(\omega\) a one-form on \(\manifold[N]\), and a function 
    %     \(h : \manifold[M] \to \manifold[N]\), the pullback \(h^*\omega\) is 
    %     defined as 
    %     \[
    %     (h^*\omega)(v)_p = \omega(h_*v)_{h(p)}.
    %     \]
    % \end{theorem}
    % \begin{theorem}[Associativity of Pullbacks]\label{thm:pullback-associativity}
    %     Analogous to \cref{thm:pushforward-associativity}, given 
    %     manifolds \(\manifold[M], \manifold[N], \manifold[P]\) and 
    %     \(h : \manifold[M] \to \manifold[N]\), \(k : \manifold[N] \to 
    %     \manifold[P]\), then 
    %     \[
    %     (k\circ h)^*=k^*\circ h^*.
    %     \]
    % \end{theorem}
    % \newpage
    % \subsubsection{Jacobian}
    % \begin{theorem}[Local Representative of Pullback]\label{thm:pullback-local-representative}
    %     Let \(\dim \manifold[M]=m, \dim \manifold[N]=n\), \(h : \manifold[M] 
    %     \to \manifold[N]\), \(\{x^1, \dots, x^m\}\) be the local coordinates of 
    %     \(\manifold[M]\) around \(p\), and \(\{y^1, \dots, y^n\}\) be the local 
    %     coordinates of \(\manifold[N]\) around \(h(p)\). Then 
    %     \[
    %     h^*\omega = \sum_{\mu=1}^{m} \sum_{\nu=1}^{n} \omega_\nu
    %     \left.\frac{\partial h^\nu}{\partial x^\mu}\right|_{p}(dx^\mu)_p,
    %     \]
    %     where \(J\indices{^\nu_\mu}:=\left.\frac{\partial h^\nu}{\partial x^\mu}\right|_{p}
    %     :=\tvec{\mu}{p}(y^\nu\circ h)\) is the Jacobian matrix.
    % \end{theorem}
    % \begin{proof}
    %     We know by \cref{def:pullback}, 
    %     \[
    %     (h^*\omega)_\mu(p) = h^*\omega(\tfld{\mu})=\omega(h_*\tfld{\mu}).
    %     \]
    %     Expand it in local coordinates of \(\manifold[N]\), 
    %     \[
    %     (h^*\omega)_\mu(p) = \omega_\nu dy^\nu(h_*\tfld{\mu}).
    %     \]
    %     Via similar procedure in \cref{thm:pushforward-local}, we arrive at 
    %     \[
    %     (h^*\omega)_\mu(p) = \omega_\nu \frac{\partial h^\nu}{\partial x^\mu}.
    %     \]
    % \end{proof}
    % \newpage
    % \subsection{Transformation Properties}
    % \begin{theorem}[Covariancy and Contravariancy]\label{thm:covariant-contravariant}
    %     Given two coordinate charts \((U, \phi)\) and \((U', \phi')\) s.t. 
    %     \(U \cap U' = S \neq \varnothing\), then on \(S\), 
    %     \begin{align*}
    %         X^{\nu'} &= \sum_{\mu=1}^{m} \frac{\partial x'^{\nu}}{\partial x^\mu}X^\mu, \\
    %         \omega_{\nu'} &= \sum_{\mu=1}^{m} \omega_\mu\frac{\partial x^\mu}{\partial x'^\nu}.
    %     \end{align*}
    %     If Jacobian matrix is given, 
    %     \begin{align*}
    %         J\indices{^{\nu'}_\mu} &:=
    %         \left.\frac{\partial x'^\nu}{\partial x^\mu}\right|_{p}
    %         :=\tvec{\mu}{p}x'^\nu, \\
    %         (J ^{-1})\indices{^\mu_{\nu'}} &:=
    %         \left.\frac{\partial x^\mu}{\partial x'^\nu}\right|_{p}
    %         :=\tvec{\nu'}{p}x^\mu, 
    %     \end{align*}
    %     then, 
    %     \begin{align*}
    %         X^{\nu'} &= J\indices{^{\nu'}_\mu}X^\mu, & &\text{(contravariant)}\\
    %         \omega_{\nu'} &= \omega_\mu (J ^{-1})\indices{^\mu_{\nu'}}.  &
    %         &\text{(covariant)}
    %     \end{align*}
    % \end{theorem}
    % \begin{proof}
    %     The contravariant part is proved in \cref{thm:contravariant-vector}. 
    %     Now we turn to the covariant part. \par
    %     Let \(h = \idd : (U, \phi) \subseteq \manifold[M] \to (U', \phi') 
    %     \subseteq \manifold[M]\), consider its pullback.
    %     \[
    %     (\idd^*\omega)_p = \omega_{\nu'}\frac{\partial x'^\nu}{\partial x^\mu} 
    %     dx^\mu.
    %     \]
    %     Then, 
    %     \[
    %     \omega_\mu = \omega_{\nu'}\frac{\partial x'^\nu}{\partial x^\mu}.
    %     \]
    %     Inverting the matrix equation above, we get the desired result.
    % \end{proof}
    % \newpage
    \begin{definition}[n-Forms]\label{def:n-forms}
        An n-form is a tensor field of type \((0, n)\) that is totally 
        skew-symmetric (or alternating, or totally antisymmetric), i.e., 
        \[
        \omega(X_1, X_2, \dots, X_n) = (\sgn \sigma)
        \omega(X_{\sigma(1)}, \dots, X_{\sigma(n)}), ~\forall \sigma \in S_n.
        \]
        The set of all n-forms on \(\manifold[M]\) is denoted as 
        \(\form{n}{M}\). \\
        The set of all forms is \(\form{}{M}=\bigoplus_{n=0}^{\dim \manifold[M]}
        \form{n}{M}\). \\
        Conventionally, we classify functions as 0-forms.
    \end{definition}
    \subsection{The Exterior Product}
    \begin{definition}[Exterior Product]\label{def:exterior-product}
        Given \(\omega_1 \in \form{n_1}{M}, \omega_2 \in \form{n_2}{M}\), their 
        exterior product is a \((n_1+n_2)\)-form given by, 
        \[
        \omega_1 \wedge \omega_2
         := \frac{1}{n_1!n_2!}
        \sum_{\sigma \in S_{n_1+n_2}}^{} (\sgn\sigma)(\omega_1 \otimes \omega_2)_\sigma.
        \]
        Written explicitly, 
        \begin{multline*}
        (\omega_1 \wedge \omega_2)(X_{1}, \dots, X_{n_1+n_2}):= \\
        \frac{1}{n_1!n_2!}
        \sum_{\sigma \in S_{n_1+n_2}}^{} (\sgn\sigma)(\omega_1 \otimes \omega_2)
        (X_{\sigma(1)}, \dots, X_{\sigma(n_1+n_2)})
        \end{multline*}
    \end{definition}
    \begin{remark}
        I'll take the alternating property and associativity of the exterior 
        product for granted. For a detailed proof, see Hoffman.
    \end{remark}
    \newpage
    \begin{theorem}[Commutativity with Pullback]\label{thm:exterior-commute-pullback}
        Given \(h : \manifold[M] \to \manifold[N]\) and \(\alpha, \beta \in 
        \form{}{N}\), then 
        \[
        h^*(\alpha \wedge \beta) = (h^*\alpha) \wedge (h^*\beta).
        \]
    \end{theorem}
    \begin{remark}
        For a "generalized" pullback, we have, 
        \[
        (h^*(\alpha))(X_1, \dots, X_{n_1}) = \alpha(h_*X_1, \dots, h_*X_{n_1}).
        \]
    \end{remark}
    \begin{proof}
        \begin{align*}
            (h^*\alpha)\wedge&(h^*\beta)\\
            &=\frac{1}{n_1!n_2!} \sum_{\sigma \in S_{n_1+n_2}}^{} 
            (\sgn\sigma) \alpha \otimes \beta (h_*X_{\sigma(1)}, \dots, h_*
            X_{\sigma(n_1+n_2)}). \\
            &=\frac{1}{n_1!n_2!} \sum_{\sigma \in S_{n_1+n_2}}^{} 
            (\sgn\sigma) h^*\left(\alpha \otimes \beta (X_{\sigma(1)}, \dots, 
            X_{\sigma(n_1+n_2)})\right). \\
            &=h^*\left(\frac{1}{n_1!n_2!} \sum_{\sigma \in S_{n_1+n_2}}^{} 
            (\sgn\sigma) \alpha \otimes \beta (X_{\sigma(1)}, \dots, 
            X_{\sigma(n_1+n_2)})\right). \\
            &=h^*(\alpha \wedge \beta).
        \end{align*}
    \end{proof}
    \begin{theorem}[Skew-Symmetry]\label{thm:skew-symmetry-exterior-product}
        The exterior product makes \(\form{}{M}\) a graded algebra with 
        skew-symmetry given by 
        \[
        \omega_1 \wedge \omega_2 = (-1)^{n_1n_2}\omega_2 \wedge \omega_1.
        \]
    \end{theorem}
    \begin{proof}
        In the definition of exterior product, first fix \(\sigma = \sigma_0\) 
        to consider only one term. \par
        When we switch \(\omega_1\) and \(\omega_2\), we are essentially doing 
        \begin{align*}
            &(\omega_2 \otimes \omega_1)
            (X_{\sigma_0(1)}, \dots, X_{\sigma_0(n_2)}, \underbrace{X_{\sigma_0(n_2+1)}, 
            \dots, X_{\sigma_0(n_1+n_2)}}) \\
            =&(\omega_1 \otimes \omega_2)
            (\underbrace{X_{\sigma_0(n_2+1)}, \dots, X_{\sigma_0(n_1+n_2)}}, X_{\sigma_0(1)}, 
            \dots, X_{\sigma_0(n_2)}).
        \end{align*}
        Now, \par
        \begin{center}
            \begin{tikzpicture}
            \node (init) at (0, 0) 
            {$\underbrace{1, 2, \dots, n_2}, \underbrace{n_2+1}, \dots, n_1+n_2$};
            \node[below=10pt of init] (initToOne) {$\downarrow\ n_2$ times};
            \node[below=10pt of initToOne] (one)
            {$\underbrace{n_2+1}, \underbrace{1, 2, \dots, n_2}, \dots, n_1+n_2$};
            \node[below=10pt of one] (oneToFinal) {$\downarrow\ (n_1-1)n_2$ times};
            \node[below=10pt of oneToFinal] (final)
            {$\underbrace{n_2+1, \dots, n_1+n_2}, \underbrace{1, 2, \dots, n_2}$};
            \end{tikzpicture}
        \end{center}
        \par
        So \(n_1n_2\) transposes can achieve the desired effect. Therefore, 
        every term in the summation is multiplied by \((-1)^{n_1n_2}\), and 
        we get the desired result.
    \end{proof}
    \begin{theorem}[Dimension of n-Forms]\label{thm:n-form-dimension}
        Let \(\dim \manifold[M]=m\). If \(1 \leq n \leq m\), then \(\form{n}{M}
        = \binom{m}{n}\). If \(n > m\), then \(\form{n}{M}=0\).\\
        Moreover, a basis for \(\form{n}{M}_p\) is given by, 
        \[
        (dx^{\mu_1})_p \wedge (dx^{\mu_2})_p \wedge \dots \wedge 
        (dx^{\mu_n})_p, ~1 \leq \mu_1 \leq \dots \leq \mu_n \leq m.
        \]
    \end{theorem}
    \begin{remark}
        The proof is quite a pleasure to read (and to think of). Please see 
        Hoffman.
    \end{remark}
    \newpage
    \subsection{The Exterior Derivative}
    \begin{definition}[Exterior Derivative]\label{def:exterior-derivative}
        Let a \(k\)-form be
        \(\omega = \omega_{i_1\dots i_k} dx^{i_1} \wedge \dots \wedge dx^{i_k}\). 
        Then 
        \[
        d\omega := \left(\frac{\partial \omega_{i_1\dots i_k}}{\partial x^{i_0}}
        dx^{i_0}\right) 
        \wedge dx^{i_1} \wedge \dots \wedge dx^{i_k}.
        \]
        If \(\omega \in \form{\dim \manifold[M]}{M}\), we define \(d\omega=0\).
    \end{definition}
    % \begin{theorem}
    %     In particular for a 0-form \(f \in C^\infty(\manifold[M])\), 
    %     \[
    %     df(X) := \lder{X}f.
    %     \]
    %     In coordinates, 
    %     \[
    %     df = (\tfld{\mu}f)(dx^\mu).
    %     \]
    % \end{theorem}
    % \begin{theorem}
    %     In particular for a 1-form \(\omega\), 
    %     \[
    %     d\omega(X, Y) = \lder{X}(\omega(Y)) - \lder{Y}(\omega(X)) - \omega([X, Y]).
    %     \]
    % \end{theorem}
    % \begin{theorem}[Coordinate Expansion for Exterior Derivative]\label{thm:exterior-derivative-coordinates}
    %     In local coordinates, if \(\omega = \omega\indices
    %     {_{\mu_1}_{\mu_2}_\dots_{\mu_n}}dx^{\mu_1}\wedge dx^{\mu_2} \wedge \dots 
    %     \wedge dx^{\mu_n}\), then 
    %     \[
    %     d\omega = \tfld{\nu}\omega\indices{_{\mu_1}_{\mu_2}_\dots_{\mu_n}}
    %     dx^\nu\wedge dx^{\mu_1}\wedge dx^{\mu_2} \wedge \dots \wedge dx^{\mu_n}
    %     \]
    % \end{theorem}
    \begin{theorem}[Exterior Derivative and Product]\label{thm:exterior-leibniz}
        \[
        d(\omega_1 \wedge \omega_2) = d\omega_1 \wedge \omega_2 + 
        (-1)^{\deg \omega_1}\omega_1 \wedge d\omega_2.
        \]
    \end{theorem}
    \begin{theorem}[Exterior Derivative and Pullback]\label{thm:exterior-derivative-pullback}
        Given \(h : \manifold[M] \to \manifold[N]\), \(\omega\) an n-form on 
        \(\manifold[N]\), then 
        \[
        d(h^*\omega) = h^*(d\omega).
        \]
    \end{theorem}
    \begin{theorem}[Functional Linearity of Exterior Derivative]
        \label{thm:f-linear-exterior-derivative}
        Let \(\omega\) be a 1-form on \(\manifold[M]\). Then \(d\omega\) 
        satisfies,
        \[
        d\omega(fX, Y) = fd\omega(X, Y),~\forall f \in C^\infty(\manifold[M]),
        \]
        where \(fX\) is a vector field that gives \((fX)(p) = f(p)X_p\).
    \end{theorem}
    \begin{proof}
        By \cref{def:exterior-derivative}, 
        \[
        d\omega(fX, Y) = \lder{fX}(\omega(Y)) - \lder{Y}(\omega(fX)) - 
        \omega([fX, Y]).
        \]
        We break it down term by term. Firstly, 
        \[
        (\lder{fX}(\omega(Y)))(p) = f(p)X_p(\omega(Y)) = 
        f(p)(\lder{X}(\omega(Y)))(p).
        \]
        So
        \[
        \lder{fX}(\omega(Y)) = f\cdot\lder{X}(\omega(Y)).
        \]
        Secondly, we tackle \(\lder{Y}(\omega(fX))\). In particular, 
        \[
        \omega(fX)(p) = \omega_p(f(p)X_p) = f(p) \omega_p(X_p) = 
        f(p)(\omega(X))(p).
        \]
        Therefore, 
        \[
        \lder{Y}(\omega(fX)) = \lder{Y}(f\cdot\omega(X)) = 
        (\lder{Y}f)\omega(X) + f\cdot\lder{Y}(\omega(X)).
        \]
        Thirdly, 
        \[
        \omega([fX, Y]) = \omega((fX)\circ Y - Y \circ (fX)).
        \]
        In particular, 
        \[
        ((Y \circ (fX))(g))(p) = Y_p((fX)(g)) = Y_p(f\cdot Xg)
        = (Y_pf)((Xg)(p)) + f(p)\cdot Y_p(Xg).
        \]
        So, 
        \[
        Y \circ (fX) = (\lder{Y}f)X + f\cdot Y\circ X.
        \]
        Substituting back, 
        \[
        \begin{aligned}
            \omega([fX, Y]) &= \omega(f\cdot X\circ Y - (\lder{Y}f)X - f\cdot Y\circ X) \\
            &= \omega(f[X, Y] - (\lder{Y}f)X) \\
            &= f\omega([X, Y]) - (\lder{Y}f)\omega(X).
        \end{aligned}
        \]
        Finally, 
        \[
        \begin{aligned}
            d\omega(fX, Y) &= \lder{fX}(\omega(Y)) - \lder{Y}(\omega(fX)) - 
            \omega([fX, Y]) \\
            &= f\cdot\lder{X}(\omega(Y)) - (\lder{Y}f)\omega(X) - 
            f\cdot\lder{Y}(\omega(X)) - f\omega([X, Y]) + (\lder{Y}f)\omega(X) \\
            &= f(\lder{X}(\omega(Y)) - \lder{Y}(\omega(X)) - \omega([X, Y])) \\
            &= fd\omega(X, Y).
        \end{aligned}
        \]
    \end{proof}
    \begin{corollary}[Local Nature of Exterior Derivative]\label{cor:local-exterior-derivative}
        When \(\omega\) is fixed, the value of \(d\omega\) depends only on 
        the local values of vector fields.
        \[
        d\omega(X, Y)(p) = X^\mu(p)Y^\nu(p)d\omega(\tfld{\mu}, \tfld{\nu})(p).
        \]
    \end{corollary}
    \begin{proof}
        Write \(X = X^\mu\tfld{\mu}\), noting that \(X^\mu \in 
        C^\infty(\manifold[M])\), and use \cref{thm:f-linear-exterior-derivative}.
    \end{proof}
    \newpage
    \subsection{DeRham Cohomology}
    \begin{theorem}[Twice Exterior Differential]\label{thm:ddzero}
        For all \(\omega \in \form{n}{M}, 1 \leq n \leq \dim M\), we have 
        \[
        d^2\omega=0.
        \]
    \end{theorem}
    \begin{remark}
        This means 
        \[
        \mathrm{Im}(d : \form{n-1}{M} \to \form{n}{M}) \subseteq 
        \mathrm{Ker}(d : \form{n}{M} \to \form{n+1}{M}).
        \]
        This type of structure is called a differential complex, and is common 
        in many structures.
    \end{remark}
    \begin{definition}[Closed Form]\label{def:closed-form}
        An n-form \(\omega\) is closed if \(d\omega=0\). The set of all closed 
        n-forms is denoted \(Z^n(\manifold[M])\).
    \end{definition}
    \begin{definition}[Exact Form]\label{def:exact-form}
        An n-form \(\omega\) is exact if \(\omega = d\beta\) for some 
        \((n-1)\)-form \(\beta\). The set of all exact n-forms is denoted 
        \(B^n(\manifold[M])\).
    \end{definition}
    \begin{remark}
        It is guaranteed that \(B^n(\manifold[M]) \subseteq Z^n(\manifold[M])\), 
        that is, exactness implies closure.
        But how much closed form is not exact is the study of cohomology theory.
    \end{remark}
    \begin{theorem}[Poincare's Lemma]\label{thm:poincare-lemma}
        On Euclidean space \(\mathbb{R}^m\), 
        \[
        B^n(\manifold[M]) = Z^n(\manifold[M]), ~\forall n > 0.
        \]
    \end{theorem}
    \begin{definition}[DeRham Cohomology Groups]\label{def:derham-groups}
        The DeRham cohomology groups \(H^n(\manifold[M]), 0 \leq n \leq 
        \dim \manifold[M]\) are the quotient spaces
        \[
        H^n(\manifold[M]) := Z^n(\manifold[M]) / B^n(\manifold[M]).
        \]
    \end{definition}
    \begin{remark}
        Recall the definition of quotient groups that \(H^n(\manifold[M])\) 
        consists of elements of form \(z + B^n(\manifold[M]), z \in 
        Z^n(\manifold[M])\). \\
        If all closed forms are exact, \(Z^n(\manifold[M]) \subseteq 
        B^n(\manifold[M])\), then \(H^n(\manifold[M]) \iso \{0\}\).
    \end{remark}
    \begin{theorem}[Criterion of Exact ODE]\label{thm:exact-ode}
        On the Euclidean space \(\mathbb{R}^2\), given a 1-form 
        \(\omega=\omega_1dx^1+\omega_2dx^2\). Then 
        \[
        \omega \in B^1(\mathbb{R}^2) \iff \tfld{2}\omega_1 = \tfld{1}\omega_2.
        \]
    \end{theorem}
    \begin{remark}
        This is an important theorem to me, for it connects the "exactness of 
        differential forms" to the familiar notion of "exactness of 
        differential equations". \\
        It also provides the first hints that we are 
        actually integrating forms, and that exterior differentiation of a 
        0-form resembles gradient in usual vector calculus terms.
    \end{remark}
    \begin{proof}
        Via Poincare lemma \cref{thm:poincare-lemma}, on \(\mathbb{R}^2\), 
        exactness is equivalent to closure. So we need only to determine the 
        condition that \(d\omega=0\). Using 
        \cref{def:exterior-derivative}, 
        \[
        \begin{aligned}
            d\omega &= \tfld{\nu}\omega\indices{_{\mu_1}}dx^\nu 
            \wedge dx^{\mu_1} \\
            &= \tfld{2}\omega_1 dx^2\wedge dx^1 + 
            \tfld{1}\omega_2 dx^1\wedge dx^2 \\
            &= \left(\tfld{2}\omega_1-\tfld{1}\omega_2\right)dx^2\wedge dx^1.
        \end{aligned}
        \]
    \end{proof}
\end{document}