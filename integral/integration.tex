\documentclass[../main.tex]{subfiles}

\begin{document}
    \section{Integration of Differential Forms}
    \subsection{Partition of Unity}
    \begin{definition}[Support]\label{def:support}
        Let \(X\) be a topological space, and \(f : X \to \mathbb{R}\). Then 
        the support of \(f\) is defined as
        \[
        \supp f := \Set{x \in X|f(x) \neq 0}.
        \]
    \end{definition}
    \begin{theorem}[Partition of Unity]\label{thm:manifold-partition-of-unity}
        Let \(\manifold[M]\) be a \(C^\infty\) manifold with dimension \(m\) with 
        atlas \(\Phi\). \\
        Let \(\Phi = \Set{\phi_j|\phi_j : V_j \to \phi_j(V_j), j \in J}\).\\
        Then it is possible to construct a set of \(C^\infty\) functions 
        \(\rho_j, j \in J\) s.t. 
        \[
        1 = \sum_{j \in J}^{} \rho_j,\; \supp \rho_j \subseteq V_j
        \]
    \end{theorem}
    \subsection{Orientation}
    \subsubsection{Definition}
    \begin{definition}[Compatible Coordinate Charts]\label{def:compatible-coord-charts}
        Given a manifold \(\manifold[M]\), its two coordinate charts are called 
        compatible (have the same orientation) if, 
        \[
        \det J > 0.
        \]
        Where \(J\) is the Jacobian matrix \cref{def:jacobian-matrix}.\\
        If the manifold has a maximal compatible atlas, then we say the manifold is 
        orientable, and we may call its corresponding 
        orientation positive and denote the atlas \(\Phi_+\).
    \end{definition}
    \begin{theorem}
        A manifold has either no orientation (any atlas is not compatible) or 
        two orientations.
    \end{theorem}
    \begin{theorem}[Orientability and Existence of Forms of Highest Degree]\label{thm:volume-form-orientability}
        A manifold is orientable iff there exists a nowhere vanishing differential 
        form of the highest degree.
    \end{theorem}
    \subsubsection{Positively Oriented Boundary}
    \begin{definition}[Positively Oriented Boundary]\label{def:positive-boundary}
        Let \(\manifold[M]\) be a orientable \(C^\infty\) manifold with dimension 
        \(m\), positively oriented by compatible atlas \(\Phi_+\). Define coordinate 
        charts on \(\partial \manifold[M]\) from \(\Phi\) as follows, 
        \[
        \phi^{\partial \manifold[M]} : U_\phi \cap \partial \manifold[M] 
        \to \mathbb{R}^{m-1}, 
        \]
        Then \(\Phi^{\partial \manifold[M]}_+ := 
        \Set{\phi^{\partial \manifold[M]}| \phi \in \Phi_+}\) determines an 
        orientation on \(\partial \manifold[M]\), called the positive orientation.
    \end{definition}
    \subsection{Pseudoforms}
    \begin{definition}[Pseudoforms]\label{def:pseudoform}
        A \(C^\infty\) pseudo \(k\)-form \(\pseudoform{\omega} \in 
        \tilde{\Lambda}^k(\manifold[M])\) consists of a 
        family of differential \(k\)-forms \\
        \(\omega_\phi \in \Lambda^k(\phi(U_\phi))\), 
        \(\phi \in \Phi\), \(\phi : U_\phi \to V_\phi = \phi(U_\phi) \subseteq 
        (-\infty, 0] \times \mathbb{R}^{m-1}\), with an additional requirement that 
        \[
        \left.\omega_{\phi'}\right|_{\phi'(U_\phi \cap U_{\phi'})} = 
        (\sgn \det J)
        \left(\phi\circ{\phi'} ^{-1}\right)^*
        \left(\left.\omega_\phi\right|_{\phi(U_\phi \cap U_{\phi'})}\right),
        \; \forall \phi, \phi' \in \Phi,
        \]
        where \(\sgn \det J\) denotes the sign of Jacobian determinant.
    \end{definition}
    \begin{remark}
        The transformation rule differ from usual forms only in the choice of sign. \\
        One cannot pullback a pseudoform.
    \end{remark}
    \newpage
    \subsection{Integration of Forms of Highest Degree}
    \begin{definition}[Integration]\label{def:top-form-integration}
        Let \(\manifold[M]\) be a paracompact \(C^\infty\) manifold of dimension 
        \(m\). 
        Choose a \(C^\infty\) partition of unity \(\rho_j, j \in J\) of 
        \(\manifold[M]\) s.t. \(\supp \rho_j \subseteq U_{\phi_j} := U_j\). \\
        Let a pseudo-\(m\)-form \(\pseudoform{\omega} \in 
        \tilde{\Lambda}^m(\manifold[M])\) has local expression 
        \(\tilde{\omega}_{\phi_j} = f_j(x)dx^1_j\wedge \dots\wedge dx^m_j\), we say 
        \[
        \int_{\manifold[M]} \tilde{\omega} = 
        \sum_{j \in J}^{} \int_{(-\infty, 0] \times \mathbb{R}^{m-1}}
        (\rho_j \circ \phi_j ^{-1})(x) f_j(x) dx^1\dots dx^m
        \]
        if the finite sum exists and has the same value for all choices of 
        \(\rho_j\) and \(\phi_j\).
    \end{definition}
    \begin{remark}
        The following theorem reveals why we integrate pseudoforms, not usual forms.
    \end{remark}
    \begin{theorem}[Criterion of Existence of Integral]\label{thm:cpt-form-integral-exist}
        If \(\supp \tilde{\omega}\) is compact, then 
        \(\int_{\manifold[M]} \tilde{\omega}\) exists.
    \end{theorem}
    \begin{proof}
        Let two sets of coordinate charts be 
        \[
        \begin{aligned}
            \phi_j &: U_j \to V_j, j \in J \\
            \phi_k'&: U'_k \to V'_k, k \in K.
        \end{aligned}
        \]
        And cooresponding partition of unity be \(\rho_j, \rho'_k\).\par
        (The goal) Show
        \[
        \begin{aligned}
            &\sum_{j \in J}^{} \int
            (\rho_j \circ \phi_j ^{-1})(x) f_j(x) dx^1\dots dx^m \\
            =&\sum_{k \in K}^{} \int
            (\rho'_k \circ {\phi'_k} ^{-1})(x') f'_k(x') {dx'}^1\dots {dx'}^m
        \end{aligned}
        \]
        \par
        (Split using \(\rho_k'\))
        \[
        \begin{aligned}
            \int_{\manifold[M]} \omega &= 
            \sum_{j \in J}^{} \int
            (\rho_j \circ \phi_j ^{-1})(x) f_j(x) dx^1\dots dx^m \\
            &= \sum_{j \in J}^{} \int \sum_{k \in K}^{}
            (\rho'_k\circ \phi_j ^{-1})(x)(\rho_j\circ \phi_j ^{-1})(x) f_j(x)
            dx^1\dots dx^m.
        \end{aligned}
        \]
        Since the sum is finite, and \(\supp\omega\) is compact, and therefore 
        the integral is not improper; thus, there can be no limit or Fubini 
        problems on exchanging sums and integrals. So 
        \[
        \begin{aligned}
            \int_{\manifold[M]} \omega
            &= \sum_{j \in J}^{}\sum_{k \in K}^{} \int 
            (\rho_j\rho'_k\circ \phi_j ^{-1})(x)f_j(x)
            dx^1\dots dx^m \\
        \end{aligned}
        \]
        \par
        (Change of variables)
        First fix \(j, k\).
        \[
        \begin{aligned}
            &\int (\rho_j\rho'_k\circ \phi_j ^{-1})(x)f_j(x) dx^1\dots dx^m \\
            =&\int (\rho_j\rho'_k\circ \phi_j ^{-1})(\phi_j\circ{\phi_k'}^{-1}(x'))
            f_j(\phi_j\circ{\phi_k'}^{-1}(x')) 
            \left|\det \left(\frac{\partial x}{\partial x'}\right)\right|
            {dx'}^1\dots {dx'}^m \\
            =&\int (\rho_j\rho'_k\circ{\phi_k'}^{-1}(x'))
            f_j(\phi_j\circ{\phi_k'}^{-1}(x')) 
            \left|\det \left(\frac{\partial x}{\partial x'}\right)\right|
            {dx'}^1\dots {dx'}^m
        \end{aligned}
        \]\par
        (Use pullback requirement) From \cref{def:pseudoform},
        \[
        \omega_{\phi'_k} = (\sgn\det J)(\phi_j\circ {\phi'_k}^{-1})^*\omega_{\phi_j}, 
        \]
        we see
        \[
        \begin{aligned}
            &f'_j(x')dx^1\wedge \dots \wedge dx^m \\
            =&(\sgn\det J)f_j(\phi_j\circ {\phi'_k}^{-1}(x'))
            \left(\frac{\partial x^1}{\partial {x'}^{\ell_1}}{dx'}^{\ell_1}\right)
            \wedge \dots\wedge 
            \left(\frac{\partial x^m}{\partial {x'}^{\ell_m}}{dx'}^{\ell_m}\right) \\
            =&(\sgn\det J)\sum_{\sigma \in S_m}^{} f_j(\phi_j\circ {\phi'_k}^{-1}(x'))
            (\sgn \sigma)\frac{\partial x^1}{\partial {x'}^{\sigma(1)}}\dots
            \frac{\partial x^m}{\partial {x'}^{\sigma(m)}}
            {dx'}^1\wedge\dots\wedge{dx'}^m \\
            =&(\sgn\det J)f_j(\phi_j\circ {\phi'_k}^{-1}(x'))
            \det \left(\frac{\partial x}{\partial x'}\right)
            {dx'}^1\wedge\dots\wedge{dx'}^m \\
            =&f_j(\phi_j\circ {\phi'_k}^{-1}(x'))
            \left|\det \left(\frac{\partial x}{\partial x'}\right)\right|
            {dx'}^1\wedge\dots\wedge{dx'}^m
        \end{aligned}
        \]
        Therefore, the integral 
        \[
        \begin{aligned}
            &\int (\rho_j\rho'_k\circ{\phi_k'}^{-1}(x'))
            f_j(\phi_j\circ{\phi_k'}^{-1}(x')) 
            \left|\det \left(\frac{\partial x}{\partial x'}\right)\right|
            {dx'}^1\dots {dx'}^m \\
            =&\int (\rho_j\rho'_k\circ{\phi_k'}^{-1}(x'))
            f'_j(x')
            {dx'}^1\dots {dx'}^m
        \end{aligned}
        \]\par
        (Closing) By moving the sum wrt \(j \in J\) into the integral and using 
        the property of partition of unity, the proof is completed.
    \end{proof}
    \subsection{Integration of Forms of Lower Degree}
    \subsubsection{Definition}
    \begin{definition}[Integration of Lower Degree Forms]\label{def:pullback-integral}
        Let \(\manifold[Z]\) be an oriented \(C^\infty\) manifold of dimension \(d\), 
        \(f : \manifold[Z] \to \manifold[M]\) be a \(C^\infty\) map to a \(C^\infty\) 
        manifold \(\manifold[M]\) of dimension \(m\). \\
        Let \(\omega \in \form{d}{M}\), we define 
        \[
        \int_{\manifold[Z]}\omega := \int_{\manifold[Z]}^{} f^*\omega
        \]
        using the positive orientation of \(\manifold[Z]\), if it latter exists 
        and the pullback function \(f\) is clear from context.
    \end{definition}
    \subsubsection{Stoke's theorem}
    \begin{theorem}[Stoke's Theorem]\label{thm:stokes-theorem}
        If \(\manifold[M]\) is an oriented \(C^\infty\) manifold of dimension \(m\) 
        and \(\omega \in \form{{m-1}}{M}\), then 
        \[
        \int_{\manifold[M]}^{}d\omega = \int_{\partial \manifold[M]}^{}\omega
        := \int_{\partial \manifold[M]}^{} i^*\omega,
        \]
        where \(i : \partial \manifold[M] \to \manifold[M]\) is just the immersion 
        map, \(i : p \mapsto p\).
    \end{theorem}
    \begin{proof}
        (Partition Using Charts) Choose a \(C^\infty\) partition of unity 
        \cref{thm:manifold-partition-of-unity} \(\rho_j, j \in J\) s.t. 
        \(\supp \rho_j\) are compact and \(\supp \rho_j \subseteq U_{\phi_j} 
        := U_j\). \par
        Now \(\omega = \sum_{j \in J}^{} \rho_j \omega\) is a finite sum. So 
        it suffices to show that, if \(\eta \in \form{m-1}{M}\), \(\supp \eta\) 
        compact and \(\supp\eta \subseteq U_\phi\) then \(\int_{\manifold[M]}^{}
        d\eta = \int_{\partial \manifold[M]}^{} \eta\), and apply 
        \(\eta = \rho_j\omega\) for all \(j \in J\). \par
        (The integral)
        Suppose the coordinates of \(\phi\) is labeled \(x^1, \dots, x^m\).
        % Since \(\supp \eta \subseteq U_\phi\), we will make no distinction between 
        % \(\eta \in \form{m-1}{M}\) and \(\eta_\phi \in \eucForm{m-1}{m}\). \par
        Locally, let 
        \[
        \eta = \sum_{\ell=1}^{m} f_\ell dx^1\wedge \cdots \cancel{dx^\ell} \cdots 
        \wedge dx^m,
        \]
        where \(f_\ell : (-\infty, 0] \times \mathbb{R}^{m-1} \to \mathbb{R}\), and 
        it is \(C^\infty\). Then, also locally, 
        \[
        \begin{aligned}
            d\eta &= \sum_{\ell = 1}^{m} \frac{\partial f_\ell}{\partial x^\ell} dx^\ell 
            \wedge dx^1\wedge \cdots \cancel{dx^\ell} \cdots \wedge dx^m \\
            &= \left(\sum_{\ell=1}^{m} (-1)^{\ell-1}
            \frac{\partial f_\ell}{\partial x^\ell}\right)dx^1\wedge \dots \wedge dx^m.
        \end{aligned}
        \]
        Then by \cref{def:top-form-integration}, 
        \[
        \int_{\manifold[M]}^{}d\eta = \int_{(-\infty, 0]\times \mathbb{R}^{m-1}}^{} 
        \left(\sum_{\ell=1}^{m} (-1)^{\ell-1}
        \frac{\partial f_\ell}{\partial x^\ell}\right)dx^1\dots dx^m.
        \]\par
        (To be precise, we need to choose another partition of unity \(\rho_j'\) to 
        do intergration. But we can just choose it to cover all of \(\supp \eta\) 
        and don't care all other parts, so that doesn't matter too much.)\par
        Choose a rectangular region \(R\) s.t. 
        \[
        \supp \eta \subseteq [a_1, 0]\times \dots \times [a_m, b_m]
        \]
        and define 
        \[
        R_\ell := [a_1, 0] \times \dots \cancel{[a_\ell, b_\ell]} \dots 
        \times [a_m, b_m].
        \]\par
        (\(\ell = 2,\dots, m\)) In this case, by Fubini and FTC, 
        \[
        \begin{aligned}
            &\int_{R}^{}\frac{\partial f_\ell}{\partial x^\ell}dx^1\dots dx^m\\
            =&\int_{R_\ell}^{}\left(\int_{a_\ell}^{b_\ell} 
            \frac{\partial f_\ell}{\partial x^\ell}dx^\ell\right)dx^1\dots
            \cancel{dx^\ell}\dots dx^m \\
            =&\int_{R_\ell}^{}
            (\underbrace{f_\ell(x^1, \dots, b_\ell, \dots, x^m)}_{0}
             - \underbrace{f_\ell(x^1, \dots, a_\ell, \dots, x^m)}_{0})
            dx^1\dots \cancel{dx^\ell}\dots dx^m \\
            =&0, 
        \end{aligned}
        \]
        since \(\supp\eta \subseteq R\), so on the boundary \(f=0\). \par
        (\(\ell = 1\)) Now the integral has only one term left.
        \[
        \begin{aligned}
            \int_{\manifold[M]}^{} d\eta &= 
            \int_{R}^{} \frac{\partial f_1}{\partial x^1}dx^1\dots dx^m \\
            &=\int_{R_1}^{} \left(\int_{a_1}^{0}\frac{\partial f_1}{\partial x^1}
            dx^1\right)dx^2\dots dx^m \\
            &=\int_{R_\ell}^{}
            (f_1(0, x^2, \dots, x^m) - \underbrace{f_1(a_1, x^2, \dots, x^m)}_{0})
            dx^2\dots dx^m \\
            &=\int_{\mathbb{R}^{m-1}}^{} (f_1\circ i) dx^2\dots dx^m \\
            &=\int_{\partial \manifold[M]}^{} i^*\eta.
        \end{aligned}
        \]
    \end{proof}
\end{document}